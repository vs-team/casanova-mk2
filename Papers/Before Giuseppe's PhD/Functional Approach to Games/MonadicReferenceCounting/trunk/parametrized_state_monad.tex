%----------------------------------------------------------------------------
%  parametrized_state_monad.tex 
%----------------------------------------------------------------------------
We now move to a more powerful definition of the state monad, the parametrized state monad \cite{1_7}. This new version of the monad allows statements to make (static) changes to the state type, rather than just (dynamic) changes to the state value. The parametrized state monad has the following type:

\begin{lstlisting}
type St p q a = p -> (a,q)
\end{lstlisting}

The definition of a counter can now take advantage of the knowledge that all the proxies f a point to the same type of storage (lists, lists of lists, maps, arrays, etc. as seen in Reference implementation), and a value with the storage type must be present in the state whenever we manipulate a proxy:

\begin{lstlisting}
class Reference f a where
  Storage f a :: *
  emptyStorage :: Storage f a
  new :: (s_a ~ Storage f a, Addable s_a s, s_a $\in$ s_a .+ s) => a -> St s (s_a .+ s) (f a)
  incr :: (s_a ~ Storage f a, s_a $\in$ s) => f a -> s -> s
  decr :: (s_a ~ Storage f a, s_a $\in$ s) => f a -> s -> s
  get :: (s_a ~ Storage f a, s_a $\in$ s) => f a -> St s s a
  count :: (s_a ~ Storage f a, s_a $\in$ s) => f a -> St s s Int
\end{lstlisting}

In particular Storage f a is the type function that associates the type of proxies with the type of actual containers. We also define the value of the empty storage with the property emptyStorage. The new function requires that the appropriate storage gets added to the input type of the state; the (idempotent) addition of an element to a heterogeneous list \cite{4_1,4_2} is the .+ operator. The added item must be available in the resulting type, and to ensure this we use the $\in$ predicate. incr and decr both require that the storage is available in the manipulated state (otherwise no incrementing and decrementing could happen because there would be no "slot" to perform the computation in). Similarly we define get and count.
We also define a convenient type-class for types with a default value; this way we can define the default value of a storage as its emptyStorage:

\begin{lstlisting}
class Default x where
  default :: x

instance Reference f a => Default (Storage f a) where
  default = emptyStorage
\end{lstlisting}

 Now we can fill the gaps of the new definition of the Counter type class. A type x can be added to a type s (the result is s .+ x) if it respects the Addable predicate:

\begin{lstlisting}
class Addable x s where
  s .+ x :: *
  add :: s -> s .+ x
\end{lstlisting}

A type x is part of the heterogeneous list s if it respects the $\in$ predicate; this predicate has two instances with respect to the heterogeneous lists constructor (.:):

\begin{lstlisting}
class x $\in$ s where
  lift :: (x -> x) -> (s -> s)

instance x $\in$ x .: s
  lift f = \(x .: s) -> (f x) .: s
instance x $\in$ s => x $\in$ y .: s
  lift f = \(y .: s) -> y .: (lift f s)
\end{lstlisting}

When we wish to add a type x to another type s, we need to check if x$\in$s; if this is the case, then the addition is simply the identity with respect to s (x is already in s). If x?s then the addition returns a heterogeneous list with x as the head and s as the tail, and the value of the head is the default value of x:

\begin{lstlisting}
instance (Default x, x $\in$ s) => Addable x s where
  s .+ x = s
  add = id

instance (Default x) => Addable x s where
  s .+ x = x .: s
  add s = default .: s
\end{lstlisting}

At this point we can define the "regular" binding operator. When binding we need to be able to decrement the bound value of type f a inside the final state, which in our case has type r. For this reason we require that Storage f a$\in$r, so that we will be able to lift the decrementing operation from Storage f a->Storage f a into r->r:

\begin{lstlisting}
(>>=) :: (Reference f a, Storage f a $\in$ r) => St p q (f a) -> (f a -> St q r b) -> St p r b
s >>= k = \p ->
  let x,q = s p
  let y,r = k x q
  in y,lift (decr x) r
\end{lstlisting}

We omit the adaptation of trampolines to the parametrized monad as it is relatively straightforward.

As a side note, it is worth realizing that a big part of the above code, especially the $\in$ predicate, will not work in current incarnations of the Haskell language and will rather give problems that can be solved (albeit in a rather verbose fashion) as seen in the \cite{4_1}. Seen that we have used the above definitions to implement our own custom meta-programming library in F\#, this has not been a problem for us.
