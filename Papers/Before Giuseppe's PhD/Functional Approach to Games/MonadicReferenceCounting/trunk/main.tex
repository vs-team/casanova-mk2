\documentclass[a4paper]{article}
\usepackage{anysize}
\marginsize{1cm}{1cm}{1cm}{1cm}

\usepackage{amssymb}
\usepackage{amsmath}
\usepackage{graphicx}
\usepackage{epsfig}
\usepackage{subfigure}
\usepackage{listings}
\usepackage{natbib}
\usepackage{verbatim}
\usepackage[T1]{fontenc} 
\lstset{language=haskell}
\lstset{commentstyle=\textit}
\lstset{mathescape=true}
\lstset{backgroundcolor=,rulecolor=}
\lstset{basicstyle=\ttfamily}
%\linespread{2.0}

\begin{document}

\title{\bf Monadic Reference Counting}

\author{Giuseppe Maggiore \quad
		  Michele Bugliesi
 \\ Universit\`a Ca' Foscari Venezia
 \\ Dipartimento di Informatica 
 \\ \{maggiore,bugliesi\}@dsi.unive.it
}

\date{}
\maketitle

\begin{abstract}
In this paper we show how a powerful memory and resource management technique such as reference counting can be implemented transparently as a library through the use of monads. While this is interesting in itself, since garbage collectors are not always the best choice for all programs, our paper also shows how the bind and return operators can be used to correctly identify the lifetime of resources and references. Finally, we discuss how with a powerful enough type system and the use of a parameterized monad we can track various interesting properties about the state of stateful computations. In our case we track the set of resource-types that the program handles, but we argue that in other cases this technique could be used to track even more complex and interesting properties.
\end{abstract}

\section{Introduction}
\label{sec:intro}
%----------------------------------------------------------------------------
%  intro.tex 
%----------------------------------------------------------------------------
Modern computer languages are very reliable when it comes to writing a large class of common, real-world applications. For example, relatively simple form applications or web sites can be built extremely easily in languages such as Java, C\# and many others. This is thanks to commonplace facilities like garbage collectors, classes and inheritance and large libraries which simplify many tasks which otherwise would be hard or error-prone. On the other hand, there is a not so small set of applications for which these languages do not perform even nearly as well; for example games, even though very powerful libraries such as XNA make them easier to write by encapsulating many useful patterns, are not so suitable for modern languages. For this reason most games are still written in C++ (sometimes even in C) and the transition to higher level languages is not happening as fast as it could. As another example we could consider mobile applications. The widespread adoption of very powerful, fully programmable smartphone like the iPhone, Google Android or Windows Phone 7 makes performance even more important to achieve: lighter applications mean much better applications where CPU cycles and battery are both scarce resources. On the other hand, to allow as many developers as possible to easily create applications for these platforms, it makes sense (as indeed it is happening) to allow programming these devices with as languages that are as high-level as possible. Finally, there are many real time or soft real time applications that migh benefit from using high level languages but which cannot afford the pauses or slowdowns that sometimes the garbage collector might require, especially on less powerful hardware.


In this paper we document the results of implementing a videogame in such a high-level language (F\#) for a mobile device. Early in our development cycle we discovered that the biggest problem of our application was that by continually allocating and deallocating instances of the same types (such as projectiles and particles) we kept triggering the garbage collector, forcing the application to a crawl everytime it started. Rather than implement object pooling, we decided to try and generalize this technique by implementing reference counting \cite{7_6} inside the state monad \cite{1_1}. The state monad contains and manages all the actual instances of the objects which require reference counting, and references to these objects can only be accessed through the state monad itself. This allowed us to track the lifetime of our entities, in a lightweight enough fashion to ensure the increase in performance that we needed and transparently enough that it became convenient to track other resource types as well, such as files or GPU memory which require manual disposal.


In this paper we discuss a possible generalization of the work described above, where in addition to a monadic reference counting system we also discuss how type-level meta-programming \cite{3_1,3_2,3_3,6_1} and the parameterized monad \cite{1_7} can be used to track the types of resources with a strongly typed heterogeneous list \cite{4_1}, thereby removing even the need for the developer to "discover" what type the state must have. Also, this system could be the stepping stone for a more refined state-tracking monad which uses type-level values and phantom types to track expected properties of the state monad at various points in the program.


An important note is about the computer language in which we have written the samples. We have used a pseudo-Haskell, and we believe our listings may be turned into a working program without too much effort. This notwithstanding, it must be noted that said effort has not been made by us: our software system is mostly interested in performance, and from the point of view of benchmarking performance Haskell may not be the best suited language given its lazy evaluation strategy. For us this pseudo-Haskell has acted as a clear and easily implementable specification, and we do not claim anything about its workability in any present or future versions of the language.
 
 

\section{State Monad}
\label{sec:state_monad}
%----------------------------------------------------------------------------
%  state_monad.tex 
%----------------------------------------------------------------------------
The state monad \cite{1_1,1_2,1_3,1_4,1_5} is a monad which allows us to write stateful, imperative computations in a pure language without sacrificing purity. A value of state monad type represents a statement, and its type requires that a value of type \emph{state} is passed as a parameter to the statement in order for it to evaluate to its result. Evaluating a statement not only returns the result of the evaluation, but also the new value of the state: the input state may have been modified, that is the evaluation of any statement may produce side-effects.

The type of the state monad reminds the type of the denotational semantics of an imperative statement. This is interesting, in that we could consider values of the state monad as the denotation of statements:

\begin{lstlisting}
type St s a = s -> (a,s)
\end{lstlisting}

We can bind statements together into composite statements and return values inside statements. When binding, we concatenate the two statements by evaluating the first and then plugging its result into the second and then evaluating it:

\begin{lstlisting}
(>>=) :: St s a -> (a -> St s b) -> St s b
p >>= k = \s ->
  let y,s' = p s
  in k y s'
\end{lstlisting}

When we wish to pack a value x inside a monad we return it:

\begin{lstlisting}
return :: a -> St s a
return x = \s -> x,s
\end{lstlisting}
 

\section{Reference}
\label{sec:reference}
%----------------------------------------------------------------------------
%  reference.tex 
%----------------------------------------------------------------------------
The state monad as presented above is well-known and commonly used in pure languages such as Haskell. This kind of world- passing-style (or store-passing-style) is powerful and allows a programmer to write perfectly fine imperative code. Since this monad is mostly (if not always exclusively) used to pass around the state and manipulate mutable portions of the state through the use of references and proxies, we now propose an extension of the monad that focuses exclusively on these references.
A system resource is anything that we wish to release as soon as we are done with it. System resources may include streams, network connections, GPU memory in GPGPU applications (as done, for example, in \cite{9_1,9_2}), threads, etc. In some cases even memory references may be treated as resources, especially when we wish to recycle memory as soon as possible rather than just leave the job to the garbage collector. A system resource can be defined in terms of an appropriate type class:

\begin{lstlisting}
class Reference f a s where
  new :: a -> St s (f a)
  incr :: f a -> St s ()
  decr :: f a -> St s ()
  get :: f a -> St s a
  count :: f a -> St s Int
\end{lstlisting}

A Reference is represented by a functor f and a type a. a is the type of our actual resource, and f a is the proxy that we will use to access this resource. The proxy f a may:
\begin{itemize}
\item be created from a value a with new
\item increment its internal counter with incr
\item decrement its internal counter with decr
\item get its internal value with get
\item get its internal counter with count
\end{itemize}

The only public functions that we leave accessible are new and get. The other functions can only be called from inside the monad implementation.

\subsection{Reference axioms}
A Reference has some requirements that it must respect. These requirements are expressed in terms of axioms, which are a way to formalize the most obvious expectations we have towards a Reference. Informally, we require that a Reference:
\begin{itemize}
\item starts with a count of 1 after being created with new
\item returns a value with get when its count is greater than 0; otherwise get returns $\uparrow$
\item incr and decr respectively increment and decrement the internal count by 1, but only if the count is greater than 0; otherwise they both return ?
\end{itemize}

We can express these requirements with the help of the do-notation as:

\begin{lstlisting}
forall s x . (do r <- new x
                 count r) s = (1,_)

forall s r . (get r) s = (_,_) if (count r s) = (c,_) with c > 0
                         $\uparrow$ otherwise

forall s r . (do incr r
            count r) s = (c+1,_) if (count r s) = (c,_) with c > 0
                         $\uparrow$ otherwise

forall s r . (do decr r
                 count r) s = (c-1,_) if (count r s) = (c,_) with c > 0
                              $\uparrow$ otherwise
\end{lstlisting}

We also expect, but this is really up to the implementation, that whenever the internal count of a resource is decremented to 0, then the resource will be freed.

\subsection{Possible reference implementation}
Our definition of Reference may appear a bit excessive. What is the meaning of the functor f? f is the type of references, which will be used as proxies of the actual values of type a. The idea behind the Reference type class is that we do not force anything about how references are represented: the type of references depends strongly on the type of the heap inside which the reference will point to. To strengthen this intuition, let us discuss a few possible implementations.
A heap may be a list of values where we add values to the end of the list; references are the index from the beginning of the list to the appropriate item and its associated counter:

\begin{lstlisting}
type F a = Int
instance Reference F a [(Int,a)]
  �
\end{lstlisting}

A heap may be a list of lists of values, for performance reasons (indexing twice helps skipping many elements):

\begin{lstlisting}
type F a = (Int,Int)
instance Reference F a [[(Int,a)]]
  ...
\end{lstlisting}

A heap may even be a more elaborate data structure, such as a map from a key k into a value a and its associated counter:

\begin{lstlisting}
instance (Map s, k ~ Key s) => Reference k a (Map k (Int,a))
  ...
\end{lstlisting}

where Map s means that s is a map and Key s is a type function that returns the type of key used to index elements of s; ~ is the binding operator for type variables.

The above definitions all define mappings from keys to values that are all pure and quite obvious. Since we are not limiting our language to a pure functional one (the thing will have to run on imperative hardware after all) it is not at all inadmissible that the implementation of the heap and its reference may somehow rely on pointers and mutable state. A simple yet effective implementation only requires that our language supports arrays. References are indices in the array, but this time the access will be much faster:

\begin{lstlisting}
type F a = Int
instance Reference F a [|(Int,a)|]
  ...
\end{lstlisting}

or

\begin{lstlisting}
type F a = Int
instance Reference F a [|[|(Int,a)|]|]
  �
\end{lstlisting}

where [ | a | ] is an array of a. We could optimize this last definition further (it has very high performance and is very easy to implement, and as such we have used it in our benchmarks) by adding to each element of the external array the number of free items (those with the counter set to 0):

\begin{lstlisting}
type F a = Int
instance Reference F a [|(Int,[|(Int,a)|])|]
  ...
\end{lstlisting}

Of course whatever implementation we pick for managing references, we must keep in mind that incrementing (decrementing) a reference to a value also requires us to inductively increment (decrement) all the references contained inside that value. For this reason we modify the Reference class so that its incrementation and decrementation functions only do one step of incrementing and decrementing, while the recursive work is left to a new pair of incr and decr functions:

\begin{lstlisting}
class Reference f a s where
  new :: a -> St s (f a)
  incrStep :: f a -> St s ()
  decrStep :: f a -> St s ()
  get :: f a -> St s a
  count :: f a -> St s Int
\end{lstlisting}

We define, in the spirit of the LIGD library ([REFERENCE]), a representation datatype using GADTs (Generalized Algebraic Datatypes):

\begin{lstlisting}
data Unit = Unit
data Sum a b = Inl a | Inr b
data Prod a b = Prod a b

data Rep t where
  RUnit :: Rep Unit
  RSum :: Rep a -> Rep b -> Rep (Sum a b)
  RProd :: Rep a -> Rep b -> Rep (Prod a b)
  RType :: Rep c -> EP b c -> Rep b
\end{lstlisting}

where the EP datatype is the witness of the isomorphism between two type b and c
and is defined as:

\begin{lstlisting}
data EP b c = EP { from :: (b -> c), to :: (c -> b) }
\end{lstlisting}

As an example, let us see how we could define a generic equality function:
\begin{lstlisting}
geq :: Rep a -> a -> a -> Bool
geq (RUnit)       Unit          Unit          = True
geq (RSum ra rb ) (Inl a1 )     (Inl a2 )     = geq ra a1 a2
geq (RSum ra rb ) (Inr b1 )     (Inr b2 )     = geq rb b1 b2
geq (RSum ra rb ) _             _             = False
geq (RProd ra rb) (Prod a1 b1 ) (Prod a2 b2 ) = geq ra a1 a2 && geq rb b1 b2
geq (RType ra ep) t1            t2            = geq ra (from ep t1) (from ep t2)
\end{lstlisting}

The generic equality function takes an additional parameter which is the representation of the type of the values being compared. This allows us to index the function based on the type (in fact we say that geq is a TIF, or "type-indexed-function").

Now, consider how we could build the representation of a list. We start with the embedding projection:

\begin{lstlisting}
fromList :: [a] -> Sum Unit (Prod a [a])
fromList [] = Inl Unit
fromList (a:as) = Inr (Prod a as)
toList :: Sum Unit (Prod a as) -> [a]
toList (Inl Unit) = []
toList (Inr (Prod a as)) = a:as
\end{lstlisting}

then we write the representation using RType:

\begin{lstlisting}
rList :: Rep a -> Rep [a]
rList ra = RType (RSum RUnit (RProd ra (rList ra)))
            (EP fromList toList)
\end{lstlisting}

At this point we define two intermediate functions incrRep and decrRep that recursively invoke themselves in order to invoke incrStep and decrStep respectively on each Reference found inside the original Reference:

\begin{lstlisting}
incrRep :: Reference f a s => Rep a -> (f a) -> St s ()
incrRep (RUnit) ref = incrStep ref
incrRep (RSum ra rb) ref  = 
  do v <- get ref
     case v of 
     | Inl x -> incrRep ra x
     | Inr x -> incrRep rb x
     incrStep ref
incrRep (RProd ra rb) ref = 
  do (x,y) <- get ref
     incrRep ra x
     incrRep rb x
     incrStep ref
incrRep (RType ra ep) ref = 
  do v <- get ref
     incrRep ra (from ep v)
     incrStep ref

decrRep :: Reference f a s => Rep a -> (f a) -> St s ()
...
\end{lstlisting}

We omit the body of decrRep since it is substantially identical to that of incrRep, to the point that both functions could be easily defined in terms of a single combinatory (a monadic version of the everywhere function [REFERENCE]).

We instance a "constant" Reference so that a simple value can be interpreted as a Reference :

\begin{lstlisting}
type Id a = a
instance Reference Id a s where
  new = return
  incrStep = return ()
  decrStep = return ()
  get = id
  count = 1
\end{lstlisting}

At this point we define a typeclass that captures all the datatypes representable in terms of the above GADT:

\begin{lstlisting}
class Representable a where
  rep :: Rep a
\end{lstlisting}

We also require that a Reference is always to a Representable datatype:

\begin{lstlisting}
class Representable a => Reference f a s where
  ...
\end{lstlisting}

in order that the representation is implicit in the Reference and must not be passed around each time. At this point we can define the actual incr and decr functions:

\begin{lstlisting}
incr :: Reference f a s => f a -> St s ()
incr ref = incrRep rep ref

decr :: Reference f a s => f a -> St s ()
decr ref = decrRep rep ref
\end{lstlisting}

\begin{lstlisting}

\end{lstlisting}
 

\section{Bind, return and lifetime}
\label{sec:scoping_state_monad}
%----------------------------------------------------------------------------
%  scoping_state_monad.tex 
%----------------------------------------------------------------------------

Let us now focus on the notion of variable scope that is implicit in a monad. Whenever we bind two statements, the scope of the bound value is limited to the body of the second parameter (unless it is returned). This means that after the bound value is passed to the second parameter, then its lifetime is exhausted and the value may be decremented. Of course, whenever we return a value then to prevent its premature reclamation we will increment it to counter its decrementing by the enclosing binding.
The new type of the bind and return operators now requires that these two only manipulate monads to references. Bind will also decrement the bound value as soon as it goes out of scope:

\begin{lstlisting}
(>>=) :: Reference f a s => St s (f a) -> (f a -> St s b) -> St s b
p >>= k = \s ->
  let y,s' = p s
  let z,s'' = k y s'
  in z,snd(decr y s'')
\end{lstlisting}

while return will increment its parameter:

\begin{lstlisting}
return :: Reference f a s => f a -> St s (f a)
return x = \s -> x,snd(incr x s)
\end{lstlisting}

This is convenient because in the rest of the paper we will have no need to use those as standalone monadic statements and instead we will use them only as functions from state to state.

\subsection{General recursion}

Whenever we are in the presence of stateful recursive functions, then we may find that some embarrassing facts occur. In particular, the lifetime of a local value inside the body of the recursive function may be unnaturally lengthened to encompass the entire sub-trees of the recursive call. This is unacceptable, especially if we think about functions that recursively open a lot of files (like when traversing the file system in search for something) or use a lot of memory.


\subsubsection{Benchmark}

Let us consider a very challenging example of this scenario. We wish to create a balanced binary tree from a set of points:

\begin{lstlisting}
bt :: [Point] -> Tree [Point]
bt pts =
  if size pts < 1000 then mk_leaf pts
  else
    let m = median pts
    let l,r = split pts m
    let tl = bt l
    let tr = bt r
    in mk_node (tl,tr)
\end{lstlisting}

In this example we can clearly see that until both calls to bt l and bt r are completed, then we may not release any memory at all! This is clearly nonsense, since pts can be released right after the call to split and l can be released right after the first recursive call to bt.


\subsubsection{Explicit continuations}

Enter Continuation Returning Style. We try and solve this problem with trampolines \cite{7_7}, that is intermediate pieces of code that are "wrapped" around our recursive calls. We will call this style Continuation Returning Style because statements in this style do not return their result when executed but rather return another statement which, when executed in its turn, will complete the job. We refer to this nested statement as a trampoline. We define trampolines as statements that capture by (explicit) closure a containing Reference which gets incremented when the trampoline is created and which is released after the trampoline is executed.
Using trampolines does not exclude the possibility of using the state monad as defined above. Whenever its conservative notion of lifetime is acceptable, we will be free to use it; whenever its notion of lifetime is too restrictive, then we will use our trampolines and jump between the two easily. A trampoline is defined as:

\begin{lstlisting}
type Trmp s a = St s (St s a)
\end{lstlisting}

It may help understanding trampolines in terms of the following diagram which shows the order in which the state flows:

\begin{figure}[h]
\centerline{\psfig{file=TrmpDiagram.png,height=5cm}} 
\caption{State flow\label{fig:trampoline_diagram}}
\end{figure}

A trampoline is constructed from the captured values that will have a lifetime at least as long as that of the trampoline and the actual body of the trampoline. The first thing the trampoline constructor does is increment the captured values, and then it binds the execution of the body of the trampoline with decrementing the captured values:

\begin{lstlisting}
trmp :: Reference f a s => f a -> (f a -> Trmp s b) -> Trmp s b
trmp ctxt p = \s -> (\s ->
  let p',s' = p ctxt s
  in p' (snd (decr ctxt s'))), snd (incr ctxt s)
\end{lstlisting}

We have a return function that creates a trampoline. To make this work we assume that our language supports operator overloading, where the most specific overload will be invoked at each binding site:

\begin{lstlisting}
return :: Reference f a s => f a -> Trmp s (f a)
return x = \s->(\s -> x,snd (incr x s)),s
\end{lstlisting}

We can turn a trampoline into a statement with relative ease by unpacking it and executing it twice:

\begin{lstlisting}
(!) :: Trmp s a -> St s a
!p = \s ->
  let p',s' = p s
  in p' s'
\end{lstlisting}

Even more interesting is the fact that we can easily bind values of the state monad with trampolines;  the result will be another trampoline. This kind of binding starts decrementing the bound value y much sooner than the standard binding, since y get decremented before the full execution of k. The idea is that k has a chance to increment y as its ctxt, and right after this has been done k y returns k'. The actual execution of k is thus stored as k', which may contain recursive calls to some caller. Before executing this (possibly time-consuming) code we get our chance to free y in case it were not needed anymore inside k':

\begin{lstlisting}
(>>=) :: Reference f a s => St s (f a) -> (f a -> Trmp s b) -> Trmp s b
p >>= k = \s ->
  let y,s' = p s
  let k',s'' = k y s'
  in k', decr y s''
\end{lstlisting}

Notice that this version of the binding operator is exactly the same as the original binding operator, and since trampolines are state monads then there is no need for the explicit definition.

\subsubsection{Solving the benchmark problem}

We can rewrite the binary tree example above as:

\begin{lstlisting}
bt :: Reference f [Point] s => f [Point] -> Trmp s Tree [Point]
bt pts =
  if size pts < 1000 then trmp pts (\pts -> return mk_leaf pts)
  else
    trmp pts (\pts ->
      do m <- median pts
         trmp (pts,m) (\(pts,m) ->
           do l,r <- split pts m
              trmp (l,r) (\(l,r) -> 
              do tl <- bt l
                 trmp (tl,r) (\(tl,r) ->
                 do tr <- bt r
                    trmp (tl,tr) (\(tl,tr) ->
                      return mk_node (tl,tr))))))
\end{lstlisting}

where we can clearly see that each continuation declares explicitly what it will keep alive (in terms of reference counting). This is a similar style to the well-known world-passing-style of the state monad, but we are also passing around the active scope.

\subsubsection{Syntactic sugar for continuations}

It is very important to notice a detail that threatens the correctness of our system. Continuations may not capture values whose type respects the Reference predicate; continuations may only use those counters that they explicitly captured, otherwise we have no guarantee that the captured counter will still be valid when accessed.
For this reason we introduce a notion of syntactic sugar for expressing our continuations where the captured variables are the free variables of type Reference accessed in the body of the continuation. We tell the compiler to search for these variables with the trmp keyword (not to be confused with the private trmp function seen above).
The translation rule is quite straightforward:

\begin{lstlisting}
	[| trmp x <- m n |] = trmp ctxt (\ctxt -> m >>= fun x -> [| trmp n|])	
	where ctxt = FC(m) $\cup$ FC(n)

	[| trmp return m |] = trmp ctxt (\ctxt -> return m)
	where ctxt = FC(m)
where FC(t) = {x:f a $\in$ FV(t) : $\exists$s . Reference f a s}
\end{lstlisting}

The resulting code is:

\begin{lstlisting}
bt :: Reference f [Point] s => f [Point] -> Trmp s Tree [Point]
bt pts =
  if size pts < 1000 then trmp return mk_leaf pts
  else
    trmp m <- median pts       -- FC = pts
       l,r <- split pts m      -- FC = m,pts
       tl <- bt l              -- FC = l,r
       tr <- bt r              -- FC = r,tl
       return mk_node (tl,tr)  -- FC = tl,tr
\end{lstlisting}

and since we can easily see that:

\begin{lstlisting}
FC(trmp m <- median pts
        l,r <- split pts m 
        tl <- bt l 
        tr <- bt r 
        return mk_node (tl,tr)) = pts
FC(trmp l,r <- split pts m 
        tl <- bt l 
        tr <- bt r 
        return mk_node (tl,tr)) = (pts,m)
FC(trmp tl <- bt l 
        tr <- bt r 
        return mk_node (tl,tr)) = (l,r)
FC(trmp tr <- bt r
        return mk_node (tl,tr)) = (tl,r)
FC(return mk_node (tl,tr)) = (tl,tr)
\end{lstlisting}

then it is clear how the sample with implicit continuations becomes identical to the one with explicit continuations.

 

\section{Parametrized state monad}
\label{sec:parametrized_state_monad}
%----------------------------------------------------------------------------
%  parametrized_state_monad.tex 
%----------------------------------------------------------------------------
We now move to a more powerful definition of the state monad, the parametrized state monad \cite{1_7}. This new version of the monad allows statements to make (static) changes to the state type, rather than just (dynamic) changes to the state value. The parametrized state monad has the following type:

\begin{lstlisting}
type St p q a = p -> (a,q)
\end{lstlisting}

The definition of a counter can now take advantage of the knowledge that all the proxies f a point to the same type of storage (lists, lists of lists, maps, arrays, etc. as seen in Reference implementation), and a value with the storage type must be present in the state whenever we manipulate a proxy:

\begin{lstlisting}
class Reference f a where
  Storage f a :: *
  emptyStorage :: Storage f a
  new :: (s_a ~ Storage f a, Addable s_a s, s_a $\in$ s_a .+ s) => a -> St s (s_a .+ s) (f a)
  incr :: (s_a ~ Storage f a, s_a $\in$ s) => f a -> s -> s
  decr :: (s_a ~ Storage f a, s_a $\in$ s) => f a -> s -> s
  get :: (s_a ~ Storage f a, s_a $\in$ s) => f a -> St s s a
  count :: (s_a ~ Storage f a, s_a $\in$ s) => f a -> St s s Int
\end{lstlisting}

In particular Storage f a is the type function that associates the type of proxies with the type of actual containers. We also define the value of the empty storage with the property emptyStorage. The new function requires that the appropriate storage gets added to the input type of the state; the (idempotent) addition of an element to a heterogeneous list \cite{4_1,4_2} is the .+ operator. The added item must be available in the resulting type, and to ensure this we use the $\in$ predicate. incr and decr both require that the storage is available in the manipulated state (otherwise no incrementing and decrementing could happen because there would be no "slot" to perform the computation in). Similarly we define get and count.
We also define a convenient type-class for types with a default value; this way we can define the default value of a storage as its emptyStorage:

\begin{lstlisting}
class Default x where
  default :: x

instance Reference f a => Default (Storage f a) where
  default = emptyStorage
\end{lstlisting}

 Now we can fill the gaps of the new definition of the Counter type class. A type x can be added to a type s (the result is s .+ x) if it respects the Addable predicate:

\begin{lstlisting}
class Addable x s where
  s .+ x :: *
  add :: s -> s .+ x
\end{lstlisting}

A type x is part of the heterogeneous list s if it respects the $\in$ predicate; this predicate has two instances with respect to the heterogeneous lists constructor (.:):

\begin{lstlisting}
class x $\in$ s where
  lift :: (x -> x) -> (s -> s)

instance x $\in$ x .: s
  lift f = \(x .: s) -> (f x) .: s
instance x $\in$ s => x $\in$ y .: s
  lift f = \(y .: s) -> y .: (lift f s)
\end{lstlisting}

When we wish to add a type x to another type s, we need to check if x$\in$s; if this is the case, then the addition is simply the identity with respect to s (x is already in s). If x?s then the addition returns a heterogeneous list with x as the head and s as the tail, and the value of the head is the default value of x:

\begin{lstlisting}
instance (Default x, x $\in$ s) => Addable x s where
  s .+ x = s
  add = id

instance (Default x) => Addable x s where
  s .+ x = x .: s
  add s = default .: s
\end{lstlisting}

At this point we can define the "regular" binding operator. When binding we need to be able to decrement the bound value of type f a inside the final state, which in our case has type r. For this reason we require that Storage f a$\in$r, so that we will be able to lift the decrementing operation from Storage f a->Storage f a into r->r:

\begin{lstlisting}
(>>=) :: (Reference f a, Storage f a $\in$ r) => St p q (f a) -> (f a -> St q r b) -> St p r b
s >>= k = \p ->
  let x,q = s p
  let y,r = k x q
  in y,lift (decr x) r
\end{lstlisting}

We omit the adaptation of trampolines to the parametrized monad as it is relatively straightforward.

As a side note, it is worth realizing that a big part of the above code, especially the $\in$ predicate, will not work in current incarnations of the Haskell language and will rather give problems that can be solved (albeit in a rather verbose fashion) as seen in the \cite{4_1}. Seen that we have used the above definitions to implement our own custom meta-programming library in F\#, this has not been a problem for us.
 

\section{Related work}
\label{sec:related_work}
Currently there are many commercial and open source solutions for developing RTS games which result to be often too specific, thus inflexible or not scalable. When users would like to extend these frameworks, this often turns out to be difficult, if not impossible, unless they change the entire structure of the project by changing the structures of entities and the connections among them.
There are not many specific RTS engines but some of the most common used are listed below.
\subsubsection*{Game maker:}
Game maker is a tool which joins the visual development with a limited scripting language. The scripting language allows only the use of strings and real numbers, possibly indexed as arrays. However, it is neither possible to pass an array as a script argument nor accessing it with a pointer except by passing a string holding the name of the array itself. R.E.A. instead is an extension of a well defined and structured programming language like Casanova with no such limitations and workarounds.
\subsubsection*{ORTS – Open real-time strategy engine:}
ORTS is a domain specific language for making RTS games based on scripts. The language of the scripts is limited; what is not supported by its primitives must be written in C. However, this lack of expressiveness is compensated by being domain specific. ORTS is not designed to be very general because it is specific only for the RTS genre, and in particular for RTS's without an articulated logic. Finally, native optimization, which is provided by our solution, is not possible in ORTS, as explained above, unless the developer codes it by himself.
\subsubsection*{Spring engine:}
Spring engine is a framework for creating RTS games. The engine specifies predefined boundaries on game dynamics, which cannot be extended. The developer has to learn a long series of keywords. Moreover, getting out of the predefined context, requires to code in a different semantic level using scripting languages such as LUA. However, Spring engine is a good RTS framework which implements a wide variety of options and, in some cases, native optimization (such as spatial optimization for collision detection). Spring engine also presents the same problems, as for the scripting language, of the other engines listed above. 

\section{Conclusions and future work}
\label{sec:conclusions}
%
% conclusions.tex
%

Innovations in game architecture are often responsible for pushing further the boundaries of game design. By building more powerful \textit{and} more scriptable game engines it is possible at the same time to create games where game logic is faster and more complex, resulting in a more compelling experience for the player.

Building a system that allows for a smarter, individualized AI could usher an era of games that offer a smarter experience where units and non-playing characters make smart choices rather than waiting for ``obvious'' instructions from the player.

It must also be noted that games have been historically responsible for many evolutions of modern personal computing, by pushing the hardware to its limits and by offering powerful visual experiences that years or decades later become paradigms in non-gaming software. For this reason this kind of language evolution must be put in its right context: it is not just a way to make better games, but it is also a prototype for a way to write better \textit{programs} in general.

\paragraph{Future research directions in the field}
From a survey of the current panorama of game development tools and languages it appears that a technological shift is happening: from the almost exclusive use of C and C++ for creating games, plus the odd use of Lua or Python as scripting languages, more and more games are being made in modern managed languages such as C\# or Objective-C. The introduction of XNA (a framework that allows independent game developers to publish commercial games on the XBox 360 without costly licenses) has marked the beginning of the adoption of C\#-based frameworks for making games; other similar frameworks have been used in games between a C++ engine and the scripting system. This shift signals that the game development industry may be starting to have trouble with its old infrastructure given the rising development costs, the shrinking budgets and the decreasing development time and it is looking into new languages with more abstraction to increase its efficiency. For this reason many researchers (the authors of this report being part of this group) are studying new ways to leverage the power coming from the field of functional and declarative languages in order to more easily automate some repetitive programming tasks that often require complex and cumbersome libraries. 

\bibliographystyle{plain}
\bibliography{references} 
\cite{*}
\nocite{}

\end{document}
