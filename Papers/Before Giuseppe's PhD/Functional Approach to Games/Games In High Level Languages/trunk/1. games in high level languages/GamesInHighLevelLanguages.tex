\documentclass[a4paper]{article}
\usepackage{anysize}
\marginsize{1cm}{1cm}{1cm}{1cm}

\usepackage{amssymb}
\usepackage{amsmath}
\usepackage{graphicx}
\usepackage{epsfig}
\usepackage{subfigure}
\usepackage{listings}
\usepackage{natbib}
\usepackage{verbatim}
\usepackage[T1]{fontenc} 
\lstset{language=haskell}
\lstset{commentstyle=\textit}
\lstset{mathescape=true}
\lstset{backgroundcolor=,rulecolor=}
\lstset{basicstyle=\ttfamily}
%\linespread{2.0}

\begin{document}

\title{\bf Building games with high-level languages}

\author{Giuseppe Maggiore \quad
		  Renzo Orsini
		  Michele Bugliesi
 \\ Universit\`a Ca' Foscari Venezia
 \\ Dipartimento di Informatica 
 \\ \{maggiore,orsini,bugliesi\}@dsi.unive.it
}

\date{}
\maketitle

\begin{abstract}
Video games are a fascinating and challenging niche of Computer Science. While games spark a strong fascination in many people, especially among the technical-savy folks, games also present a unique blend of complex challenges. Games require careful usage of the available hardware resources, since they must run quickly enough in order to provide a sufficiently interactive and fluid environment. Also, drawing a realistic scene and special effects in real-time is quite resource-intensive. On the opposite end of the spectrum, games have very complex logic: from physics simulations to animate the in-game-world to AI for the non-playing-characters to networking code for multiplayer games. In short, games are one of the most challenging areas of software development and software engineering, where a strong necessity for performance is more often than not at odds with a further need for the abstraction that only the most high-level languages and frameworks seem capable of offering. In this paper we make the case that high-level languages such as modern functional languages offer two distinct advantages:
\begin{itemize}
\item they allow us to split the logic and the renderind in two threads with no synchronization for greater performance
\item they allow us to define the operations on sequences of data in a much more readable form, without 
\item they allow us to define finite-state-machines (an extremely common case in games) with great ease
\item they allow us to abstract complex patterns for implementing customizable logic in games (the so called ``scripting")
\end{itemize}
We will show various examples and benchmarks to document our claims that a very high-level language does not only make code more readable but also more easily parallelizable and thus faster.
\end{abstract}

\section{Introduction}
\label{sec:intro}
%%%%%%%%%%%%%%%%%%%%%%%%%%%%%%%%%%%%%%%%%%%%%%%%%%%%%%%%%%
% intro.tex
%%%%%%%%%%%%%%%%%%%%%%%%%%%%%%%%%%%%%%%%%%%%%%%%%%%%%%%%%%

List processing libraries are ubiquitous [] in functional languages. Virtually any functional program will employ either lists, sets, maps, lazy sequences or some variations and with those the accompanying higher-order-functions such as map, filter, reduce, etc. []

These libraries allow for shorter, cleaner and more readable code []. Unfortunately, imperative code which ``condensates" various chained operations can very easily be turned into a faster equivalent than its functional counterpart, even though this comes at the cost of conciseness and readability []. This leads to a very undesirable dicotomy between good code and fast code []: no such choice should ever exist!

In this paper we discuss how list processing libraries can be augmented with static information about the shape of the chain of operations performed on the original data source. We use these augmentations to build a static optimizer capable of automatically turning chains of list processing operations into faster equivalents.

To achieve this result we will use the most historically renowned [] set of knowledge in the field of accessing and transforming data: relational algebra. We observe the strict correspondence between most list processing higher order functions and well-known, studied and understood relational operators.

Our technique can be described as follows: list processing operators do not simply return lists; rather, list processing operators return values that can be converted into lists but which type can be used to describe exactly what kind of operation will be performed on the source data. Static operators will then build appropriate optimization and execution functions specifically tailored at compile time for the shape of each operation.

We will use a Haskell-like syntax for describing, optimizing and executing queries. Overlapping instances are so common in our case that we do not claim that our code is actually Haskell, if not for the apparently identical syntax when describing type classes. The semantics of our ``imaginary" compiler and runtime are best described in [Mark Jones]: type classes are logical clauses that are resolved at compile-time through back-tracking. This makes our type system capable of computing essentially any transformation that we may wish for on our code.

We argue that this kind of use of types and metaprogramming is the future of many important compiler optimizations. Until now compiler optimizations have been monolithically embedded in compilers and have offered very little customization opportunities (most of the time just a few compiler switches are offered: hardly any customization at all). Thanks to the emergence of powerful metaprogramming mechanism such as type classes (with overlapping instances handled gracefully) we can finally start building libraries that can be executed completely at compile-time and which realize lots of poweful optimizations, from memory management and reference counting [] to algorithmic optimizations such as the one presented in this paper and even up to analysis of one's source code with abstract interpretation and similar approaches.
 

\section{Plain imperative implementation}
\label{sec:imperative_implementation}
%%%%%%%%%%%%%%%%%%%%%%%%%%%%%%%%%%%%%%%%%%%%%%%%%%%%%%%%%%
% plain_imperative.tex
%%%%%%%%%%%%%%%%%%%%%%%%%%%%%%%%%%%%%%%%%%%%%%%%%%%%%%%%%%

The game we will implement is a simple vertical shooter. In the game the player controls a shooting ship with the directional arrows and asteroids fall down from the top of the screen. The player must destroy the asteroids before they fall out of the screen, and must also avoid colliding with an asteroid. After too many asteroids fallen out of the screen or after too many hits the game is lost, while after destroying enough asteroids the game is won.

The game starts with a menu. The menu features a simple animation: the various entries slide in from the left side of the screen. When the user selects the ``New Game" button, the menu is closed and the actual game begins. At the end of the game either the ``victory screen" or the ``defeat screen" is launched, depending on how the game ended; these two screens flash some animated text before closing and then launching the menu.

We can see the game as a finite state machine; the menu, the victory and defeat screens, and the game itself are the states of the machine. The transitions occur when the user selects the ``New Game" button, or when the victory or defeat conditions are met. Each state of the finite state machine is implemented in XNA by inheriting from the $Microsoft.Xna.Framework.DrawableGameComponent$ class. This class offers three virtual methods:
\begin{itemize}
\item Initialize (and LoadContent) for the initialization of the internal state of the class
\item Update for the update portion of the main loop
\item Draw for the drawing portion of the main loop
\end{itemize}

Thanks to this mechanism of game components we do not have to worry about managing the main loop and invoking the initialize, update and draw functions by hand; we just create a component, activate it and its functions will be invoked at the appropriate time.

The menu component is the first we will see. This component performs a simple animation so that the various items of the menu enter the screen from the left one after another.

The first structure we need for support is a container for a single menu item; a menu item has some text to be displayed on screen and an Action (a callback function) to be invoked when this item is selected:
\begin{lstlisting}
struct MenuEntry
{
  public string Name;
  public Action OnSelect;
}
\end{lstlisting}

The menu component itself inherits $DrawableGameComponent$:
\begin{lstlisting}
public class MenuScreen : Microsoft.Xna.Framework.DrawableGameComponent
\end{lstlisting}

The menu component stores the amount of time $T$ since its creation (to correctly perform the animation) and an array of entries:
\begin{lstlisting}
float T = 0.0f;
bool has_current_entry = false;
int current_entry = -1;
MenuEntry[] entries;
\end{lstlisting}

In the Initialize method we simply initialize the menu entries; notice how the callback for ``New Game" adds to $Game$, the manager of the main loop of the application, another component while removing itself from the list of active components:
\begin{lstlisting}
entries = new MenuEntry[]
{
  new MenuEntry()
  {
    Name = "New Game",
    OnSelect = () => {
      Game.Components.Add(new GameScreen(Game));
      Game.Components.Remove(this);
      this.Dispose();
    }
  },
  new MenuEntry() {
    Name = "Quit",
    OnSelect = () => Game.Exit()
  }
};
\end{lstlisting}

The update method takes as input from the main loop some timing information. This parameter, called $gameTime$, stores the amount of time since the launch of the game and the amount of time elapsed since the last call to update. This allows us to time our animations correctly: if the elapsed time is very short then we will have to perform little movement of our entities on screen, while if the elapsed time is longer then the movement of our entities on screen will be proportionately longer:
\begin{lstlisting}
public override void Update(GameTime gameTime)
\end{lstlisting}

The update function starts by incrementing the time counter:
\begin{lstlisting}
T += (float)gameTime.ElapsedGameTime.TotalSeconds;
\end{lstlisting}

If the animation is over then we can process the user input, either changing the index of the currently selected entry or invoking the callback of the current entry:
\begin{lstlisting}
if (input[Keys.Up])
{
  has_current_entry = true;
  current_entry = current_entry - 1;
}
else ...
\end{lstlisting}

When it's time to draw, then we iterate all the menu entries. We use a specific formula that allows us to interpolate the position of each menu entry between its starting position $start$ and its destination $end$ so that all the menu entries will enter the screen one at a time:
\begin{lstlisting}
var h = ScreenHeight / (entries.Length);
for (int i = 0; i < entries.Length; i++)
{
  var t = entries[i].Name;
  var c = has_current_entry && i == current_entry ? Color.LightYellow : Color.CornflowerBlue;
  var start = new Vector2(-200, i * h + h / 2);
  var end = new Vector2(400, i * h + h / 2);
  var pos = Vector2.SmoothStep(start, end, (T - EntryAnimationTime * i * 0.25f) / EntryAnimationTime);
  spriteBatch.DrawString(font, t, pos, c, 0.0f, font.MeasureString(t) * 0.5f,
    1.0f, 0, 0);
}
\end{lstlisting}

The $MenuScreen$ class is relatively straightforward, but we have to register a few aspects. First of all, we are using \textbf{a lot} of infrastructure in order to manage the macro-transitions of our game painlessly: we need a definition of the $DrawableGameComponent$ virtual class, and we also need a main loop that is smart enough to manage a list of instances of such class. Secondly, given the nature of the update and draw methods, the behavior of the class is harder to infer than it should be; in particular, the animation is computed analytically inside the draw method, rather than being described ``linearly" as we might wish to do with a straightforward loop such as (in pseudocode):
\begin{lstlisting}
for e in entries do
  e.start <- ...
  e.end <- ...
  while e.pos <> e.end do
    move e.pos towards e.end
\end{lstlisting}

The second $DrawableGameComponent$ that we will see is the one responsible for the actual gameplay. We will not see the $VictoryScreen$ and $DefeatScreen$ components since they resemble very closely the menu component.

The first definition that is needed by the main game is that of the $Entity$ structure. First of all this is a structure, not a class, so it will be compiled as a value type and not as a reference type. The main advantage is that we will not weigh too much on the garbage collector, since value types are not garbage collected:
\begin{lstlisting}
struct Entity
{
  public float life, damage;
  public Vector2 position, velocity;
  public float drag;
\end{lstlisting}

The $Entity$ struct has an update function to update its position with respect to its velocity and to apply ``drag" effect to its velocity:
\begin{lstlisting}
public void Update(float dt)
{
  position += velocity * dt;
  velocity *= (1.0f - dt * drag);
}
\end{lstlisting}

The $Entity$ struct also has some helper functions that allow us to determine whether the areas of two entities intersect; to determine intersection we also need to know the textures (the images) with which the entities will be drawn to screen:
\begin{lstlisting}
public static bool collision(Entity e1, Texture2D t1, Entity e2, Texture2D t2)
\end{lstlisting}

The actual $GameScreen$ class is declared as inheriting $DrawableGameComponent$:
\begin{lstlisting}
public class GameScreen : Microsoft.Xna.Framework.DrawableGameComponent
\end{lstlisting}

The most important fields of the $GameScreen$ class are those that describe the current state of the game; we have the description of the ship, of the asteroids, of the projectiles and the current score:
\begin{lstlisting}
Entity ship;
List<Entity> asteroids;
List<Entity> projectiles;
int missed_asteroids;
int destroyed_asteroids;
\end{lstlisting}

In the initialize method, we set up these fields so that they represent the ``initial" state of the game:
\begin{lstlisting}
missed_asteroids = 0;
destroyed_asteroids = 0;

ship = new Entity()
{
  damage = 50.0f,
  life = 100.0f,
  position = new Vector2(400, 440),
  velocity = Vector2.Zero,
  drag = 2.0f
};

asteroids = new List<Entity>();
projectiles = new List<Entity>();
\end{lstlisting}

The update methods begins by reading the current $\Delta t$; this represents the amount of time we will have to integrate for:
\begin{lstlisting}
var dt = (float)gameTime.ElapsedGameTime.TotalSeconds;
\end{lstlisting}

We update the ship position with respect to the user's input; we also force the ship to ``bounce" against the screen bounds:
\begin{lstlisting}
if (input.KeyboardState.IsKeyDown(Keys.Left))
  ship.velocity -= Vector2.UnitX * dt * SHIP_VELOCITY;
(** ... **)
\end{lstlisting}

If the user presses the \textit{space} key, then a new projectile is added exactly where the ship is. Notice that being $Entity$ a value type when we say $p\ =\ ship$ then we are actually copying the contents of $ship$ into $p$:
\begin{lstlisting}
if (input[Keys.Space])
{
  var p = ship;
  p.velocity = -Vector2.UnitY * PROJECTILE_VELOCITY;
  p.life = 1.0f;
  p.drag = 0.0f;
  p.damage = 5.0f;
  projectiles.Add(p);
}
\end{lstlisting}

We also generate new asteroids in random positions at the top of the screen with a certain (low) probabilty; this way if update is called $n$ times per second then we will generate on average $n \times p_{gen\ ast}$ asteroids per second:
\begin{lstlisting}
if ((float)random.NextDouble() <= ASTEROID_GENERATION_P)
{
  asteroids.Add(new Entity()
  {
    position = new Vector2((float)random.NextDouble() * ScreenWidth, 
                           -asteroid_t.Height),
    velocity = Vector2.UnitY * ASTEROID_VELOCITY * 
               ((float)random.NextDouble() * 1.0f + 0.5f),
    life = 10.0f,
    damage = 34.0f,
    drag = 0.0f
  });
}
\end{lstlisting}

At this point we need to update both asteroids and projectiles. We begin with projectiles:
\begin{lstlisting}
for (int i = 0; i < asteroids.Count; i++)
{
  var a = asteroids[i];
  a.Update(dt);
\end{lstlisting}

we store the value of each projectile into a temporary variable $a$. Until we perform $asteroids_i =\ a$ the modifications done on $a$ will not be reflected on the $asteroids$ list.

If there is a collision between the current asteroid and the ship, then we decrement the life of both:
\begin{lstlisting}
  if (Entity.collision(a, asteroid_t, ship, ship_t))
  {
    a.life -= ship.damage;
    ship.life -= a.damage;
  }
\end{lstlisting}

At this point we iterate all the projectiles. For each projectile that collides with an asteroid then we decrement the life of both; moreover, if a projectile is damaged to the point that its life is $\leq\ 0$, then we remove it from the $projectiles$ list:
\begin{lstlisting}
  for (int j = 0; j < projectiles.Count; j++)
  {
    var p = projectiles[j];

    if (Entity.collision(a, asteroid_t, p, plasma_t))
    {
      a.life -= p.damage;
      p.life -= a.damage;
    }

    if (p.life <= 0.0f)
    {
      projectiles.RemoveAt(j);
      j--;
      continue;
    }
    else
      projectiles[j] = p;
  }
\end{lstlisting}

The asteroids update loop then continues by checking if the current asteroid is destroyed or if it has moved beyond the bottom of the screen; if this is the case, then the asteroid is removed and the score is modified appropriately; otherwise the new value of the asteroid is copied back into the array:
\begin{lstlisting}
  if (a.life <= 0.0f || a.position.Y >= ScreenHeight)
  {
    if (a.life > 0.0f)
      missed_asteroids++;
    else
      destroyed_asteroids++;

    asteroids.RemoveAt(i);
    i--;
    continue;
  }
  else
    asteroids[i] = a;
}
\end{lstlisting}

The update for projectiles is very similar to the previous loop:
\begin{lstlisting}
for (int i = 0; i < projectiles.Count; i++)
{
  var p = projectiles[i];
  p.Update(dt);
  if (p.life <= 0.0f || p.position.Y <= -plasma_t.Height / 2)
    (** REMOVE PROJECTILE p **)
  else
    projectiles[i] = p;
}
\end{lstlisting}

Similarly to how we moved from the menu to the actual game, if the score counters have reached certain thresholds then the game has been won (or lost) and the $GameScreen$ must be removed while the appropriate screen must be added to the set of active components:
\begin{lstlisting}
if (destroyed_asteroids >= 10)
{
  Game.Components.Add(new VictoryScreen(Game));
  Game.Components.Remove(this);
  this.Dispose();
}

if (missed_asteroids >= 10 || ship.life <= 0.0f)
{
  Game.Components.Add(new DefeatScreen(Game));
  Game.Components.Remove(this);
  this.Dispose();
}\end{lstlisting}

The draw function simply iterates all the entities and draws them, together with the current score and the life of the ship:
\begin{lstlisting}
public override void Draw(GameTime gameTime)
{
  spriteBatch.Begin();
  asteroids.ForEach(a => draw(a, asteroid_t));
  projectiles.ForEach(p => draw(p, plasma_t));
  draw(ship, ship_t);

  spriteBatch.DrawString(font, "Life: " + ship.life.ToString("###"), 
                         new Vector2(5, GraphicsDevice.Viewport.Height - 30), 
                         Color.CornflowerBlue);
  spriteBatch.DrawString(font, "Destroyed: " + destroyed_asteroids, 
                         new Vector2(5, 5), Color.LawnGreen);
  var t = "Missed: " + missed_asteroids;
  var t_s = font.MeasureString(t);
  spriteBatch.DrawString(font, t, 
                         new Vector2(ScreenWidth - 5 - t_s.X, 5), Color.Orange);
  spriteBatch.End();

  base.Draw(gameTime);
}
\end{lstlisting}

The code above is very fast and wastes very little processing power. The collision detection is not optimized (as indeed it should and would be in an actual game) but nevertheless this very imperative style where collections are modified in-place has the advantage that we are wasting very little memory and very little CPU cycles. On the other hand parallelizing the above code is quite the challenge: splitting the update and draw loops in particular, which might be very desirable (especially given that virtually no PCs bought today have less than two cores). Sadly, this very style of programming means that whenever we update one collection (either $asteroids$ or $projectiles$) then if that collection is being drawn we will risk a memory leak (the .Net framework is actually very conservative about this: if a collection is modified during its enumeration an exception will be thrown; there is in fact no guarantee that the current element being enumerated is still part of the collection).
 

\section{Immutable implementation}
\label{sec:immutable_implementation}
%%%%%%%%%%%%%%%%%%%%%%%%%%%%%%%%%%%%%%%%%%%%%%%%%%%%%%%%%%
% immutable.tex
%%%%%%%%%%%%%%%%%%%%%%%%%%%%%%%%%%%%%%%%%%%%%%%%%%%%%%%%%%

The second implementation is based on the simple idea of executing two parallel loops: one for all the update calls and the other for all the draw calls. Of course this requires us to encode the current state of the game (the various entities and scores) in such a way that concurrent accesses are possible. To achieve this goal we do not really need to have strong synchronization between the last update and the last draw: we want the application to be robust enough so that the two loops may run at different speed without each one influencing the other. This is particularly desirable because a game that is ``GPU bound", that is a game where rendering the scene is particularly heavy, will not force the update loop to run slower while a game that has very complex logic will not be rendered too slowly.

The various game screens (menu, victory and defeat) are exactly the same as before. The only changes in this implementation happen in the $GameScreen$ class. We will use a technique known as immutability. Immutability means that we will respect the contract that objects are all readonly, and modifying an object is impossible; rather we create copies of each object and said copies will incorporate the modification.

The entity structure is indeed very similar to the previous one:
\begin{lstlisting}
struct Entity
{
  public float life, damage;
  public Vector2 position, velocity;
  public float drag;
\end{lstlisting}

The update function does not modify the entity, but rather it returns the updated entity:
\begin{lstlisting}
  public Entity Update(float dt)
  {
    var other = this;
    other.position += velocity * dt;
    other.velocity *= (1.0f - dt * drag);
    return other;
  }
\end{lstlisting}

Collision detection is identical to the previous implementation. Modifying a field of an entity returns a copy of the original entity with the field in question modified; as an example, consider the $SetLife$ method:
\begin{lstlisting}
  public Entity SetLife(float life)
  {
    var other = this;
    other.life = life;
    return other;
  }
\end{lstlisting}

The state of the game is defined as a structure containing all the information pertaining to the current state of the game; notice that the $missed_{-}asteroids$ and $destroyed_{-}asteroids$ counters have not been promoted to the game state; single variables (the same could be said for $ship$, too) do not really pose challenges in terms of synchronization. Moreover, it is not a problem if a draw call draws the previous value of $missed_{-}asteroids$ and the updated value of $destroyed_{-}asteroids$, so we keep the state leaner by only including what we wish to synchronize:
\begin{lstlisting}
struct GameState
{
  public Entity ship;
  public List<Entity> asteroids;
  public List<Entity> projectiles;
}
\end{lstlisting}

The $GameScreen$ class is once again derived from $DrawableGameComponent$:
\begin{lstlisting}
public class GameScreen : Microsoft.Xna.Framework.DrawableGameComponent
\end{lstlisting}

Its fields are of course a $GameState$ and the scoring counters:
\begin{lstlisting}
GameState game_state;
int missed_asteroids;
int destroyed_asteroids;
\end{lstlisting}

The initialize method starts by initializing the state of the game:
\begin{lstlisting}
missed_asteroids = 0;
destroyed_asteroids = 0;

game_state = new GameState()
{
  ship = new Entity()
  {
    damage = 50.0f,
    life = 100.0f,
    position = new Vector2(400, 440),
    velocity = Vector2.Zero,
    drag = 2.0f
  },
  asteroids = new List<Entity>(),
  projectiles = new List<Entity>(),
};
\end{lstlisting}

Rather than overriding the update function of the $DrawableGameComponent$ though, we define our own private variation that is invoked in a loop that is part of a thread. This thread is launched right away in the initialize function. The condition of the thread loop is a shared boolean variable that is set to false whenever the $GameScreen$ is disposed. Since we cannot rely anymore on a $GameTime$ value passed to the update function, we also have to perform our own timing here; we can even control the number of invocations to the update function separately, by inserting $Thread.Sleep$ calls when appropriate:
\begin{lstlisting}
new Thread(() =>
{
  var now = DateTime.Now;
  while (running)
  {
    var now1 = DateTime.Now;
    var dt = (now1 - now).TotalSeconds;
    Update((float)dt);
    now = now1;

    if (dt < 20)
      Thread.Sleep(30 - (int)dt);
  }
}).Start();
\end{lstlisting}

The update function invoked by the thread takes as input the current $\Delta\ t$ and has the job of creating a new and updated game state; we start by updating the ship. Notice that we are not modifying the ship from the current game state directly:
\begin{lstlisting}
var ship = game_state.ship;
if (input.KeyboardState.IsKeyDown(Keys.Left))
  ship.velocity -= Vector2.UnitX * dt * SHIP_VELOCITY;
(** ... **)
ship = ship.Update(dt);
\end{lstlisting}

We then invoke the $update_{-}entities$ function that, thanks to the use of LINQ (which allows us to query lists in an immutable fashion), allows us to obtain a \textit{lazy} collection of all the updated entities checked for collision against some other collection of entities. The laziness is a distinct advantage because generating a query does not actually allocate the resulting collection; this way we can control how and when the query will be actually executed. The asteroids are updated and checked for collisions against both the ship and the projectiles, while the projectiles are updated and checked for collisions against the asteroids:
\begin{lstlisting}
var asteroids = update_entities(dt, game_state.asteroids, asteroid_t,
  game_state.projectiles.Concat(new[] { game_state.ship }), plasma_t);
var projectiles = update_entities(dt, game_state.projectiles, plasma_t, game_state.asteroids, asteroid_t);
\end{lstlisting}

When we need to create a new projectile, then we invoke the various set functions in a cascading fashion. Then we add (still, lazily!) the new projectile to the collection of updated projectiles:
\begin{lstlisting}
if (this[Keys.Space])
{
  var p = ship.SetVelocity(-Vector2.UnitY * PROJECTILE_VELOCITY)
              .SetLife(1.0f)
              .SetDrag(0.0f)
              .SetDamage(5.0f);
  projectiles = projectiles.Concat(new[] { p });
}
\end{lstlisting}

The same goes for generating asteroids:
\begin{lstlisting}
if ((float)random.NextDouble() <= ASTEROID_GENERATION_P)
{
  asteroids = asteroids.Concat(new[]{new Entity()
  {
    (** ... **)
  }
  });
}
\end{lstlisting}

The state transitions are managed as we did before, by instancing and activating the proper $DrawableGameComponent$s:
\begin{lstlisting}
if (destroyed_asteroids >= 10)
{
  Game.Components.Add(new VictoryScreen(Game));
  Game.Components.Remove(this);
  this.Dispose();
}
(** ... **)
\end{lstlisting}

At this point we perform the actual execution of the queries that will return the updated asteroids and projectiles; we have to execute by hand (with a $for$ loop) the asteroids query because we have to count the removed asteroids for scoring purposes:
\begin{lstlisting}
var asteroids_list = new List<Entity>();
foreach (var a in asteroids)
{
  if (a.life <= 0.0f)
  {
    if (Entity.collision(a, asteroid_t, ship, ship_t))
      ship.life -= a.damage;
    destroyed_asteroids++;
  }
  else if (a.position.Y >= ScreenHeight)
    missed_asteroids++;
  else
    asteroids_list.Add(a);
}
\end{lstlisting}

Finally, we create the new game state with the new asteroids and the new projectiles; the new projectiles can be generated directly from the query with the $ToList$ method:
\begin{lstlisting}
game_state = new GameState()
{
  ship = ship,
  asteroids = asteroids_list,
  projectiles = (from p in projectiles where p.life > 0.0f && p.position.Y >= - plasma_t.Height / 2 select p).ToList()
};
\end{lstlisting}

The $update_{-}entities$ function allows us to achieve something we couldn't (at least not as easily) in the first implementation. In particular, it allows us to capture the general shape of collection traversal that is used for incorporating collisions damage into a collection of entities; gaining the capability of abstracting more and capturing more patterns is something to be valued because it greatly reduces the amount of boilerplate code:
\begin{lstlisting}
private IEnumerable<Entity> update_entities(float dt, 
    IEnumerable<Entity> src1, Texture2D t1,
    IEnumerable<Entity> src2, Texture2D t2)
{
  return from a in src1
          let cs = from p in src2
                  where Entity.collision(a, t1, p, t2)
                  select p.damage
          let d = cs.Sum()
          select a.Update(dt).SetLife(a.life - d);
}
\end{lstlisting}

The draw function simply takes the current state (after assigning it to a temporary variable, so that it will not change during the draw call) and draws it as we did before:
\begin{lstlisting}
public override void Draw(GameTime gameTime)
{
  spriteBatch.Begin();
  var game_state = this.game_state;
  game_state.asteroids.ForEach(a => draw(a, asteroid_t));
  game_state.projectiles.ForEach(p => draw(p, plasma_t));
  draw(game_state.ship, ship_t);

  (** ... **)
}
\end{lstlisting}

\subsection{Benchmarks}

!!!!!!!!!!!!!!!!!!!!!!!!!!!!!!!!!!!!!!!!!!!!!!!!!!

!!!!!!!!!!!!!!!!!!!!!!!!!!!!!!!!!!!!!!!!!!!!!!!!!!

!!!!!!!!!!!!!!!!!!!!!!!!!!!!!!!!!!!!!!!!!!!!!!!!!!

!!!!!!!!!!!!!!!!!!!!!!!!!!!!!!!!!!!!!!!!!!!!!!!!!!

!!!!!!!!!!!!!!!!!!!!!!!!!!!!!!!!!!!!!!!!!!!!!!!!!!
 

\section{Monads refresher}
\label{sec:monads_refresher}
%%%%%%%%%%%%%%%%%%%%%%%%%%%%%%%%%%%%%%%%%%%%%%%%%%%%%%%%%%
% monads refresher.tex
%%%%%%%%%%%%%%%%%%%%%%%%%%%%%%%%%%%%%%%%%%%%%%%%%%%%%%%%%%

\begin{lstlisting}
\end{lstlisting}

\begin{lstlisting}
\end{lstlisting}

\begin{lstlisting}
\end{lstlisting}

\begin{lstlisting}
\end{lstlisting}

\begin{lstlisting}
\end{lstlisting}

\begin{lstlisting}
\end{lstlisting}

\begin{lstlisting}
\end{lstlisting}

\begin{lstlisting}
\end{lstlisting}

\begin{lstlisting}
\end{lstlisting}

\begin{lstlisting}
\end{lstlisting}

\begin{lstlisting}
\end{lstlisting}

\begin{lstlisting}
\end{lstlisting}

\begin{lstlisting}
\end{lstlisting}

\begin{lstlisting}
\end{lstlisting}

\begin{lstlisting}
\end{lstlisting}

\begin{lstlisting}
\end{lstlisting}
\begin{lstlisting}
\end{lstlisting}

\begin{lstlisting}
\end{lstlisting}

\begin{lstlisting}
\end{lstlisting}

\begin{lstlisting}
\end{lstlisting}

\begin{lstlisting}
\end{lstlisting}

\begin{lstlisting}
\end{lstlisting}

\begin{lstlisting}
\end{lstlisting}

\begin{lstlisting}
\end{lstlisting}

\begin{lstlisting}
\end{lstlisting}

\begin{lstlisting}
\end{lstlisting}

\begin{lstlisting}
\end{lstlisting}

\begin{lstlisting}
\end{lstlisting}

\begin{lstlisting}
\end{lstlisting}

\begin{lstlisting}
\end{lstlisting}

\begin{lstlisting}
\end{lstlisting}

\begin{lstlisting}
\end{lstlisting}
 

\section{Functional implementation}
\label{sec:functional_implementation}
%%% Generated by GrindEQ Word-to-LaTeX 2008 
% ========== UNREGISTERED! ========== Please register! ==========
% LaTeX/AMS-LaTeX

\documentclass[a4paper]{article}
\usepackage{anysize}
\marginsize{1cm}{1cm}{1cm}{1cm}

\usepackage{amssymb}
\usepackage{amsmath}
\usepackage[pdftex]{graphicx}
\usepackage{epsfig}
\usepackage{subfigure}
\usepackage{listings}
\lstset{language=haskell}
\lstset{commentstyle=\textit}
\lstset{mathescape=true}
%\lstset{labelstep=1}
%\lstset{backgroundcolor=,framerulecolor=}
\lstset{backgroundcolor=,rulecolor=}
\linespread{2.0}

\begin{document}

%%% remove comment delimiter ('%') and select language if required
%\selectlanguage{spanish} 

\noindent 
\section{Running the Examples}

\noindent Having a concrete implementation means that now we can try and run the examples of which we only listed the code and gave the type. Let us start with:

\begin{lstlisting}
$ex_1$ = eval i >>= ($\lambda$v. i := (v $\times$ 10) >> eval i)
\end{lstlisting}

We wish to evaluate the term $runST\ ex_1\ (5:Nil)$, assuming that:

\begin{lstlisting}
i=Reference ($\lambda$ h.(hRead h (_:Z),h)) ($\lambda$ v.$\lambda$ h.( (),hWrite v h (_:Z)))
\end{lstlisting}

the resulting evaluation is:

\begin{lstlisting}
= runST (eval i >>= ($\lambda$v. $\dots$ )) (5:Nil) \ 
= runST ((ST $\lambda$h.(hRead h (_:Z),h)) >> =($\lambda$v. $\dots$ )) (5:Nil) \ 
= runST (i := 5 $\times$ 10 >> $\dots$ ) (5:Nil) \ 
= runST ((ST $\lambda$h.((),hWrite 50 h (_:Z))) >> $\dots$ ) (5:Nil) \ 
= runST (eval i) (50:Nil) \ 
= runST (ST $\lambda$h.(hRead h (_:Z),h)) (50:Nil) \ 
= 50
\end{lstlisting}

This is exactly the result we would have expected. Also notice that for all practical purpose our programs act as if exactly one heap is accessible at all times. Depending on the capabilities of the runtime this might also be essentially true, as most of the parts of a heap could be shared thanks to the immutability of the data structures of the various heaps which are allocated with various shared parts.

\noindent Let us now consider our second example:

\begin{lstlisting}[frame=tb,mathescape]{somecode}
$ex_2'$ =
	do 10 $>>=$ ($\lambda$i.
	do "hello " $>>=$ ($\lambda$s.
	do s *= ($\lambda$x.x++"world")
	   let $i'$ = downcast i
	   v$\leftarrow$eval s
	   x$\leftarrow$eval $i'$
	   return v ++ show x))
\end{lstlisting}

We will evaluate this as if it were a program launched all by itself (that is with an empty heap and no external references), by evaluating the term $runST\ ex_2'\ Nil$; we will evaluate this term with a (simpler) small-step semantics that refers to the effect our statements have on the heap and the various bound variables:

\begin{tabular}{|p{1.1in}|p{0.2in}|p{0.7in}|p{0.2in}|p{0.7in}|p{1.0in}|} \hline 
Statement & i & s & x & v & Heap \\ \hline 
ex\_2\^{}'=do 10$\gg $+($\lambda $i. & - & - & - & - & 10::Nil \\ \hline 
do(\_ \^{}'')hello\^{}''$\gg $+($\lambda $s. & 10 & ``Hello'' & - & - & ``Hello''::10::Nil \\ \hline 
do s*=($\lambda $x.x++\^{}'' world\^{}'' ) & 10 & ``Hello World'' & - & - & ``Hello World''::10::Nil \\ \hline 
let i\^{}'=downcast i & 10 & ``Hello World'' & - & - & ``Hello World''::10::Nil \\ \hline 
 v?eval s & 10 & ``Hello World'' & 10 & ``Hello World'' & ``Hello World''::10::Nil \\ \hline 
 x?eval i' & 10 & ``Hello World'' & 10 & ``Hello World'' & ``Hello World''::10::Nil \\ \hline 
 return v++show x)) & - & - & - & - & Nil \\ \hline 
\end{tabular}



\noindent The returned result is (as expected) $"10Hello\ World"\ $.

\noindent 

\noindent 

\noindent 

\noindent 


\end{document}

% == UNREGISTERED! == GrindEQ Word-to-LaTeX 2008 ==

 

\section{Unfolding animations and sequential operations}
\label{sec:unfold_monad}
%%
% conclusions.tex
%

Innovations in game architecture are often responsible for pushing further the boundaries of game design. By building more powerful \textit{and} more scriptable game engines it is possible at the same time to create games where game logic is faster and more complex, resulting in a more compelling experience for the player.

Building a system that allows for a smarter, individualized AI could usher an era of games that offer a smarter experience where units and non-playing characters make smart choices rather than waiting for ``obvious'' instructions from the player.

It must also be noted that games have been historically responsible for many evolutions of modern personal computing, by pushing the hardware to its limits and by offering powerful visual experiences that years or decades later become paradigms in non-gaming software. For this reason this kind of language evolution must be put in its right context: it is not just a way to make better games, but it is also a prototype for a way to write better \textit{programs} in general.

\paragraph{Future research directions in the field}
From a survey of the current panorama of game development tools and languages it appears that a technological shift is happening: from the almost exclusive use of C and C++ for creating games, plus the odd use of Lua or Python as scripting languages, more and more games are being made in modern managed languages such as C\# or Objective-C. The introduction of XNA (a framework that allows independent game developers to publish commercial games on the XBox 360 without costly licenses) has marked the beginning of the adoption of C\#-based frameworks for making games; other similar frameworks have been used in games between a C++ engine and the scripting system. This shift signals that the game development industry may be starting to have trouble with its old infrastructure given the rising development costs, the shrinking budgets and the decreasing development time and it is looking into new languages with more abstraction to increase its efficiency. For this reason many researchers (the authors of this report being part of this group) are studying new ways to leverage the power coming from the field of functional and declarative languages in order to more easily automate some repetitive programming tasks that often require complex and cumbersome libraries. 

\section{Benchmarks}
\label{sec:benchmarks}
%%
% conclusions.tex
%

Innovations in game architecture are often responsible for pushing further the boundaries of game design. By building more powerful \textit{and} more scriptable game engines it is possible at the same time to create games where game logic is faster and more complex, resulting in a more compelling experience for the player.

Building a system that allows for a smarter, individualized AI could usher an era of games that offer a smarter experience where units and non-playing characters make smart choices rather than waiting for ``obvious'' instructions from the player.

It must also be noted that games have been historically responsible for many evolutions of modern personal computing, by pushing the hardware to its limits and by offering powerful visual experiences that years or decades later become paradigms in non-gaming software. For this reason this kind of language evolution must be put in its right context: it is not just a way to make better games, but it is also a prototype for a way to write better \textit{programs} in general.

\paragraph{Future research directions in the field}
From a survey of the current panorama of game development tools and languages it appears that a technological shift is happening: from the almost exclusive use of C and C++ for creating games, plus the odd use of Lua or Python as scripting languages, more and more games are being made in modern managed languages such as C\# or Objective-C. The introduction of XNA (a framework that allows independent game developers to publish commercial games on the XBox 360 without costly licenses) has marked the beginning of the adoption of C\#-based frameworks for making games; other similar frameworks have been used in games between a C++ engine and the scripting system. This shift signals that the game development industry may be starting to have trouble with its old infrastructure given the rising development costs, the shrinking budgets and the decreasing development time and it is looking into new languages with more abstraction to increase its efficiency. For this reason many researchers (the authors of this report being part of this group) are studying new ways to leverage the power coming from the field of functional and declarative languages in order to more easily automate some repetitive programming tasks that often require complex and cumbersome libraries. 

\section{Conclusions and future work}
\label{sec:conclusions}
%%
% conclusions.tex
%

Innovations in game architecture are often responsible for pushing further the boundaries of game design. By building more powerful \textit{and} more scriptable game engines it is possible at the same time to create games where game logic is faster and more complex, resulting in a more compelling experience for the player.

Building a system that allows for a smarter, individualized AI could usher an era of games that offer a smarter experience where units and non-playing characters make smart choices rather than waiting for ``obvious'' instructions from the player.

It must also be noted that games have been historically responsible for many evolutions of modern personal computing, by pushing the hardware to its limits and by offering powerful visual experiences that years or decades later become paradigms in non-gaming software. For this reason this kind of language evolution must be put in its right context: it is not just a way to make better games, but it is also a prototype for a way to write better \textit{programs} in general.

\paragraph{Future research directions in the field}
From a survey of the current panorama of game development tools and languages it appears that a technological shift is happening: from the almost exclusive use of C and C++ for creating games, plus the odd use of Lua or Python as scripting languages, more and more games are being made in modern managed languages such as C\# or Objective-C. The introduction of XNA (a framework that allows independent game developers to publish commercial games on the XBox 360 without costly licenses) has marked the beginning of the adoption of C\#-based frameworks for making games; other similar frameworks have been used in games between a C++ engine and the scripting system. This shift signals that the game development industry may be starting to have trouble with its old infrastructure given the rising development costs, the shrinking budgets and the decreasing development time and it is looking into new languages with more abstraction to increase its efficiency. For this reason many researchers (the authors of this report being part of this group) are studying new ways to leverage the power coming from the field of functional and declarative languages in order to more easily automate some repetitive programming tasks that often require complex and cumbersome libraries. 

\bibliographystyle{plain}
\bibliography{references} 
\cite{*}
\nocite{}

\end{document}
