%
% rts.tex
%

Real-time strategy games are the ideal field of application of techniques that increase the number of supported characters, as demonstrated by the number of units available. In these games, the player controls large numbers of \textit{units} which are selected and which receive commands; the execution of commands by each unit is controlled by the AI; the smarter the AI, the more entertaining the game: a player wants his units to follow his instructions correctly, without having to specify too many details; this way, when the player has issued orders to some units, he can move to other portions of the map and give other orders to other units while the original orders are being carried out. Unfortunately unit behavior in current RTSes is quite primitive, usually modeled with simple finite state machines. As a result, players need to issue many orders to achieve a good level of coordination between his units. Processing scripts that handle the correct behavior of each unit is very expensive at run-time, since each unit is typically processed separately. A typical solution to this problem is that of processing units in groups (centralized AI); with this technique each script controls a large number of units, which then behave with good coordination. For example, in the game \textit{Warcraft III}, the AI for each computer player is divided in two commanders: defence and attack. Centralized AI often has problems when trying to maintain actions over multiple fronts (more than the number of virtual commanders available), and it is of no help to the human player. Having smarter units would benefit both the computer player and the human player.

Note that centralized AI is just an ad-hoc form of set-at-a-time processing, explicitly implemented by the game designer who aggregates units knowing that they will perform a similar role. The authors propose to build a scripting language that makes grouping units behaviors together more systematic with a declarative language like SQL, paired with a set of optimizations appropriate for games.

The most important functionality of scripts will thus be implemented with a purely functional language with aggregate operations on sets. The case study will be a synthetic RTS with three types of units:

\begin{itemize}
\item armored \textbf{knights} who can move and attack at short range
\item \textbf{archers} who can move and attack at long range
\item \textbf{healers} who heal friendly units within a certain range; healing auras are nonstackable
\end{itemize}

