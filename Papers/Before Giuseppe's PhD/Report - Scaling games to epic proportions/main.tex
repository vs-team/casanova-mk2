\documentclass{article}

\usepackage{amssymb}
\usepackage{amsmath}
\usepackage{graphicx}
\usepackage{epsfig}
\usepackage{subfigure}
\usepackage{listings}
\usepackage{natbib}
\usepackage{verbatim}
\usepackage[T1]{fontenc} 
\usepackage[hyphens]{url}
\usepackage{stmaryrd}
\lstset{language=ml}
\lstset{commentstyle=\textit}
\lstset{mathescape=true}
\lstset{backgroundcolor=,rulecolor=}
\lstset{frame=single}
\lstset{breaklines=true}
\lstset{basicstyle=\ttfamily}

\begin{document}

\title{Query languages applied to game development}

\author{
\textbf{Giuseppe Maggiore} \and \textbf{Giulia Costantini} \\
       Universit\`a Ca' Foscari Venezia\\
       Dipartimento di Scienze Ambientali,\\
       Informatica e Statistica\\
       \texttt{\{maggiore,costantini\}@dais.unive.it}
}

\date{\today}
\maketitle

\begin{abstract}
In this report we summarize the application of declarative query languages to game development. Thanks to the use of a language that is very similar to SQL the authors have been able to build games with complex logic, fast runtime and very little and easily readable code.
\end{abstract}

\section{Introduction}
\label{sec:intro}
%----------------------------------------------------------------------------
%  intro.tex 
%----------------------------------------------------------------------------
Modern computer languages are very reliable when it comes to writing a large class of common, real-world applications. For example, relatively simple form applications or web sites can be built extremely easily in languages such as Java, C\# and many others. This is thanks to commonplace facilities like garbage collectors, classes and inheritance and large libraries which simplify many tasks which otherwise would be hard or error-prone. On the other hand, there is a not so small set of applications for which these languages do not perform even nearly as well; for example games, even though very powerful libraries such as XNA make them easier to write by encapsulating many useful patterns, are not so suitable for modern languages. For this reason most games are still written in C++ (sometimes even in C) and the transition to higher level languages is not happening as fast as it could. As another example we could consider mobile applications. The widespread adoption of very powerful, fully programmable smartphone like the iPhone, Google Android or Windows Phone 7 makes performance even more important to achieve: lighter applications mean much better applications where CPU cycles and battery are both scarce resources. On the other hand, to allow as many developers as possible to easily create applications for these platforms, it makes sense (as indeed it is happening) to allow programming these devices with as languages that are as high-level as possible. Finally, there are many real time or soft real time applications that migh benefit from using high level languages but which cannot afford the pauses or slowdowns that sometimes the garbage collector might require, especially on less powerful hardware.


In this paper we document the results of implementing a videogame in such a high-level language (F\#) for a mobile device. Early in our development cycle we discovered that the biggest problem of our application was that by continually allocating and deallocating instances of the same types (such as projectiles and particles) we kept triggering the garbage collector, forcing the application to a crawl everytime it started. Rather than implement object pooling, we decided to try and generalize this technique by implementing reference counting \cite{7_6} inside the state monad \cite{1_1}. The state monad contains and manages all the actual instances of the objects which require reference counting, and references to these objects can only be accessed through the state monad itself. This allowed us to track the lifetime of our entities, in a lightweight enough fashion to ensure the increase in performance that we needed and transparently enough that it became convenient to track other resource types as well, such as files or GPU memory which require manual disposal.


In this paper we discuss a possible generalization of the work described above, where in addition to a monadic reference counting system we also discuss how type-level meta-programming \cite{3_1,3_2,3_3,6_1} and the parameterized monad \cite{1_7} can be used to track the types of resources with a strongly typed heterogeneous list \cite{4_1}, thereby removing even the need for the developer to "discover" what type the state must have. Also, this system could be the stepping stone for a more refined state-tracking monad which uses type-level values and phantom types to track expected properties of the state monad at various points in the program.


An important note is about the computer language in which we have written the samples. We have used a pseudo-Haskell, and we believe our listings may be turned into a working program without too much effort. This notwithstanding, it must be noted that said effort has not been made by us: our software system is mostly interested in performance, and from the point of view of benchmarking performance Haskell may not be the best suited language given its lazy evaluation strategy. For us this pseudo-Haskell has acted as a clear and easily implementable specification, and we do not claim anything about its workability in any present or future versions of the language.
 
 

\section{Data-driven games}
\label{sec:ddg}
%
% ddg.tex
%

A data-driven game \cite{10} separates game content from the game code. This design allows both game programmers and game designers (groups with very different skillsets) to contribute separately to game development. This separation has long been practiced, for example by storing models, textures and sounds in data files separate from the game engine; this trend has recently accelerated by trying to move as much content as possible out of the engine: character and story-line data is increasingly \cite{25} being stored in external configuration files. Modern design goes even further, storing game logic specific to game play in external files written in a scripting language such as Lua, Python or a custom-made scripting language \cite{SCRIPTING_LUA,SCRIPTING_PYTHON,UNREALSCRIPT_LATENT_FUNCTIONS}.

\begin{figure*}
\begin{center}
\includegraphics[scale=0.5]{engine_architecture.png}
\end{center}
\label{fig:data_driven_games}
\caption{Data-driven game architecture}
\end{figure*}

In Figure 2 we can see the general architecture of a data-driven game. The game engine (AI, physics, rendering, audio and discrete simulation) is written and maintained by programmers. Typically, the discrete simulation engine puts together the interactions between physics, AI and game logic in a meaningful way, and exposes the resulting game state and events to the rendering and audio engine which give some feedback to the player. The game content is created by game designers, who are responsible with creating and populating the game world; the game world contains both artistic elements like models and sounds, but also logical elements such as behaviors and character statistics. Logical elements are defined by scripts; scripts are either compiled or interpreted and executed by the discrete simulation engine. Separating scripts becomes important in all those game where character behavior must be tested and adjusted constantly during development to ensure that there is no single optimal strategy that can ensure victory (otherwise the game would lose much of its challenge and interest). Also, separating game logic into scripts allows for ``modding'', that is players can modify the game even without access to its source code; the creativity of the players can extend the lifetime of a game beyond the original intentions of its developers (\cite{21}). The entire focus of the authors is on the discrete simulation engine and how techniques taken from the database comunity can be used to build a better performing implementation that can be heavily scripted.

\paragraph{The Discrete Simulation Engine}
Game are processed in clock ticks. During each clock tick the simulation engine processes the current state of the world, thereby computing its updated version. This usually consists in considering the actions of all the characters of the game (some characters may perform null actions, but still they are given an opportunity at every tick); each action may produce several \textit{effects}.  Effects produced during the same tick are written into the game state simultaneously, allowing us to cleanly separate each clock tick into three stages:

\begin{itemize}
\item a query stage when we read the contents of the game data
\item a decision stage where we choose the actions of each character
\item an update stage where we write into the game state the effects produced by each action
\end{itemize}

Since actions are all updating the game data simultaneously, we use a transaction model to describe how these updates are processed. Since effects usually increase or decrease numeric values then such a model becomes simply an aggregate function such as \texttt{(+), (-), max, min, ...}. Effects are usually separated into \textit{stackable} and \textit{non-stackable}, depending on whether or not they accumulate during one tick or only one update is picked; for example, damage is stackable since a unit that suffers damage from multiple opponents receives the sum of all these damages, while healing auras are non-stackable because only the most beneficial aura is applied while all the others are discarded.

\section{Real-time strategy games}
\label{sec:rts}
%
% rts.tex
%

Real-time strategy games are the ideal field of application of techniques that increase the number of supported characters, as demonstrated by the number of units available. In these games, the player controls large numbers of \textit{units} which are selected and which receive commands; the execution of commands by each unit is controlled by the AI; the smarter the AI, the more entertaining the game: a player wants his units to follow his instructions correctly, without having to specify too many details; this way, when the player has issued orders to some units, he can move to other portions of the map and give other orders to other units while the original orders are being carried out. Unfortunately unit behavior in current RTSes is quite primitive, usually modeled with simple finite state machines. As a result, players need to issue many orders to achieve a good level of coordination between his units. Processing scripts that handle the correct behavior of each unit is very expensive at run-time, since each unit is typically processed separately. A typical solution to this problem is that of processing units in groups (centralized AI); with this technique each script controls a large number of units, which then behave with good coordination. For example, in the game \textit{Warcraft III}, the AI for each computer player is divided in two commanders: defence and attack. Centralized AI often has problems when trying to maintain actions over multiple fronts (more than the number of virtual commanders available), and it is of no help to the human player. Having smarter units would benefit both the computer player and the human player.

Note that centralized AI is just an ad-hoc form of set-at-a-time processing, explicitly implemented by the game designer who aggregates units knowing that they will perform a similar role. The authors propose to build a scripting language that makes grouping units behaviors together more systematic with a declarative language like SQL, paired with a set of optimizations appropriate for games.

The most important functionality of scripts will thus be implemented with a purely functional language with aggregate operations on sets. The case study will be a synthetic RTS with three types of units:

\begin{itemize}
\item armored \textbf{knights} who can move and attack at short range
\item \textbf{archers} who can move and attack at long range
\item \textbf{healers} who heal friendly units within a certain range; healing auras are nonstackable
\end{itemize}



% syntax, semantics (informal), optimization (informal)
\section{SGL}
\label{sec:sgl}
%
% sgl.tex
%

Game data is modeled as a relation $E$. This table is a multiset and it needs not have keys. Each row in the table represents a unit or object, with information such as health, speed, damage, special properties, etc. Each row also contains data representing messages to this unit, such as data coming from the pathfinding system, cooldown periods, and so on. One possible definition of $E$ for our example could be:

\begin{lstlisting}
E(key,player,posx,posy,health,cooldown,
  weaponused,movevect_x,movevect_y,damage,inaura)
\end{lstlisting}

The attributes \texttt{key ... cooldown} represent the state of the unit; these attributes are not modified directly from a script, but rather they are modified accordingly to the values of \texttt{weaponused ... inaura}, which represent the effects that unit is being subjected to.

An SGL script consists of a single action for a single unit, and at each tick it will be executed and it will take an entire environment $E$ and it will return a new environment $E_u$. All environments $E_u$ are then combined into a single environment in which the effects are applied with a post-processing step. A possible post-processing related to the schema above could be (where \texttt{norm} normalizes movement velocity):

\begin{lstlisting}
SELECT u.key, u.player, 
       u.posx + u.movevect_x * norm AS posx, 
       u.posy + u.movevect_y * norm AS posy, 
       u.health - u.damage + u.inaura AS health, 
       u.cooldown - 1 + u.weaponused*_TIME_RELOAD AS cooldown, 
       0 AS weaponused, 
       0 AS movevect_x, 
       0 AS movevect_y, 
       0 AS damage, 
       0 AS inaura
FROM E u 
WHERE u.health > 0
\end{lstlisting}

The post-processing step above applies the effects to the state and resets the effects to avoid applying them again during the next time step.

\paragraph{Syntax of SGL}
SGL scripts have a simple syntax consisting of a mix of ML (\texttt{let, if-then-else}) and SQL:

\begin{lstlisting}
main(u) { 
  (let c = CountEnemiesInRange(u,u.range)) 
  (let away_vector = (u.posx, u.posy) -
         CentroidOfEnemyUnits(u, u.range)) { 
           if (c > u.morale) then
             perform MoveInDirection(u,away_vector); 
           else if (c > 0 and u.cooldown = 0) then
             (let target_key = getNearestEnemy(u).key) { 
             perform FireAt(u, target_key);
} } }
\end{lstlisting}

Where the auxiliary functions described above can be defined as:

\begin{lstlisting}
function CountEnemiesInRange(u, range) 
returns SELECT Count( * ) 
        FROM	E
        WHERE E.x >= u.posx - range
        AND	E.x <= u.posx + range
        AND	E.y >= u.posy - range
        AND	E.y <= u.posy + range
        AND	E.player <> u.player;
\end{lstlisting}

In general, actions have the following syntax:

\begin{lstlisting}
action ::= (let attributename = term) 
         | action action; action 
         | if cond then action 
         | if cond then action else action 
         | perform action_name
\end{lstlisting}


\paragraph{Combining effects}
Effects in an environment $E$ must be combined somehow. To do so, we add annotations to each attribute in the environment schema:

$$E(K,A_1:\tau_1,...,A_n:\tau_n)$$

where $\tau_i$ is an aggregate function such as \texttt{min, max, avg, sum, ...} which defines the way multiple effects on each unit are combined. A combination operator $\oplus R$ is defined as:

\begin{lstlisting}
select K, $f_{i_1}$($A_{i_1}$) as $A_{i_1}$,...,$f_{i_m}$($A_{i_m}$) as $A_{i_m}$
from R group by K,$A_{i_1}$,...,$A_{i_l}$;
\end{lstlisting}

where $f_i(A_i)$ is defined as $A_i$ if $\tau_i = const$ (that is no aggregation is needed on attribute $i$) while it is defined as $\tau_i$ otherwise (that is it uses the aggregate function described by $\tau_i$ itself). We can define (unambiguously) another meaning for $\oplus$: $R \oplus S = \oplus(R \biguplus S)$ where $\biguplus$ denotes multiset union.

In the case of the schema described above, the $\oplus E$ environment can be computed as:

\begin{lstlisting}
SELECT key, player, posx, posy, health, cooldown, 
       max(weaponused) AS weaponused, 
       sum(movevect_x) AS movevect_x, 
       sum(movevect_y) AS movevect_y, 
       sum(damage) as damage,
       max(inaura) as inaura
FROM E
GROUP BY key, player, posx, posy, health, cooldown
\end{lstlisting}

Each action can be seen as a function from a unit and the environment into the environment which at each evaluation step applies the combination operator; actions have type:

$action : Env \times Multiset(Env) \rightarrow Multiset(Env)$

which is the denotational equivalent of a stateful function. The first parameter is the tuple representing the current unit, the second parameter is the set of all units, the result is the new set of all units. The semantics of SGL actions can be given in terms of a semantic function $\llbracket \bullet \rrbracket$:

\begin{lstlisting}
$\llbracket$let v = t in f$\rrbracket$(u,E) 	:= $\llbracket$f[v $\mapsto$ $\llbracket$ t $\rrbracket$](u,E)$\rrbracket$(u,E)
$\llbracket$f;g$\rrbracket$(u,E)             	:= $\llbracket$f$\rrbracket$(u,E) $\oplus$ $\llbracket$g$\rrbracket$(u,E)
$\llbracket$if c then f$\rrbracket$(u,E)     	:= if $\llbracket$c$\rrbracket_{cond}$(u,E) then
$\llbracket$perform(G)$\rrbracket$(u,E)     	:= $\llbracket$G$\rrbracket$(u,E)
\end{lstlisting}

where \texttt{G} is a built-in action function or a user-defined auxiliary function, and where \texttt{if-then-else} can easily be defined in terms of sequential conditionals with the same condition (negated in the \texttt{else} branch).

\paragraph{Optimization}
As with typical relational systems, there are two possible optimization venues: algebraic and physical. Algebraic optimizations are applied to queries in order to ensure that a script combines as often as possible in order to keep the number of elements to process to the minimum possible; for example, rather than compute:

$$\oplus(R \biguplus S)$$

it is usually convenient to compute:

$$\oplus(\oplus R \biguplus \oplus S)$$

thanks to the fact that the $\oplus$ operator returns a smaller set (effects are condensed together into the state).

More important in terms of performance gain is the use of indices.

Sometimes a unit must process an aggregate function for example to count the number of nearby units; for example, a unit may wish to flee if its allies are overwhelmed by a large enemy force. This script can be na\"ively computed with an $O(n^2)$ algorithm. This computation can be greatly accelerated with the introduction of an index for the aggregate that allows to share the results of the aggregation across several units.

SGL queries have the advantage that the set of queries is fixed a priori, and thus the indices can be tailored around each individual query plan. Also, it is important to realize that while indices can be used to share information between nearby units, indices are discarded and rebuilt at the beginning of each clock tick beacuse of the very high dynamicity of unit data even across one single tick; for the most important and often updated indices such as those on unit positions, it is more efficient to do so than it is to maintain a dynamic index.

The use of indices reduces the complexity of many common AI algorithms from $O(n^2)$ to $O(n \log n)$. This gives games the possibility to scale to an order of magnitude more units without incurring in a significant performance hit.

\section{Conclusions}
\label{sec:conclusions}
%
% conclusions.tex
%

Innovations in game architecture are often responsible for pushing further the boundaries of game design. By building more powerful \textit{and} more scriptable game engines it is possible at the same time to create games where game logic is faster and more complex, resulting in a more compelling experience for the player.

Building a system that allows for a smarter, individualized AI could usher an era of games that offer a smarter experience where units and non-playing characters make smart choices rather than waiting for ``obvious'' instructions from the player.

It must also be noted that games have been historically responsible for many evolutions of modern personal computing, by pushing the hardware to its limits and by offering powerful visual experiences that years or decades later become paradigms in non-gaming software. For this reason this kind of language evolution must be put in its right context: it is not just a way to make better games, but it is also a prototype for a way to write better \textit{programs} in general.

\paragraph{Future research directions in the field}
From a survey of the current panorama of game development tools and languages it appears that a technological shift is happening: from the almost exclusive use of C and C++ for creating games, plus the odd use of Lua or Python as scripting languages, more and more games are being made in modern managed languages such as C\# or Objective-C. The introduction of XNA (a framework that allows independent game developers to publish commercial games on the XBox 360 without costly licenses) has marked the beginning of the adoption of C\#-based frameworks for making games; other similar frameworks have been used in games between a C++ engine and the scripting system. This shift signals that the game development industry may be starting to have trouble with its old infrastructure given the rising development costs, the shrinking budgets and the decreasing development time and it is looking into new languages with more abstraction to increase its efficiency. For this reason many researchers (the authors of this report being part of this group) are studying new ways to leverage the power coming from the field of functional and declarative languages in order to more easily automate some repetitive programming tasks that often require complex and cumbersome libraries.

\bibliographystyle{plain}
\bibliography{references} 

%\cite{*}
\nocite{}

\end{document}
