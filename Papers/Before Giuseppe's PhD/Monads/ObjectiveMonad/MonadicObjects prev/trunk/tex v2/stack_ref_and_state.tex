\section{Introduction}
Our goal is to define a set of operators that allow us to write object-oriented programs in the Haskell language. Object orientation can be
used in conjunction with various paradigms, such as (these are by no means the only possible fields of application):
\begin{itemize}
\item mutable programs where the state inside each object can be transparently mutated by method calls and other kinds of manipulations
\item concurrent programs where the inner state of each object does not belong to the same thread, process or even machine
\item reactive programs where each stateful operation is recomputed whenever the values it depends from change
\item transactional programs where each stateful operation is recorded and blocks of stateful operations can be undone
\end{itemize}

We definitely wish to define our operators so that we can use anyone of these paradigms for running our object-oriented programs. For this
reason we define the object oriented operators abstractly, that is inside type classes; we then proceed to give the concrete implementations 
that will allow us to actually run our code.

We also wish to make it simple to use our objects. One of the best ways to make some abstraction simpler to code in the Haskell language is to
take advantage of monads and their syntactic sugar: we will try to make use of monads whenever possible, and we will even require so at the
level of our typeclasses.

To make a working object oriented system in a type-safe and purely functional language such as Haskell we are forced to define many 
type-level functions and predicates. A relevant side effect of this is that all our entities are first class entities in the host language;
this allows us to freely manipulate labels, method applications, and so on, and even to design a type safe reflection system.


\section{Stack}
We start by defining a stack typeclass, which will define the basic environment of our computations. We choose to use a stack for simplicity,
and also because the state monad is much harder to make work with a heap. The stack predicate is defined as follows:

\begin{verbatim}
class Stack s where
  type Push s :: * -> *
  push :: s -> a -> Push s a
  pop  :: Push s a -> s
\end{verbatim}

the stack is characterized by three operators:
\begin{itemize}
\item a type function that takes our stack and another type as input and returns the new stack obtained by pushing the second parameter 
on top of the input stack
\item a $push$ function which works exactly as the $Push$ type function but operating on values rather than types
\item a $pop$ function which does the opposite of what the $push$ function does
\end{itemize}


\section{References and Statements}
We will never work directly with values, since what we are trying to accomplish requires that values are packed inside "smart" containers
that are capable of doing more complex operations such as mutating a shared state, sending messages to other processes or tracking dependencies
from other smart values to implement reactive updates. For this reason we will represent values in two different ways:
\begin{itemize}
\item as references whenever we wish to represent a pointer to some value inside the current stack
\item as statements whenever we wish to represent the result of arbitrary computations
\end{itemize}

References and statements are defined respectively with one single predicate:

\begin{verbatim}
class (Stack s,Monad (st s)) => RefSt ref st s where
  eval :: ref s a -> st s a
  (:=) :: ref s a -> a -> st s ()
  (*=) :: ref s a -> (a->a) -> st s ()
  new :: s'~New s a => a -> (ref s' a -> st s' b) -> st s b
\end{verbatim}

The $st$ functor applied to the stack $s$ is required by this definition to be a monad; thanks to this we can use our operators on references
taking advantage of the syntactic sugar that Haskell offers, obtaining code that is much more intuitive to a programmer used to traditional
object oriented languages.

The $eval$ function takes as input a reference to a value of type $a$ inside a stack of type $s$ and returns a statement that will evaluate 
to a value of type $a$ inside a state of type $s$. The $:=$ and $*=$ operators respectively assign and in-place modify a reference and return
a statement that will evaluate to a value of type unit inside a state of type $s$. The $new$ function allocates a value of type $a$ on the 
stack $s$, obtaining the stack $s'$, performs some operation on this stack inside a function that takes as input the reference to the 
freshly allocated value of type $a$ and returns a statement that will evaluate to a $b$ in an $s'$, and then returns a statement that will
evaluate to the same $b$ in an $s$. The $new$ function is essentially a scoping function, which will:
\begin{itemize}
\item allocate a new value on the stack
\item create a reference to this new value in the new stack
\item do some work with this reference and obtain a result through the input function
\item deallocate the value from the stack
\item return a result that represent the result returned by the work done in the input function
\end{itemize}

An alternative definition (possibly even better) of the new operation could have been:
\begin{verbatim}
new :: s'~New s a => a -> (ref s' a -> st s' b) -> st s' b
\end{verbatim}

which does not automatically destroy the allocated value and which instead maintains the larger stack when it returns its value. This would
have worked well with a heap rather than a stack, since we could have given a heap capable of deallocating its references with a type function
such as:

\begin{verbatim}
class (Heap h, Monad (st h)) => RefSt ref st h where
  Delete h -> ref h a :: *
  delete h -> ref h a -> Delete h (ref h a)
  ...
\end{verbatim}

Sadly this approach has a major shortcoming. In fact this would make it much harder to work with monads, since at every allocation and
deallocation the type of the monadic constructor would change: $(st\ h)\ \ne\ (st\ (New\ h\ a))$, and we would have been forced to define 
an explicit binding operator that is capable of taking care of changes in the first parameter of the statement functor:

\begin{verbatim}
explicit_bind :: st h a -> (a -> st (New h b) a') -> st (New h b) a'
\end{verbatim}

but this would, interesting as it may be, would somewhat defeat the point of using a monad for simplicity and readability in the first place.
