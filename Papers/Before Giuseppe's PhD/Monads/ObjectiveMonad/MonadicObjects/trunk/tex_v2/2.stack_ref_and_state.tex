\subsection{Memory}
We start by defining a memory typeclass, which will define the basic environment for our computations. We model our memory after a stack for simplicity. The memory predicate is defined as follows:

\begin{lstlisting}
class Memory m where
  data Malloc :: * -> * -> *
  malloc :: m -> a -> Malloc a m
  free   :: Malloc a m -> m
  read   :: Malloc a m -> a
  write  :: a -> Malloc a m -> Malloc a m
\end{lstlisting}

Our memory supports type-safe allocation and deallocation 
thanks to two function, $malloc$ and $free$, and a type 
function $Malloc$ that defines the type of our memory 
to which a value of type $a$ is added. We also define accessors to $read$ and $write$ the elements of our memory.

\subsection{References and Statements}
We will never work directly with values, since what we are trying to accomplish requires that values are packed inside "smart" containers that are capable of doing more complex operations such as mutating a shared state, sending messages to other processes or tracking dependencies from other smart values to implement reactive updates. For this reason we will represent values in two different ways:
\begin{itemize}
\item as references whenever we wish to represent a pointer to some value inside the current memory
\item as statements whenever we wish to represent the result of arbitrary computations
\end{itemize}

References and statements are defined respectively with the $State$ predicate. Notice that we automatically associate
a reference to a state $st$ thanks to the type function 
$Ref$. We depart slightly from the standard representation 
of statements and stateful computations in that a statement in our system has different input and output types for the 
state, to allow the manipulation of the type of the memory
to be tracked.

\begin{lstlisting}
class State st where
  data Ref st :: * -> * -> *
  eval :: (Memory m) => Ref st m a -> st m m a
  (.=) :: (Memory m) => Ref st m a -> a -> st m m ()
  delete :: (Memory m) => st (Malloc a m) m ()
  new :: (Memory m) => a -> st m (Malloc a m) (Ref st (Malloc a m) a)
  (>>>=) :: (Memory m, Memory m', Memory m'') => st m m' a -> (a -> st m' m'' b) -> st m m'' b
  (>>>) :: (Memory m, Memory m', Memory m'') => st m m' a -> st m' m'' b -> st m m'' a
\end{lstlisting}

The $st$ functor applied to the memory $m$ twice is required by this definition to be a monad; thanks to this we can use our operators on references taking advantage of the syntactic sugar that Haskell offers, obtaining code that is much more intuitive to a programmer used to traditional object oriented languages.

We give an evaluation operator that evaluates (dereferences) 
a reference into a corresponding statement and an assignment 
operator to assign a reference a constant value.

The $new$ and $delete$ operators respectively add and remove a single value from our memory.

To concatenate regular statements and state transition statements we define the $(>>+)$ operator, which is a generalized binding operator: $(>>=)$ is defined in terms of $(>>+)$ for the state monad, since $(>>=)$ simply imposes the constraint that both parameters of $st$ are the same.

In our examples, we will use syntactic sugar that is not available in Haskell to make using the $(>>+)$ operator transparent; we call this the $do+$ notation.

We also define a shortcut for in-place modification of a 
reference:
\begin{lstlisting}
(*=) :: (State st,Memory s, Monad (st s s)) => Ref st s a -> (a->a) -> st s s ()
ref *= f = do v <- eval ref
              ref .= (f v)
\end{lstlisting}
