%----------------------------------------------------------------------------
%  intro.tex 
%----------------------------------------------------------------------------
In this introduction, we start with a brief presentation of 
monads: what they are, how they are used to hide boilerplate
code for repetitive operation, and we show one particular 
monad, the \textbf{state monad} \cite{1}.

We discuss \textbf{phantom types} and how they are used to 
tag terms with static information to perform various kinds 
of static analyses but without any performance hit whatsoever
since phantom types exist only at compile time and often have 
only $\bot$ as their sole inhabitant \cite{11,12,13}.

We study a bit of \textbf{type functions} and 
\textbf{type classes} to show what kind of power the Haskell
type system puts at our disposal \cite{10,8,6}.