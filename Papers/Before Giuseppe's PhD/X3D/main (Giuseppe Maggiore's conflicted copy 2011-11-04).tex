\documentclass{acm_proc_article-sp}

\usepackage{amssymb}
\usepackage{amsmath}
\usepackage{graphicx}
\usepackage{epsfig}
\usepackage{subfigure}
\usepackage{listings}
\usepackage{natbib}
\usepackage{verbatim}
\usepackage[T1]{fontenc} 
\usepackage[hyphens]{url}
\lstset{language=ml}
\lstset{commentstyle=\textit}
\lstset{mathescape=true}
\lstset{backgroundcolor=,rulecolor=}
\lstset{frame=single}
\lstset{breaklines=true}
\lstset{basicstyle=\ttfamily}

\begin{document}

\title{A compilation technique to increase X3D performance and safety}

\numberofauthors{4}
\author{
Giuseppe Maggiore \and Fabio Pittarello \and Michele Bugliesi \and Mohamed Abbadi\\
       \affaddr{Universit\`a Ca' Foscari Venezia}\\
       \affaddr{Dipartimento di Scienze Ambientali,}\\
       \affaddr{Informatica e Statistica}\\
       \email{\{maggiore,pitt,michele,mabbadi\}@dais.unive.it}
}

\date{}

\maketitle

\begin{abstract}
...
\end{abstract}

\category{D.1.1}{Programming Techniques}{Applicative (Functional) Programming} 
\category{D.2.2}{Soft\-ware Engineering}{Software Libraries}[Design Tools and Techniques]
\category{D.2.13}{Soft\-ware Engineering}{Reusable Software}[Domain engineering, Reusable libraries, Reuse models]
\category{D.3.3}{Programming Languages}{Language Constructs and Features}
\category{D.3.4}{Pro\-gramming Languages}{Processors}[Optimization, Run-time environments]
\category{H.5.1}{Information Systems}{Information Interfaces and Presentation}[Multimedia Information Systems]

\terms{Performance,Reliability,Languages}

\keywords{x3d, monads, compilation}

\section{Introduction}
\label{sec:intro}
%----------------------------------------------------------------------------
%  intro.tex 
%----------------------------------------------------------------------------
Modern computer languages are very reliable when it comes to writing a large class of common, real-world applications. For example, relatively simple form applications or web sites can be built extremely easily in languages such as Java, C\# and many others. This is thanks to commonplace facilities like garbage collectors, classes and inheritance and large libraries which simplify many tasks which otherwise would be hard or error-prone. On the other hand, there is a not so small set of applications for which these languages do not perform even nearly as well; for example games, even though very powerful libraries such as XNA make them easier to write by encapsulating many useful patterns, are not so suitable for modern languages. For this reason most games are still written in C++ (sometimes even in C) and the transition to higher level languages is not happening as fast as it could. As another example we could consider mobile applications. The widespread adoption of very powerful, fully programmable smartphone like the iPhone, Google Android or Windows Phone 7 makes performance even more important to achieve: lighter applications mean much better applications where CPU cycles and battery are both scarce resources. On the other hand, to allow as many developers as possible to easily create applications for these platforms, it makes sense (as indeed it is happening) to allow programming these devices with as languages that are as high-level as possible. Finally, there are many real time or soft real time applications that migh benefit from using high level languages but which cannot afford the pauses or slowdowns that sometimes the garbage collector might require, especially on less powerful hardware.


In this paper we document the results of implementing a videogame in such a high-level language (F\#) for a mobile device. Early in our development cycle we discovered that the biggest problem of our application was that by continually allocating and deallocating instances of the same types (such as projectiles and particles) we kept triggering the garbage collector, forcing the application to a crawl everytime it started. Rather than implement object pooling, we decided to try and generalize this technique by implementing reference counting \cite{7_6} inside the state monad \cite{1_1}. The state monad contains and manages all the actual instances of the objects which require reference counting, and references to these objects can only be accessed through the state monad itself. This allowed us to track the lifetime of our entities, in a lightweight enough fashion to ensure the increase in performance that we needed and transparently enough that it became convenient to track other resource types as well, such as files or GPU memory which require manual disposal.


In this paper we discuss a possible generalization of the work described above, where in addition to a monadic reference counting system we also discuss how type-level meta-programming \cite{3_1,3_2,3_3,6_1} and the parameterized monad \cite{1_7} can be used to track the types of resources with a strongly typed heterogeneous list \cite{4_1}, thereby removing even the need for the developer to "discover" what type the state must have. Also, this system could be the stepping stone for a more refined state-tracking monad which uses type-level values and phantom types to track expected properties of the state monad at various points in the program.


An important note is about the computer language in which we have written the samples. We have used a pseudo-Haskell, and we believe our listings may be turned into a working program without too much effort. This notwithstanding, it must be noted that said effort has not been made by us: our software system is mostly interested in performance, and from the point of view of benchmarking performance Haskell may not be the best suited language given its lazy evaluation strategy. For us this pseudo-Haskell has acted as a clear and easily implementable specification, and we do not claim anything about its workability in any present or future versions of the language.
 
 

\section{Solution Workflow}
\label{sec:solution_workflow}
%%%%%%%%%%%%%%%%%%%%%%%%%%%%%%%%%%%%%%%%%%%%%%%%%%%%%%%%%%
% solution_workflow.tex
%%%%%%%%%%%%%%%%%%%%%%%%%%%%%%%%%%%%%%%%%%%%%%%%%%%%%%%%%%

\begin{figure}
\begin{center}
\includegraphics[scale=0.6]{Solution_workflow.png}
\end{center}
\label{fig:solution_workflow}
\caption{Solution workflow}
\end{figure}

In Figure \ref{fig:solution_workflow} we can see a diagram depicting the steps used by our system when processing an X3D scene (plus its accompanying scripts). In the figure red blocks represent data while the blue blocks represent computations.
We start with an X3D file which describes our scene. This file may contain some scripts as routes, in \texttt{script} node, or the scripts may be stored into an external file. There are two layers of transformations described by our system:

\begin{itemize}
\addtolength{\itemsep}{-0.5\baselineskip}
\item a transformation from the original external scripts into our F\# scripts
\item a transformation from the original entities and routes of the X3D file into the final program
\end{itemize}

In the current development stage we have implemented part of the first transformation (we translate those scripts that are expressed as routes) and the second transformation (we can process any scene entity). The X3D scene and its routes are translated into F\# source code. This source code contains a type definition that describes the entire scene and the routes, plus an update function that translates the activation of scripts as a consequence of the changes in the entities that result from a user action or from the temporal evolution of the scene. Such conversion therefor supports the translation of regular X3D nodes that describe shapes and the various scene entities, and also routing nodes that describe basic scripts.

Our system supports \textit{external scripts}, that is any script that is not expressed as a route, but only if they are provided already translated to F\#; this means that our system cannot translate and process scripts written in Javascript and Java, but it can process those scripts if the translation has been done by hand. External scripts in F\# are validated against our type definitions, to ensure that they correctly access the scene. If their validation succeeeds, the final program is produced that integrates both the scene and all the scripts. This second processing gives our system support for any other scripts which are not easily expressed with routes.

We have used F\# \cite{FRIENDLY_FSHARP,FSHARP}, a multi-paradigm functional programming language targeting the .NET Framework. It is a variant of ML and is largely compatible with the OCaml implementation. F\# enjoys full support in the .NET Framework, meaning that it can take advantage of all .Net libraries (such as XNA for game development, which is most useful to us) and powerful IDEs such as Visual Studio and MonoDevelop.

\section{Compiling The Scene}
\label{sec:compiling_scene}
%%%%%%%%%%%%%%%%%%%%%%%%%%%%%%%%%%%%%%%%%%%%%%%%%%%%%%%%%%
% compiling_scene.tex
%%%%%%%%%%%%%%%%%%%%%%%%%%%%%%%%%%%%%%%%%%%%%%%%%%%%%%%%%%

In this section we show an outline of our compilation technique. 

The first step our compiler performs is deserializing the xml definition of our X3D scene. The scene is then processed and turned into a record, a type definition that describeds the static structure of our scene. The record contains:

\begin{itemize}
\addtolength{\itemsep}{-0.5\baselineskip}
\item a field for each static node of the scene; each field has the name of the node if the node has a \texttt{DEF} attribute
\item a field for a list of dynamic nodes
\item a field for a list of active scripts
\end{itemize}

A sample state for a scene with a timer and a box could be:

\begin{lstlisting}
type Scene  =
  {
    myClock       : Timer 
    box           : Box
    dynamic_nodes : List<Node>
    script        : Script
  }
\end{lstlisting}

Where \texttt{Timer} and \texttt{Box} are the concrete classes for a timer and a box respectively, and they both inherit from the \texttt{Node} class.
A list of nodes is needed to represent the dynamic portions of the scene, and a list of scripts maintains the sequence of currently running scripts.

This state definition is quite important, since it represents the interface between our scene and our scripts, and since it allows us fast lookups of specific nodes. Finding a node now just requires reading from a field in the state, an operation which is both fast and certain not to fail. For example, looking for the \texttt{time} field of the \texttt{"myClock"} node would simply require writing:

\begin{lstlisting}
scene.myClock.time
\end{lstlisting}

We then proceed to the initialization of the state. This amounts to creating instances of each node, and then assigning these instances to the fields of the \texttt{scene} variable.

An \texttt{update} function is then constructed that performs the update of all the statically known fields of the state, and which also executes the various routes of the scene. Also, the \texttt{update} function invokes the (dynamically dispatched) \texttt{update} function of each dynamic node; this is necessary because it would be unrealistic to hope that a complex virtual world can exclusively rely on statically known nodes, and a balance must be struck between optimizing static nodes and supporting dynamic ones.

The \texttt{update} function also performs a tick for all currently running scripts.

The update function that updates the state seen above would simply become:

\begin{lstlisting}
let update (dt:float32) =
  scene.myClock.update dt
  scene.box.update dt
  for node in scene.dynamic_nodes do
    node.update dt
  scene.script.update dt
\end{lstlisting}

We could add a simple route that moves the \texttt{box} node along the Y-axis according to the current time of the \texttt{myClock} node by adding the following line of code to the \texttt{update} function:

\begin{lstlisting}
  scene.box.Position.Y <- scene.myClock.Time
\end{lstlisting}

In general, routes are simple assignments when translated into our system.

\section{Representing the SAI}
\label{sec:compiling_scripts}
%%%%%%%%%%%%%%%%%%%%%%%%%%%%%%%%%%%%%%%%%%%%%%%%%%%%%%%%%%
% compiling_scripts.tex
%%%%%%%%%%%%%%%%%%%%%%%%%%%%%%%%%%%%%%%%%%%%%%%%%%%%%%%%%%

Scripting is a very important part of game development \citep{BETTER_SCRIPTS_GAMES}. For this reason we have adapted to our system a scripting solution that is derived from game engines. Whereas many game engines either use Lua, Python or even C\# as scripting languages (with various advantages and disadvantages) \cite{SCRIPTING_LUA,SCRIPTING_PYTHON, UNITY_YIELD} we have used F\# which we believe offers a powerful mix of the best features of all these languages: coroutines, flexibility and a lightweight syntax make F\# scripts similar to LUA and Python while static typing and support for .Net libraries and IDEs put F\# on par with C\# in terms of broader support.

To give additional expressive power to our scene, we add support to external scripts; external scripts are all those scripts that cannot be expressed in terms of routes. External scripts are very general, that is they can perform complex data conversions when copying values across entities, and they may even create, remove and modify nodes in any way possible. To represent external scripts, rather than using arbitrary objects that can access the state we have chosen to use coroutines, a widely used mechanism for representing computations in interactive applications \cite{PYTHON_COROUTINES,GPU_GEMS_6}. Coroutines are subroutines that can be suspended and resumed at certain locations. With coroutines the code for a SM is written ``linearly'' one statement after another, but each action may suspend itself (an operation often called ``yield'') many times before completing. A coroutine stores a temporary, internal state transparently inside its continuation.

We build a monadic framework \cite{COMPR_MON,DECL_IMP,EFF_MON,MOGGI_MON} for coroutines that allows us greater customization flexibility. This way we can define our own system for combining scripts running them in parallel, concurrently, etc. For a detailed discussion of this monadic framework for scripts and coroutines see \cite{X3D_TR1}.

A script in our system is defined as a normal F\# program surrounded by \texttt{\{ \}} brackets. A script runs another script with the statements \texttt{let!} and \texttt{do!}, and scripts can be combined with a small set of operators.

The main operators to combine scripts are:
\begin{itemize}
\item \texttt{parallel} ($s_1 \wedge s_2$) executes two scripts in parallel and returns both results
\item \texttt{concurrent} ($s_1 \vee s_2$) executes two scripts concurrently and returns the result of the first to terminate
\item \texttt{guard} ($s_1 \Rightarrow s_2$) executes and returns the result of a script only when another script evaluates to \texttt{true}
\item \texttt{repeat} ($\uparrow s$) keeps executing a script over and over
\end{itemize}

A sample script that moves the box \texttt{myBox} when the user enters a certain region \texttt{myRegion} could be the following:

\begin{lstlisting}
let my_script (scene:Scene) =
  let rec animate =
    script {
      if scene.myBox.Position.Y < 100.0f then
        scene.myBox.Position.Y <- scene.myBox.Position.Y + 0.1f
        do! animate }
  script {
    do! guard
         script {
           return inside(scene.Camera.Position, myRegion) }
         animate }
\end{lstlisting}

Notice that our script has a parameter of type \texttt{Scene}. If this parameter is used incorrectly (for example the scene this script is applied to does not have a \texttt{Box} node with name \texttt{myBox}) we will get a compile-time error. This makes it easier to build larger, reusable script modules since a mistake in using a pre-made module is easier to spot and requires less testing. Using scripts which have been made for different scenes would require extensive testing to ensure at least that all node accesses are correct.

Our scripting system is expressive enough to represent many scripts running together, even if at a first glance it may appear that our system supports only a single script. By using the \texttt{parallel} operator we can combine together a large number of scripts. For example, let us say we have many scripts $s_1, ... , s_n$ that must all run together with our scene. Each scripts has a different duration, that is the not all scripts will end at the same time (indeed, a script may even run indefinitely). The main script would chain each of the various actual scripts in the following manner:

\begin{lstlisting}
let my_script (scene:Scene) =
  parallel $s_1$ (parallel $s_2$ ... $s_n$) ... )
\end{lstlisting}
 

\section{A Case Study}
\label{sec:case_study}
%%%%%%%%%%%%%%%%%%%%%%%%%%%%%%%%%%%%%%%%%%%%%%%%%%%%%%%%%%
% case_study.tex
%%%%%%%%%%%%%%%%%%%%%%%%%%%%%%%%%%%%%%%%%%%%%%%%%%%%%%%%%%

We will now present a more detailed example to see our compiler in action by showing how it handles all the features of an X3D scene: entities and routes. We will consider an X3D scene that contains a looping timer which updates a color that in turn updates the material used when drawing a box:

\begin{lstlisting}[language=xml]
<Scene>
  <ColorInterpolator DEF='myColor'
    keyValue='1 0 0, 0 1 0, 0 0 1, 1 0 0'
    key='0.0 0.333 0.666 1.0'/>
  <TimeSensor DEF='myClock' cycleInterval='10.0' loop='true'/>
  <Shape>
    <Box/>
    <Appearance>
      <Material DEF='myMaterial'/>
    </Appearance>
  </Shape>
  <ROUTE fromNode='myClock' fromField='fraction_changed'
         toNode='myColor' toField='set_fraction'/>
  <ROUTE fromNode='myColor' fromField='value_changed'
         toNode='myMaterial' toField='diffuseColor'/>
</Scene>
\end{lstlisting}

Our compiler produces the following state definition from the above scene:

\begin{lstlisting}
type Scene =
  {
    myColor       : ColorInterpolator
    myClock       : TimeSensor
    myMaterial    : Material
    dynamic_nodes : List<Node>
    script        : Script
  }
\end{lstlisting}

where pointers to all statically known nodes are maintained.

The initialization function for our state initializes a set of local variables, one for each named node, and then builds the actual scene state. Notice that at this point routes are ignored, since they will be used only for the update function:

\begin{lstlisting}
let scene = 
  let myColor = 
       ColorInterpolator(
         keyValue = [ ... ],
         key = [ ... ])
  let myClock = 
       TimeSensor(
         cycleInterval = 10.0,
         loop = true)
  let myMaterial = Material()
  let dynamic_nodes = 
        [
          Shape(
            Value = 
              Box(Appearance(Value = myMaterial)))
        ]
  {
    myColor        = myColor
    myClock        = myClock
    myMaterial     = myMaterial
    dynamic_nodes  = dynamic_nodes
    script         = null
  }         
\end{lstlisting}

After initializing the scene without a script, we can load the script from an external parameter that will be assigned in the linking phase. Loading a script requires passing to it the scene, so that the script may access the scene to manipulate it:

\begin{lstlisting}
scene.script := load_script scene
\end{lstlisting}

The update function invokes the internal update function of all nodes, starting from the statically known and ending with the dynamic ones. Routes are executed in the update function:

\begin{lstlisting}
let update dt = 
  scene.myClock.update dt
  scene.myColor.update dt
  scene.myMaterial.update dt
  for node in scene.dynamic_nodes do
    node.update dt
  scene.script.update dt
  
  myColor.fraction <- myClock.fraction
  myMaterial.diffuseColor <- myColor.value
\end{lstlisting}

It is important to notice that routes in the update function are represented by the actual chains of field updates that need to be performed; there is no overhead when dynamically propagating the update events. Also, if a field does not start a route then there are no ``hidden'' costs as we would have when firing a \texttt{FieldModified} event with no routes listening.


\section{Benchmarks}
\label{sec:benchmarks}
%%%%%%%%%%%%%%%%%%%%%%%%%%%%%%%%%%%%%%%%%%%%%%%%%%%%%%%%%%
% benchmarks.tex
%%%%%%%%%%%%%%%%%%%%%%%%%%%%%%%%%%%%%%%%%%%%%%%%%%%%%%%%%%

Our system is mainly concerned with optimizing away the overhead that dynamically building and maintaining an X3D scene produces. To show that we have achieved our objective of increasing performance in X3D scenes with only regular nodes and routes (we have not yet profiled external scripts extensively), we have tested the same scene on multiple browsers and profiled the resulting framerates. The browsers we have used are BS Contact and Octaga.

We have tested for scenes with a relatively low number of shapes (300 and 680). We are not really interested in testing the rendering performance, since such a test would mainly compare the efficiency of the underlying rendering APIs and would not be relevant in this context. Both scenes are compared against two other scenes with the same shapes but with 3 \texttt{color interpolators}, 2 \texttt{timers} and 6 \texttt{routes} for each shape. The resulting routing and logic are quite heavy and constitutes a good test the underlying execution model for routes and logical nodes. The tested X3D files are an advanced version of the example seen in Section \ref{sec:case_study}: there is a (rather large) set of shapes, colors and timers and the colors of the shapes are changed according to the timers through heavy use of routes. This benchmark shows how heavy the traditional dynamic model is when handling many routes and large scenes.

Tables 1 through 3 show a comparison in performance for each browser with various hardware configurations; we have compared the performance of our implementation against Octaga and BS Contact, apart from the second testing machine where Octaga had trouble installing and then running at a reasonable speed (to avoid polluting the results we have omitted Octaga from that test):

\begin{table}[htb]\small
\centering
\begin{tabular}{|l|c|c|c|}
\hline
Browser & FPS & FPS (with routes) & Diff \% 	 \\
\hline
\textbf{Test machine 1}  & & & \\
\hline
Ours (300 shapes) & 580 & 510 & -12 \\
Ours (680 shapes) & 265 & 224 & -15 \\
Octaga (300 shapes) &  670 & 340 & -49 \\
Octaga (680 shapes) &  372 & 150 & -60 \\
BS C. (300 shapes) & 370 & 300 & -19 \\
BS C. (680 shapes) & 185 & 145 & -22 \\
\hline
\textbf{Test machine 2}  & & & \\
\hline
Ours (300 shapes) & 670 & 590 & -12 \\
Ours (680 shapes) & 310 & 265 & -15 \\
BS C. (300 shapes) & 530 & 368 & -31 \\
BS C. (680 shapes) & 285 & 146 & -49 \\
\hline
\textbf{Test machine 3}  & & & \\
\hline
Ours (300 shapes) & 640 & 600 & -6 \\
Ours (680 shapes) & 310 & 280 & -10 \\
Octaga (300 shapes) &  720 & 403 & -44 \\
Octaga (680 shapes) &  345 & 181 & -48 \\
BS C. (300 shapes) & 500 & 360 & -28 \\
BS C. (680 shapes) & 215 & 135 & -37 \\
\hline
\end{tabular}
\caption{Test results}
\end{table}

\begin{figure}
\begin{center}
\includegraphics[scale=0.2]{browsers.jpg}
\end{center}
\caption{WP7 Emulator, BS Contact and XNA Windows Application}
\end{figure}

It is clear that thanks to our approach the scene logic weighs far less than it does in the other browsers.

Moreover, as we can see in Figure 3, the code that is generated by our system can be run, \textit{without modification} also in Windows Phone 7 devices; in the figure we can see the same scene run in the Windows Phone 7 emulator (top left), our system (bottom left) and BS Contact (top right). 

Table 4 shows the performance of running, on a Windows Phone 7 device, two compiled scenes with 150 and 300 shapes respectively plus the same routes for each shape used in the tests discussed above. The performance is very good when considered that it is a mobile device; the same technique could be applied to other mobile devices such as iOS or Android, but in our case having used XNA a porting to Windows Phone 7 required literally no effort beyond modifying a compiler switch.

\begin{table}[htb]\small
\centering
\begin{tabular}{|l|c|}
\hline
Scene & FPS 	 \\
\hline
150 shapes with routes & 30 \\
300 shapes with routes & 24 \\
\hline
\end{tabular}
\caption{WP7 (LG Optimus 7)}
\end{table}

At this point we have completed supporting the static aspects of an X3D scene, those that are involved in nodes that are not added or removed dynamically. This approach clearly yields an increase in performance for scenes with a complex logic in terms of timers, routes, interpolators, etc.
 

\section{Conclusions and future work}
\label{sec:conclusions}
%
% conclusions.tex
%

Innovations in game architecture are often responsible for pushing further the boundaries of game design. By building more powerful \textit{and} more scriptable game engines it is possible at the same time to create games where game logic is faster and more complex, resulting in a more compelling experience for the player.

Building a system that allows for a smarter, individualized AI could usher an era of games that offer a smarter experience where units and non-playing characters make smart choices rather than waiting for ``obvious'' instructions from the player.

It must also be noted that games have been historically responsible for many evolutions of modern personal computing, by pushing the hardware to its limits and by offering powerful visual experiences that years or decades later become paradigms in non-gaming software. For this reason this kind of language evolution must be put in its right context: it is not just a way to make better games, but it is also a prototype for a way to write better \textit{programs} in general.

\paragraph{Future research directions in the field}
From a survey of the current panorama of game development tools and languages it appears that a technological shift is happening: from the almost exclusive use of C and C++ for creating games, plus the odd use of Lua or Python as scripting languages, more and more games are being made in modern managed languages such as C\# or Objective-C. The introduction of XNA (a framework that allows independent game developers to publish commercial games on the XBox 360 without costly licenses) has marked the beginning of the adoption of C\#-based frameworks for making games; other similar frameworks have been used in games between a C++ engine and the scripting system. This shift signals that the game development industry may be starting to have trouble with its old infrastructure given the rising development costs, the shrinking budgets and the decreasing development time and it is looking into new languages with more abstraction to increase its efficiency. For this reason many researchers (the authors of this report being part of this group) are studying new ways to leverage the power coming from the field of functional and declarative languages in order to more easily automate some repetitive programming tasks that often require complex and cumbersome libraries. 

\bibliographystyle{plain}
\bibliography{references} 

%\cite{*}
\nocite{}

\end{document}
