\documentclass[a4paper]{llncs}

\usepackage{amssymb}
\usepackage{amsmath}
\usepackage{graphicx}
\usepackage{epsfig}
\usepackage{subfigure}
\usepackage{listings}
\usepackage{natbib}
\usepackage{verbatim}
\usepackage{enumitem}
\usepackage[utf8]{inputenc}
\usepackage[T1]{fontenc}
\usepackage[hyphens]{url}
\usepackage{listings}
\usepackage[font=small,labelfont=bf]{caption}
\lstset{language=ml}
\lstset{commentstyle=\textit}
\lstset{mathescape=true}
\lstset{backgroundcolor=,rulecolor=}
\lstset{frame=single}
\lstset{breaklines=true}
\lstset{basicstyle=\footnotesize\ttfamily}


\begin{document}

\title{Resource Entity Action: A Generalized Design Pattern for RTS games}

\author{
Mohamed Abbadi\and Francesco Di Giacomo\and Renzo Orsini\and\\
Aske Plaat \and Pieter Spronck \and Giuseppe Maggiore\\
E-mail: \{mabbadi,fdigiacomo,orsini\}@dais.unive.it,
\{a.plaat,p.spronck\}@uvt.nl,
maggiore.g@nhtv.nl
}

\institute{Universit\`a Ca' Foscari DAIS - Computer Science, Venice, Italy \\[1mm]
Tilburg University, Netherlands\\[1mm]
NHTV University of Applied Sciences, Breda, Netherlands\\[1mm]
}

\date{}
\maketitle

\begin{abstract}
In many Real-Time Strategy (RTS) games, players develop an army in real time, then attempt to take out one or more opponents. Despite the existence of basic similarities among the many different RTS games, engines of these games are often built ad hoc, and code re-use among different titles is minimal. We created a design pattern called ``Resource Entity Action'' (REA) that abstracts the basic interactions that entities have with each other in most RTS games. This paper discusses the REA pattern and its language abstraction. We also discuss the implementation in the Casanova game programming language. Our analysis shows that the pattern forms a solid basis for a playable RTS game, and that it achieves considerable gains in terms of lines of code and runtime efficiency. We conclude that the REA pattern is a suitable approach to the implementation of many RTS games. 
\end{abstract}

\section{Introduction}
\label{sec:introduction}
%%%%%%%%%%%%%%%%%%%%%%%%%%%%%%%%%%%%%%%%%%%%%%%%%%%%%%%%%%
% intro.tex
%%%%%%%%%%%%%%%%%%%%%%%%%%%%%%%%%%%%%%%%%%%%%%%%%%%%%%%%%%

%%%%%%%%%%%%%%%%%%%%%%%%%%%%%%%%
%edit
%%%%%%%%%%%%%%%%%%%%%%%%%%%%%%%%
Computer games promise to be the next frontier in entertainment, with game sales being comparable to movie and music sales in 2010 \cite{ESA}. 

This unprecedented market prospects and potential for computer-game diffusion among end-users has created substantial interest in research on principled design techniques and on cost-effective development technologies for game architectures. Our present endeavour makes a step along these directions. 

Making games is an extremely complex business. Games are large pieces of software with many heterogeneous requirements, the two crucial being high quality and high performance \cite{GAME_OPT}. 


%%%%%%%%%%%%%%%%%%%%%%%%%%%%%%%%
%edit
%%%%%%%%%%%%%%%%%%%%%%%%%%%%%%%%
High-quality in games is comprised by two main factors: visual quality and simulation quality. Visual quality in games has made huge leaps forward, and many researchers continuously push the boundaries of real-time rendering towards photorealism. Simulation quality, on the other hand, is often lacking in modern games; game entities often react to the player with little intelligence, input controllers are used in simplistic ways and the logic of game levels is more often than not completely linear. Building a high-quality simulation is very complex in terms of development effort and also results in computationally expensive code. To make matters worse, gameplay and many other aspects of the game are modified (and often even rebuilt from scratch) many times during the course of development. For this reason game architectures require a lot of flexibility.

To manage all this complexity, game developers use a variety of strategies. Object-oriented architectures, components, reactive progamming, etc have all been used with some degree of success for this purpose \cite{COMPONENTS1,GAMEOBJECTS,FRP}. 

In this paper we will present the Casanova language, a language for making games. Casanova offers a mixed declarative/procedural style of programming which has been designed in order to facilitate game development. The basic idea of the language is to require from the developer only and exclusively those aspects of the game code which are specific to the game being developed. The language aims for simplicity and expressive power, and thanks to automated optimizations it is capable of generating code that is much faster than hand-written code and at no effort for the developer. The language offers primitives to cover the development of the game logic, and incorporates the typical processing of a game engine. Also, the language is built around a theoretical model of games with a ``well-formedness'' definition, in order to ensure that game code is always a good model of the simulated virtual world.

In the remainder of the paper we show the Casanova language in action. We begin with a description of the current state of game engines and game programming in Section \ref{sec:background}. In Section \ref{sec:model} we define our model of games. We describe the Casanova language in Section \ref{sec:casanova}.  We show an example of Casanova in action, and also how we have rewritten the game logic of an official XNA sample from Microsoft \cite{XNA_SAMPLES} in Casanova with far less code and higher runtime performance in Section \ref{sec:case_study}. In Section \ref{sec:conclusions} we discuss our results and some future work.


\section{Essential elements of RTS games}
\label{sec:the_problem}
RTS are a variation of strategy games where two or more players achieve specific (often conflicting) objectives by performing actions simultaneously in real time. The typical elements which arise from this genre are \textit{units} (characters, armies), \textit{buildings}, \textit{resources} and \textit{battle statistics}. Players command units to perform different types of actions. These actions can affect several entities in the game world.

Units and buildings are the entities that players control to achieve their objectives. Units usually fight or harvest resources, while buildings may be used to create new units or research upgrades. Resources are gathered from the playing field and fuel the economy of the game entities. Battle statistics determine the offensive and defensive abilities of units in a fight. This taxonomy of the elements of an RTS game can be applied successfully to multiple games: Starcraft, C\&C, and Age of Empires all feature units, buildings, resources, and battle statistics, amongst other elements.

In order to arrive at our design pattern we will now apply a simplification. Battle statistics can be interpreted as \textit{resources}, as for instance: ``the life of a unit is the cost for killing it, payable in attack power.'' We can also merge units and buildings together into a new category called \textit{entities}. This leads us to a simpler view of an RTS as a game that is based on Resources, Entities and Actions:

\begin{enumerate}
\item Resources: numerical values in the battle and economic system of the game. In this group we find the \textit{attack}, \textit{defense}, and \textit{life} patterns of entities. Resources also cover building materials and costs of production, deployment of units, development of new weapons, etc. (Resources are scalars.)
\item Entities: container for resources. They have physical properties and, as for the game logic, the difference among them lies only in the interactions. These interactions take place with resource exchanges through the actions. (Entities are vectors.)
\item Actions: resource flow among entities. Our model can be viewed as a directed weighted graph where the nodes are the entities, the weights are the amounts of exchanged resources, and the edges are the actions, that is, the elements which connect entities to one another. (Actions are transformation matrices.)
\end{enumerate}

Next, we discuss how we model Resources, Entities and Actions. 

\section{The REA design pattern}
\label{sec:the_idea}
In this section we will define a model for an algebra to show that the REA (Resource Entity Action) model can be reduced to a problem of linear algebra. We then show how games that use this model can be further simplified by linguistic constructs.

\subsection{Action algebra}

An action consists of a transfer of resources from a source entity to one or more target entities. We require that each entity has a resource vector, which contains the current amount of resources of the entity. The resource vector is sparse, since most actions involve only few resource types. An action is expressed by a transformation matrix $A$.

Consider a set of target entities $T = \left\lbrace t_{1},t_{2},...,t_{n}\right\rbrace$, which are the targets of the action, and a source entity $e$. Each entity $t_{i}$ (including the source entity type) has a resource vector
$\mathbf{r_{i}}=\left(r_{i_{1}},r_{i_{2}},...,r_{i_{m}} \right)$. The source entity also has a transformation matrix $A$ of size $m \times m$, which defines the interactions between all the resources of the source entity and all the resources of the target entities. We also consider an integrator $dt$ which contains the time difference between the current frame and the previous one. We then compute $\mathbf{w_{e}} = \left(  w_{e_{1}},w_{e_{2}},...,w_{e_{m}} \right) = \mathbf{r_{s}} \times A \cdot dt$. From the definition of matrix multiplication, it immediately follows that each component of $ \mathbf{w_{e}}$ represents how a resource will change by applying the effect of all the other resources to it. We compute the vector $\mathbf{r'_{i}} = \mathbf{r_{i}} + \mathbf{w_{e}} \; \forall e_{i} \in E$
which replaces the resource vector in each target entity.

For instance, consider the action of a spaceship (entity) using laser to damage (resource) an enemy spaceship (entity). This involves a  vector resource of two elements: laser and life points. The action must transfer laser points to subtract from the enemy life points. Suppose that the vector resource of the targeting ship is $r_{s} = (20,500)$ and the vector resource of the targeted ship is $r_{t} = (15,1000)$. Let the transformation matrix be
$A =
\begin{bmatrix}
0 & -1\\
0 & 0\\
\end{bmatrix}
$
which means that the source entity will reduce the life of the target by the number of its laser points.
Thus $w_{e} = r_{s} \times A  \cdot dt = (20,500) \times A  \cdot dt = (0,-20) \cdot dt$. At this point, assuming $dt = 1$ second, we have $r'_{t} = r_{t} + w_{e} = (20,1000) + (0,-20) \cdot dt = (20,980)$.

\subsection{A declarative language extension}

We now describe a language extension that implements the REA design pattern and its associated algebra for the Casanova game programming language \citep{Casanova}. The language extension is purely declarative. Its semantics are described using the SQL query language, which has the advantage of familiarity to most programmers.

%Implementing the action algebra is done using an abstract class which contains an abstract method which performs the action. Each action is a class which extends the previous abstract class and implements the abstract method. This method will fetch the world looking for the information needed to find what entities are affected by the action execution. Each entity of the game will have a collection of actions it can perform, automatically run by Casanova.

To identify the set of target entities $T$ given a source entity and its action, we create a new type definition called \textit{action}. An action is a declarative construct which is used to describe not only the resource exchange between entities, but also what kinds of entities participate in the exchange. The resource exchange is based on \textit{transfers} (Add, Subtract, and Set), while the target determination is based on \textit{predicates}: we filter the game world entities depending on their types, attributes and radius (specifying the distance beyond which the action is not applied). Some actions, called threshold actions, are not continuous and make use of special predicates to delay the execution (Output) until certain conditions are met.

Using actions it is possible to specify an exchange of resources in a fully declarative manner, so that the developer does not have to rewrite similar pieces of code ad hoc for each action. 

\section{Action syntax and semantics}
\label{sec:details}
We now give the syntax and semantics of actions in Casanova. The
grammar allows the definition of actions, which make up the body of
spatial Casanova entities which act as placeholders for
actions. When an entity contains such an action, the Casanova
runtime will apply it to all appropriate targets.

\subsection{Action Grammar Definition}

We first provide a taxonomy for actions. We divide the actions into three
kinds: (1) constant transfer actions, (2) mutable transfer actions, and
(3) threshold actions.

\textbf{Constant Transfer} actions update the target fields with a constant
value or a value taken from one of the source fields. The source field
is not affected by the resource transfer. An example of a constant transfer
action is a defense tower with infinite ammunition shooting an arrow at an infantry unit.

\begin{lstlisting}[language=sql]
TARGET Infantry; RESTRICTION Owner <> Owner; RADIUS 1000.0; TRANSFER CONSTANT Life - ArrowDamage;
\end{lstlisting}

\textbf{Mutable Transfer} actions are used when the resource exchange transfers resources from the source entity
to the target entity, or vice versa. An example of a mutable transfer action is a spaceship
transferring minerals from its holds to a shipyard.

\begin{lstlisting}[language=sql]
TARGET Shipyard; RESTRICTION Owner = Owner; RADIUS 150.0; TRANSFER MineralStash + Minerals;
\end{lstlisting}

\textbf{Threshold} actions follow the same transfer semantics as the previous
two types of actions. In addition, they have a collection of threshold
values and output operations. The output operations are executed once
when all the threshold values are reached. The threshold values are on
fields belonging to the source entity. The output operations modify
only fields of the source entity following the semantics of the
transfer operations. An example for a threshold action is a worker building a town hall. When
the \texttt{integrity} of the town hall reaches 100, a flag
\texttt{completed} is set (which is one of its fields) which warns the
system to replace the partially constructed building with the complete
building.
\begin{lstlisting}[language=sql]
TARGET ConstructionTownHall; RESTRICTION Owner = Owner; RADIUS 10.0; TRANSFER CONSTANT Integrity + 1.0; THRESHOLD Integrity = 100.0; OUTPUT Completed := true
\end{lstlisting}

Below we give a formal definition for the grammar instances presented in
the examples above, using the extended Backus-Naur form. A \emph{Casanova Entity} is an entity in the game
world represented as a record; the special keyword
\texttt{Self} is used to refer to the entity owning the action as one of
its fields.

\begin{lstlisting}
<Action> ::= TARGET <TARGET LIST> <RESTRICTION LIST> [<RADIUS CLAUSE>] <TRANSFER LIST>
   <INSERT LIST> [<THRESHOLD BLOCK>]
<TARGET LIST> ::= <ACTION ELEMENT>+
<ACTION ELEMENT> ::= Casanova Entity | Self
<RESTRICTION LIST> ::= {<RESTRICTION CLAUSE>}
<RESTRICTION CLAUSE> ::= RESTRICTION Boolean Expression of <SIMPLE PRED>
<SIMPLE PRED> ::= Self Casanova Entity Field (= | <>) Target Casanova Entity Field
<TRANSFER LIST> ::= {<TRANSFER CLAUSE>}
<TRANSFER CLAUSE> ::= (TRANSFER | TRANSFER CONSTANT)
(Target Casanova Entity Field) <Operator> ((Self Casanova Entity Field) | (Field Val)) [* Float Val]
<Operator> ::= + | - | :=
<RADIUS CLAUSE> ::= RADIUS (Float Val)
<INSERT LIST> ::= {<INSERT CLAUSE>}
<INSERT CLAUSE> ::= INSERT (Target Casanova Entity Field) -> (Self Casanova Entity Field List)
<THRESHOLD BLOCK> ::= <THRESHOLD CLAUSE>+
<OUTPUT CLAUSE>+
<THRESHOLD CLAUSE> ::= THRESHOLD
(Self Casanova Entity Field) Field Val
<OUTPUT CLAUSE> ::= OUTPUT
(Self Casanova Entity Field) <Operator> ((Self Casanova Entity Field) | (Field Val)) [* Float Val]
\end{lstlisting}

\subsection{Formal semantic definition}

Given the fact that actions resemble queries on entities, we specify
their semantics via translation to SQL. This allows us to
leverage existing discussions on SQL correctness \citep{SQLsemantic}.

In defining our translation rules formally, we consider a set $T =
\left\lbrace t_{1},t_{2},...,t_{n} \right\rbrace $ of target types and
a source entity type \textit{s}. In all actions we select a subset of
targets in each $t_{i}$ to which we apply the action, using
the specified restrictions. After that we apply the resource transfer.

We assume that each entity type is represented by an SQL relation and
that there exists a key attribute called \textbf{Id} for each
relation. We now consider each of the three translation cases.
In the translation rules we use notations inside the SQL code
taken from the Backus-Naur form for grammar definitions. We also extend the SQL grammar with a global
variable $dt$ which is the time difference between the current and the
last game frame. In this way the increments of the entity attribute
values can be made proportional to the elapsed time. All types of actions
evaluate the predicates in the restriction conditions and apply a
filter to their targets. All targets further than the radius
are automatically discarded when executing the action. The transfer
predicates are executed immediately on all filtered targets.

For a \textbf{CONSTANT TRANSFER} we must update each target with the value in the source fields or
constant values specified in the transfer clause. For simplicity, we
assume that constant values are stored as attributes of the
source entity.

Consider a set of resource attributes
$A = \left\lbrace  a_{j_{1}},a_{j_{2}},...,a_{j_{m}} \right\rbrace $
of the source entity used to update the target $t_{i}$. To
compute the contribution of all sources of the same type on the target
$t_{i}$, we specify a relation of which the tuples represent the
target id, followed by the total amount of resource $a_{j_{r}}$ to
transfer, called $\Sigma_{r}$:

\begin{center}
\begin{tabular}
{| c | c | c | c | c |}
\hline
\multicolumn{5}{|c|}{Transfer} \\
\hline
$ID$ & $\Sigma_{1}$ & $\Sigma_{2}$ & $\cdots$ & $\Sigma_{m}$ \\
\hline
\end{tabular}
\end{center}

The following SQL instruction implements the relation definition above:
\begin{lstlisting}[language=sql]
SELECT	$t_{i}$.id, SUM($s.a_{_j{1}}$} AS $\Sigma_{1}$,
	SUM($s.a_{_j{2}}$} AS $\Sigma_{2}$,...,
	SUM($s.a_{_j{m}}$} AS $\Sigma_{m}$

FROM	Target $t_{i}$, Source $s$
WHERE	<RESTRICTION LIST> [AND <RADIUS CLAUSE>]
GROUP BY $t_{i}.id$
\end{lstlisting}

$\forall t_{i} \in T$ we update the target attributes $A' =
\left\lbrace a_{t_{1}},a_{t_{2}},...,a_{t_{m}} \right\rbrace$ using
one of the target operators defined in the grammar (Set, Add,
Subtract) with the attributes of the previous relation scheme.
\begin{lstlisting}
WITH	Transfer AS(
		SELECT	$t_{i}$.id, SUM($s.a_{_j{1}}$} AS $\Sigma_{1}$,
			SUM($s.a_{_j{2}}$} AS $\Sigma_{2}$,...,
			SUM($s.a_{_j{m}}$} AS $\Sigma_{m}$)

FROM	Target $t_{i}$, Source $s$
WHERE	[<RESTRICTION LIST>] [AND <RADIUS CLAUSE>]
GROUP BY $t_{i}.id$)
UPDATE	Target $t_{i}$
SET	$t_{i}.a_{t_{1}}$ = u.$\Sigma_{1}$ | $t_{i}.a_{t_{1}}$ = $t_{i}.a_{t_{1}}$ + u.$\Sigma_{1}$ * $dt$ | $t_{i}.a_{t_{1}}$ =
$t_{i}.a_{t_{1}}$ - u.$\Sigma_{1}$ * $dt$\
$\cdots$
FROM	Transfer $u$
WHERE	$u.id = t_{i}.id$
\end{lstlisting}

For a \textbf{MUTABLE TRANSFER} the field of the source involved in the resource
transfer is updated depending on the applied transfer
operator. The resource is subtracted from
the source field and added to the target field proportionally to
$dt$, or vice versa.

To translate this semantic rule we must first determine how many
targets (if any) are affected by each source entity, in order to
obtain the following relation scheme:

\begin{center}
\begin{tabular}
{| c | c |}
\hline
\multicolumn{2}{|c|}{TotalTargets} \\
\hline
\textit{Source ID} & \textit{TargetCount} \\
\hline
\end{tabular}
\end{center}

The SQL code implementing the previous scheme is the following:

\begin{lstlisting}[language=sql]
TotalTargets =
SELECT	s.id,COUNT(*) AS TargetCount
FROM	Source s, Target $t_{1}$, Target $t_{2}$,...,Target $t_{n}$
WHERE	<RESTRICTION LIST> [AND <RADIUS CLAUSE>]
GROUP BY s.id
HAVING	COUNT(*) > 0
\end{lstlisting}

$\forall t_{i} \in T$ we need to obtain a relation storing what target each of the source entities is affecting, including a count of affected targets, using the following relation scheme:

\begin{center}
\begin{tabular}
{| c | c | c |}
\hline
\multicolumn{3}{|c|}{OutputSharing} \\
\hline
\textit{Source ID} & \textit{Target ID} & \textit{Output Sharing}\\
\hline
\end{tabular}
\end{center}

This scheme is implemented by the following SQL code: % there was a vague remark here on the use of RelationName = [...], which I did not understand so I removed it.

\begin{lstlisting}[language=sql]
OutputSharing =
SELECT	*
FROM	TotalTargets c, SourceOutput c1
WHERE	c.s_id = c1.s_id
	AND SourceOutput =
		SELECT	$s.id$ AS s_id,$t_{i}.id$ AS t_id
		FROM	Source s, Target $t_{i}$
		WHERE	<RESTRICTION LIST> [AND <RADIUS CLAUSE>]
	AND TotalTargets = [...]
\end{lstlisting}

Each target attribute receives an amount of resources equal to the total transferred resources divided by the number of targets. The complete SQL code to update the target $t_{i}$ is the following:

\begin{lstlisting}[language=sql]
WITH	Transfer AS(
	SELECT	$t_{i}$.id, SUM($s.a_{j_{1}}$ / o.TargetCount) AS $\Sigma_{0}$,SUM($s.a_{j_{2}}$ / o.TargetCount) AS $\Sigma_{2}$,...,SUM($s.a_{j_{m}}$ / o.TargetCount) AS $\Sigma_{m}$
	FROM	Source s, Target $t_{i}$,OutputSharing o
	WHERE	OutputSharing = [...] AND s.id = o.s_id AND t.id = o.t_id)
	GROUP BY $t_{i}$.id
UPDATE	Target $t_{i}$
SET	$t_{i}.a_{t_{1}}$ = u.$\Sigma_{1}$ |$t_{i}.a_{t_{1}}$ = $t_{i}.a_{t_{1}}$ + u.$\Sigma_{1}$ * $dt$ |$t_{i}.a_{t_{1}}$ = $t_{i}.a_{t_{1}}$ - u.$\Sigma_{1}$ * $dt$
$\cdots$
FROM	Transfer u
WHERE	$t_{i}$.id = u.id
\end{lstlisting}

To update the Source relation we use a relation similar to the one used to update the target, but this time there is no need to save the count of the affected targets.

\begin{lstlisting}[language=sql]
WITH	TotalTransfer AS(
	SELECT	s.id,s.$a_{j_{1}}$,s.$a_{j_{2}}$,...,s.$a_{j_{m}}$
	FROM	Source s, Target $t_{1}$,...,Target $t_{n}$
	WHERE	<RESTRICTION LIST>
		[AND <RADIUS CLAUSE>]
	GROUP BY	s.id,$s.a_{j_{1}}$,$s.a_{j_{2}}$,...,$s.a_{j_{m}}$
	HAVING 	COUNT(*) > 0)
	
UPDATE	Source s
SET	s.$a_{j_{1}}$ = s.$a_{j_{1}}$ - s.$a_{j_{1}} * dt$|s.$a_{j_{1}}$ = s.$a_{j_{1}}$ + s.$a_{j_{1}} * dt $
$\cdots$
FROM	TotalTansfer u
WHERE	s.id = u.id
\end{lstlisting}

The \textbf{THRESHOLD} action is defined as the previous two types, i.e., it has a resource transfer definition which is always executed, and a set of threshold conditions that, if met, activate the Output operations which are always towards the source entity. The attributes of the source entity affected by Output operations are updated with constant values, or with values from other attributes in the source entity. In the latter case the transfer is treated as for the mutable transfer case.

Consider a set of updating attributes $U = \left\lbrace a_{k_{1}},a_{k_{2}},...,a_{k_{l}} \right\rbrace$ and a set of attributes to be updated $U'= \left\lbrace a_{s_{1}},a_{s_{2}},...,a_{s_{l}} \right\rbrace$ in the output operation. We first check that all the conditions in the threshold clauses are met, then we update the attributes in the source entity appropriately.

\begin{lstlisting}[language=sql]
WITH	TotalOutput AS(
	SELECT	s.id,s.$a_{k_{1}}$,s.$a_{k_{2}}$,...,s.$a_{k_{l}}$
	FROM	Source s
	WHERE	<THRESHOLD CLAUSE 1>
		[AND <THRESHOLD CLAUSE 2>]
		.
		.
		.
		[AND <THRESHOLD CLAUSE $l$>])
UPDATE	Source s
SET	s.$a_{s_{1}}$ = o.$a_{k_{1}}$|s.$a_{s_{1}}$ = (s.$a_{s_{1}}$ + o.$a_{k_{1}}) * dt$; o.$a_{k_{1}}$ = o.$a_{k_{1}}$ - o.$a_{k_{1}} * dt$|s.$a_{s_{1}}$ = (s.$a_{s_{1}}$ - o.$a_{k_{1}}) * dt$; o.$a_{k_{1}}$ = o.$a_{k_{1}}$ + o.$a_{k_{1}} * dt$
$\cdots$
FROM	TotalOutput o
WHERE	s.id = o.id
\end{lstlisting}

%\subsection{Casanova implementation}

%The process of evaluating actions was added to Casanova, which, using a compiler, generates assembly code specific for each action. The generated code executes the actions at each game frame. Besides, the compiler checks that the targets are valid and that the fields used in all the predicates are contained in those entities. To improve performance an index is built at compile time, to speed up resolution of radius restrictions. The implementation uses type attributes for actions, so the syntax is different even though there is a mapping between elements of the syntax presented here and those of the concrete syntax. 

\section{Case study}
\label{sec:evaluation}
We now discuss the main issues that we originally set out to address (see Section \ref{sec:problem}). Some issues are addressed through an analytical evaluation, some are based on rigorous performance benchmarks, and some are based on user studies.

\subsection{Comparison with existing approaches}
As a point of comparison, we have considered three representative networking samples: a tiny C\# game with the Lidgren networking library, a chat program in Erlang, and a chat in C\# with Windows Communication Foundation (WCF). Each sample provides a different target of comparison: \textit{(i)} the C\#/Lidgren sample is a low level framework that only supports basic connections and transmission of primitive data across the network; \textit{(ii)} Erlang is a highly concurrent language centred around networked communications and concurrency; and \textit{(iii)} WCF is an extremely high level system for handling remote communications.

Erlang is a concurrency and networking-oriented language. It excels at almost anything related to networking, besides two aspects: \textit{(i)} dynamic typing is wasteful in terms of performance, severely limiting the usability of the language in game clients; and \textit{(ii)} it does not have a concept of local/remote entities, requiring the developer to build it by hand for every game.

C\#/Lidgren, being a very low level solution, barely simplifies any networking-related task. There is no support for concurrency (in terms of automatically dispatching messages to the appropriate listener), nor is there any support for fast automated serialization. Finally, there is no support for tracking locally and remote entities. To counterbalance these negative aspects though, Lidgren allows programmers to hand-optimize almost everything, to great potential in terms of performance.

C\#/WCF is a very high level solution. This means that it tends to lack in performance, because of the use of reflection-based serialization, and it has no support for local/remote entities. Besides this, though, it is quite simple and immediate to use, if one ignores the (usually automatically generated) very large configuration files that it needs.

The results are summarized in Table \ref{tab:comparison with existing approaches}. In conclusion, Casanova offers a set of primitives that are similarly powerful to those of Erlang or other high level solutions, with a performance profile that is more appropriate for games, and with additional syntactic and semantic support for the central game-related concept of local/remote entities.

\begin{table}
\center
\begin{tabular}{ c p{2cm} p{2cm} p{2cm} p{4cm} p{2cm}  }
\hline
Language & Concise & Serializes & Concurrent & Performance & Local/remote \\
\hline
Erlang   & Very    & Yes        & Yes        & Medium (dynamic typing, no incremental updates) & No \\
C\#/Lidgren & No   & No         & No         & High        & No \\
C\#/WCF  & Partially (large config files) & Yes & Yes & Medium (slow serialization) & No \\
\hline
\end{tabular}
\label{tab:comparison with existing approaches}
\caption{Comparison with existing approaches}
\end{table}


\subsection{Simplicity and expressive power}
Simplicity and expressive power are two perspectives on the same issue. A good programming language should be \textit{expressive enough} to express solutions in the chosen problem domain. A good programming language should also be \textit{not too much expressive} as to offer too many subtly overlapping ways to solve the same problem.

When a language is not expressive enough, then either some programs cannot be written in it at all (for example a language lacking recursion mechanisms or unbound loops), or (as is more often the case) some programs become very hard to build.

Casanova networking is both simple and has the right expressive power. Expressive power has been assessed by building a series of different games with networking in Casanova. The games belong to multiple genres: a shooting game, an RTS, an RPG, and a fighting game. Even if the games have been built at a simplified level (building full-fledged AAA games would not be feasible in a research context), they feature many important aspects of networking: lobby mechanisms, interpolation/extrapolation, limited use of bandwidth, fault tolerance and player disconnection, etc. The primitives offered by Casanova are amply sufficient to express all of these mechanisms with \textit{relative ease}. Relative ease means that networking is still a complex domain, which requires reasoning in terms of many instances at the same time: the underlying thinking processes that the developer must perform remain quite complex. The games are available as Open Source Software at \cite{CNV_MULTIPLAYER_SAMPLES}.

Simplicity has been assessed with a user study performed with a group of first, second, and third year game development students at NHTV University of Applied Sciences. All students were required to study and understand the behaviour of the samples above. After a given amount of time they received a questionnaire that they filled in. The questionnaire measured their understanding, which on average was high enough to be satisfactory with regards to the understandability of the language. Given the short time frame of the experiment, and the lack of previous knowledge of Casanova, we believe that in this case \textit{understandability strongly correlates with simplicity}. The questionnaire is also stored online at \cite{CNV_MULTIPLAYER_USERSTUDY_QUESTIONNAIRE}. The results of the test are found in Table \ref{tab:user study}.

\begin{table}
\center
\begin{tabular}{ l c r }
  1 & 2 & 3 \\
  4 & 5 & 6 \\
  7 & 8 & 9 \\
\end{tabular}
\label{tab:user study}
\caption{Questionnaire results}
\end{table}

We also observe that simplicity in Casanova networking comes from a specific factor. The number of primitives offered is really small. Four keywords for sending and receiving, and three keywords for specifying ownership and connections. A total of seven keywords is really a small set for something as big as building multi-player games. Existing libraries, when compared to Casanova networking, are far larger, at least an order of magnitude, when counting the number of primitives they offer. In this case, we counted non-trivial methods as primitives. The results of this comparison are summarized in Table \ref{tab:networking libraries}.

\begin{table}
\center
\begin{tabular}{ l c r }
  1 & 2 & 3 \\
  4 & 5 & 6 \\
  7 & 8 & 9 \\
\end{tabular}
\label{tab:networking libraries}
\caption{Comparison with other libraries}
\end{table}

\subsection{Robustness}
The ability of a game to function even in the presence of network issues or client disconnections is a very important issue. Wireless or mobile networks are unreliable, the Internet loses many messages that are sent with faster UDP connections, and game instances are often (purposefully or by accident) disconnected from the game.

Most of the networking primitives of Casanova simply never fail. Only \texttt{send\_reliably}, on the other hand, blocks a rule execution until the networking operation is not completed. Casanova offers a rather aggressive, but reasonable, behaviour in the case of such failures (which have a large enough timeout to ensure no premature response!): the unreachable party will be forcibly disconnected from the game, and the other will have a chance to abort execution of the rule because at that point \texttt{send\_reliably} will return \texttt{false}. Of course it is possible to add manual handling of these failures, in order to avoid flat-out disconnections, for example by writing:

\begin{lstlisting}
{
  send_reliably<T>(x)
  ...
} || { 
  wait(timeout)
  ...
}
\end{lstlisting}

In this case we are requiring that all operations are completed within a reasonable time-out, or else we locally abort rule execution without forcing the other party to disconnect in case of failed transmission.

Finally, when a peer disconnects, then Casanova automatically removes it from the list of peers. This guarantees that we will not send or receive anything to and from this peer any more. Disconnection may result in \textit{orphaned entities}, which are entities that are only stored as slaves in all available instances. The orphaned entities can be handled with a brief voting session, where the first player to claim the entity will assign it to himself and signal to the others that they must connect to the new master entity. Alternatively, the various instances may be programmed to simply discard the orphaned entities, for example the avatar of a disconnected player.

\subsection{CPU usage}
The overhead of the Casanova networking libraries is rather small. As a benchmark, we have run a dummy instance of a game where all networked primitives immediately return a predefined value, instead of performing any actual networking operation. The difference in time for a single tick of the simulation was, on average, quite small. The results are summarized in Table \ref{tab:cpu usage}.

\begin{table}
\center
\begin{tabular}{ l c r }
  1 & 2 & 3 \\
  4 & 5 & 6 \\
  7 & 8 & 9 \\
\end{tabular}
\label{tab:cpu usage}
\caption{CPU usage}
\end{table}

\subsection{Bandwidth and latency limitations}
Bandwidth usage is a difficult aspect to assess, as it is heavily dependent on implementation details. For example, an application that often relies on \texttt{send<T>} for entities \texttt{T} which contain large amounts of data is going to use more bandwidth than an application that does the same less often. In turn, an application that only sends primitive values such as \texttt{send<bool>}, \texttt{send<int>}, and \texttt{send<float>} and which does so seldom, will use even less bandwidth. In short, depending on the requirements of the application, for example running inside a fast LAN or across the Internet, developers may choose a heavier or more optimized approach.

We show that it is \textit{possible} to reach very limited bandwidth usage. We built the samples \cite{CNV_MULTIPLAYER_SAMPLES} with limited bandwidth in mind. Usage is well within the suggested maximum average bandwidth required for modern games. Bandwidth usage also heavily influences latency. Few transmissions, uncongested, result in lower latency. The results are summarized in Table \ref{tab:bandwidth and latency}.

\begin{table}
\center
\begin{tabular}{ l c r }
  1 & 2 & 3 \\
  4 & 5 & 6 \\
  7 & 8 & 9 \\
\end{tabular}
\label{tab:bandwidth and latency}
\caption{Bandwidth and latency}
\end{table}


%\section{Comparison with related systems}
%\label{sec:related_work}
%Currently there are many commercial and open source solutions for developing RTS games which result to be often too specific, thus inflexible or not scalable. When users would like to extend these frameworks, this often turns out to be difficult, if not impossible, unless they change the entire structure of the project by changing the structures of entities and the connections among them.
There are not many specific RTS engines but some of the most common used are listed below.
\subsubsection*{Game maker:}
Game maker is a tool which joins the visual development with a limited scripting language. The scripting language allows only the use of strings and real numbers, possibly indexed as arrays. However, it is neither possible to pass an array as a script argument nor accessing it with a pointer except by passing a string holding the name of the array itself. R.E.A. instead is an extension of a well defined and structured programming language like Casanova with no such limitations and workarounds.
\subsubsection*{ORTS – Open real-time strategy engine:}
ORTS is a domain specific language for making RTS games based on scripts. The language of the scripts is limited; what is not supported by its primitives must be written in C. However, this lack of expressiveness is compensated by being domain specific. ORTS is not designed to be very general because it is specific only for the RTS genre, and in particular for RTS's without an articulated logic. Finally, native optimization, which is provided by our solution, is not possible in ORTS, as explained above, unless the developer codes it by himself.
\subsubsection*{Spring engine:}
Spring engine is a framework for creating RTS games. The engine specifies predefined boundaries on game dynamics, which cannot be extended. The developer has to learn a long series of keywords. Moreover, getting out of the predefined context, requires to code in a different semantic level using scripting languages such as LUA. However, Spring engine is a good RTS framework which implements a wide variety of options and, in some cases, native optimization (such as spatial optimization for collision detection). Spring engine also presents the same problems, as for the scripting language, of the other engines listed above.

\section{Conclusions}
\label{sec:conclusions}
%
% conclusions.tex
%

Innovations in game architecture are often responsible for pushing further the boundaries of game design. By building more powerful \textit{and} more scriptable game engines it is possible at the same time to create games where game logic is faster and more complex, resulting in a more compelling experience for the player.

Building a system that allows for a smarter, individualized AI could usher an era of games that offer a smarter experience where units and non-playing characters make smart choices rather than waiting for ``obvious'' instructions from the player.

It must also be noted that games have been historically responsible for many evolutions of modern personal computing, by pushing the hardware to its limits and by offering powerful visual experiences that years or decades later become paradigms in non-gaming software. For this reason this kind of language evolution must be put in its right context: it is not just a way to make better games, but it is also a prototype for a way to write better \textit{programs} in general.

\paragraph{Future research directions in the field}
From a survey of the current panorama of game development tools and languages it appears that a technological shift is happening: from the almost exclusive use of C and C++ for creating games, plus the odd use of Lua or Python as scripting languages, more and more games are being made in modern managed languages such as C\# or Objective-C. The introduction of XNA (a framework that allows independent game developers to publish commercial games on the XBox 360 without costly licenses) has marked the beginning of the adoption of C\#-based frameworks for making games; other similar frameworks have been used in games between a C++ engine and the scripting system. This shift signals that the game development industry may be starting to have trouble with its old infrastructure given the rising development costs, the shrinking budgets and the decreasing development time and it is looking into new languages with more abstraction to increase its efficiency. For this reason many researchers (the authors of this report being part of this group) are studying new ways to leverage the power coming from the field of functional and declarative languages in order to more easily automate some repetitive programming tasks that often require complex and cumbersome libraries.


\bibliographystyle{plain}
\bibliography{Sections/reference}

\end{document} 