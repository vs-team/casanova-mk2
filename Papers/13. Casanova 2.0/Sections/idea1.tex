In this section we discuss a model for computation which was directly inspired by the ``orchestration model'' \cite{misra2007computation}, and we show how this model simplifies the example above.

\subsection{Our model}
We begin with the observation that components can be treated as independent and concurrent programs.
Our model is tiny since it consists of four operators:  (\texttt{$\rightarrow$}) the bind operator, (\texttt{||}) the parallel operator, (\texttt{$\downarrow$}) the yield operator, and (\texttt{let}) the expression definition operator.

Given two concurrent programs, \texttt{g} and \texttt{f}: \texttt{f$\rightarrow$g} runs \texttt{g} after \texttt{f} has terminated. The expression becomes \texttt{f>g} if we do not want any result of \texttt{f} to enter in the scope of \texttt{g}); \texttt{(f||g)} runs \texttt{f} and \texttt{g} in parallel. \texttt{$\downarrow x$, v} changes the value of v into x (x is a variable defined either as an argument of the current function or a variable defined outside the function itself); \texttt{let} allows the assignment of expressions to labels in the form of \texttt{let name = expr}. The \textit{state} is made up from the definition of all the variables involved in the logic.


\subsection{Re-implementation of the case study}
We now show how our model can express the dynamic of a game by using it to re-write our case study. 
The implementation is divided into three blocks:
\begin{enumerate}
\item \texttt{UPDATE}, implements the logic ;
\item \texttt{DRAW}, draws the entities of the game; 
\item \texttt{GAME}, coordinates the execution of both \texttt{UPDATE} and \texttt{DRAW}.
\end{enumerate}

The new implementation of the game is now given:
In the \texttt{UPDATE} block we define the underlying logic of the game. The key pressure is caught by \texttt{WaitSpacePressed} which waits until the \texttt{space} key is first pressed and then released. The logic of the light switch waits until the space key is pressed, and once this event occurs it turns \texttt{on} the light switch. We turn back to \texttt{off} when we press again the space button. Afterwards, it recurses.
\begin{lstlisting}[caption=LOGIC]
let WaitSpacePressed = 
  wait (Keyboard.Down(Space)) $\rightarrow$
  wait (Keyboard.Up(Space))
let Update sprite_color tick_status = 
  WaitSpacePressed $\rightarrow$
  $\downarrow$ tick_status, Status.Play $\rightarrow$
  $\downarrow$ sprite_color Color.White $\rightarrow$  
  WaitSpacePressed $\rightarrow$
  $\downarrow$ tick_status, Status.Play $\rightarrow$
  $\downarrow$ sprite_color Color.Black $\rightarrow$
  Update sprite_color tick_status
\end{lstlisting}
The \texttt{DRAW} block draws the light switch (which is called \texttt{sprite}) and then recurs.
\begin{lstlisting}[frame=single, caption=DRAW]
let Draw sprite = DrawSprite sprite $\rightarrow$ Draw
\end{lstlisting}
The \texttt{GAME} block loads the light switch sprite (setting its initial color to \texttt{Black}), and runs \texttt{UPDATE} and \texttt{DRAW} in parallel.
\begin{lstlisting}[frame=single, caption=LOOP]]
let sprite = LoadSprite "mySprite.png"
let sound = LoadSoundManager()
let Game = Update(sprite.Color, sound.Status) || 
           Draw(sprite)				
\end{lstlisting}
The spurious constructs used to maintain the state machine are no longer present explicitly, but stored and handled implicitly in the form of \textit{program counters} of the various processes. This is due to the control flow semantics of our model. The resulting program has more in common (from the perspective of readability) with the \textit{definition} of the light switch problem than the program presented in Section 2.


We will now proceed with the definition of the Casanova 2 programming language that is inspired by the concepts of orchestration languages. 