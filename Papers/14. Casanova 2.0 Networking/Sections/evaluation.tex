We now discuss the main issues that we originally set out to address (see Section \ref{sec:problem}). Some issues are addressed through an analytical evaluation, some are based on rigorous performance benchmarks, and some are based on user studies.

\subsection{Comparison with existing approaches}
As a point of comparison, we have considered three representative networking samples: a tiny C\# game with the Lidgren networking library, a chat program in Erlang, and a chat in C\# with Windows Communication Foundation (WCF). Each sample provides a different target of comparison: \textit{(i)} the C\#/Lidgren sample is a low level framework that only supports basic connections and transmission of primitive data across the network; \textit{(ii)} Erlang is a highly concurrent language centred around networked communications and concurrency; and \textit{(iii)} WCF is an extremely high level system for handling remote communications.

Erlang is a concurrency and networking-oriented language. It excels at almost anything related to networking, besides two aspects: \textit{(i)} dynamic typing is wasteful in terms of performance, severely limiting the usability of the language in game clients; and \textit{(ii)} it does not have a concept of local/remote entities, requiring the developer to build it by hand for every game.

C\#/Lidgren, being a very low level solution, barely simplifies any networking-related task. There is no support for concurrency (in terms of automatically dispatching messages to the appropriate listener), nor is there any support for fast automated serialization. Finally, there is no support for tracking locally and remote entities. To counterbalance these negative aspects though, Lidgren allows programmers to hand-optimize almost everything, to great potential in terms of performance.

C\#/WCF is a very high level solution. This means that it tends to lack in performance, because of the use of reflection-based serialization, and it has no support for local/remote entities. Besides this, though, it is quite simple and immediate to use, if one ignores the (usually automatically generated) very large configuration files that it needs.

The results are summarized in Table \ref{tab:comparison with existing approaches}. In conclusion, Casanova offers a set of primitives that are similarly powerful to those of Erlang or other high level solutions, with a performance profile that is more appropriate for games, and with additional syntactic and semantic support for the central game-related concept of local/remote entities.

\begin{table}
\center
\begin{tabular}{ c p{2cm} p{2cm} p{2cm} p{4cm} p{2cm}  }
\hline
Language & Concise & Serializes & Concurrent & Performance & Local/remote \\
\hline
Erlang   & Very    & Yes        & Yes        & Medium (dynamic typing, no incremental updates) & No \\
C\#/Lidgren & No   & No         & No         & High        & No \\
C\#/WCF  & Partially (large config files) & Yes & Yes & Medium (slow serialization) & No \\
\hline
\end{tabular}
\label{tab:comparison with existing approaches}
\caption{Comparison with existing approaches}
\end{table}


\subsection{Simplicity and expressive power}
Simplicity and expressive power are two perspectives on the same issue. A good programming language should be \textit{expressive enough} to express solutions in the chosen problem domain. A good programming language should also be \textit{not too much expressive} as to offer too many subtly overlapping ways to solve the same problem.

When a language is not expressive enough, then either some programs cannot be written in it at all (for example a language lacking recursion mechanisms or unbound loops), or (as is more often the case) some programs become very hard to build.

Casanova networking is both simple and has the right expressive power. Expressive power has been assessed by building a series of different games with networking in Casanova. The games belong to multiple genres: a shooting game, an RTS, an RPG, and a fighting game. Even if the games have been built at a simplified level (building full-fledged AAA games would not be feasible in a research context), they feature many important aspects of networking: lobby mechanisms, interpolation/extrapolation, limited use of bandwidth, fault tolerance and player disconnection, etc. The primitives offered by Casanova are amply sufficient to express all of these mechanisms with \textit{relative ease}. Relative ease means that networking is still a complex domain, which requires reasoning in terms of many instances at the same time: the underlying thinking processes that the developer must perform remain quite complex. The games are available as Open Source Software at \cite{CNV_MULTIPLAYER_SAMPLES}.

Simplicity has been assessed with a user study performed with a group of first, second, and third year game development students at NHTV University of Applied Sciences. All students were required to study and understand the behaviour of the samples above. After a given amount of time they received a questionnaire that they filled in. The questionnaire measured their understanding, which on average was high enough to be satisfactory with regards to the understandability of the language. Given the short time frame of the experiment, and the lack of previous knowledge of Casanova, we believe that in this case \textit{understandability strongly correlates with simplicity}. The questionnaire is also stored online at \cite{CNV_MULTIPLAYER_USERSTUDY_QUESTIONNAIRE}. The results of the test are found in Table \ref{tab:user study}.

\begin{table}
\center
\begin{tabular}{ l c r }
  1 & 2 & 3 \\
  4 & 5 & 6 \\
  7 & 8 & 9 \\
\end{tabular}
\label{tab:user study}
\caption{Questionnaire results}
\end{table}

We also observe that simplicity in Casanova networking comes from a specific factor. The number of primitives offered is really small. Four keywords for sending and receiving, and three keywords for specifying ownership and connections. A total of seven keywords is really a small set for something as big as building multi-player games. Existing libraries, when compared to Casanova networking, are far larger, at least an order of magnitude, when counting the number of primitives they offer. In this case, we counted non-trivial methods as primitives. The results of this comparison are summarized in Table \ref{tab:networking libraries}.

\begin{table}
\center
\begin{tabular}{ l c r }
  1 & 2 & 3 \\
  4 & 5 & 6 \\
  7 & 8 & 9 \\
\end{tabular}
\label{tab:networking libraries}
\caption{Comparison with other libraries}
\end{table}

\subsection{Robustness}
The ability of a game to function even in the presence of network issues or client disconnections is a very important issue. Wireless or mobile networks are unreliable, the Internet loses many messages that are sent with faster UDP connections, and game instances are often (purposefully or by accident) disconnected from the game.

Most of the networking primitives of Casanova simply never fail. Only \texttt{send\_reliably}, on the other hand, blocks a rule execution until the networking operation is not completed. Casanova offers a rather aggressive, but reasonable, behaviour in the case of such failures (which have a large enough timeout to ensure no premature response!): the unreachable party will be forcibly disconnected from the game, and the other will have a chance to abort execution of the rule because at that point \texttt{send\_reliably} will return \texttt{false}. Of course it is possible to add manual handling of these failures, in order to avoid flat-out disconnections, for example by writing:

\begin{lstlisting}
{
  send_reliably<T>(x)
  ...
} || { 
  wait(timeout)
  ...
}
\end{lstlisting}

In this case we are requiring that all operations are completed within a reasonable time-out, or else we locally abort rule execution without forcing the other party to disconnect in case of failed transmission.

Finally, when a peer disconnects, then Casanova automatically removes it from the list of peers. This guarantees that we will not send or receive anything to and from this peer any more. Disconnection may result in \textit{orphaned entities}, which are entities that are only stored as slaves in all available instances. The orphaned entities can be handled with a brief voting session, where the first player to claim the entity will assign it to himself and signal to the others that they must connect to the new master entity. Alternatively, the various instances may be programmed to simply discard the orphaned entities, for example the avatar of a disconnected player.

\subsection{CPU usage}
The overhead of the Casanova networking libraries is rather small. As a benchmark, we have run a dummy instance of a game where all networked primitives immediately return a predefined value, instead of performing any actual networking operation. The difference in time for a single tick of the simulation was, on average, quite small. The results are summarized in Table \ref{tab:cpu usage}.

\begin{table}
\center
\begin{tabular}{ l c r }
  1 & 2 & 3 \\
  4 & 5 & 6 \\
  7 & 8 & 9 \\
\end{tabular}
\label{tab:cpu usage}
\caption{CPU usage}
\end{table}

\subsection{Bandwidth and latency limitations}
Bandwidth usage is a difficult aspect to assess, as it is heavily dependent on implementation details. For example, an application that often relies on \texttt{send<T>} for entities \texttt{T} which contain large amounts of data is going to use more bandwidth than an application that does the same less often. In turn, an application that only sends primitive values such as \texttt{send<bool>}, \texttt{send<int>}, and \texttt{send<float>} and which does so seldom, will use even less bandwidth. In short, depending on the requirements of the application, for example running inside a fast LAN or across the Internet, developers may choose a heavier or more optimized approach.

We show that it is \textit{possible} to reach very limited bandwidth usage. We built the samples \cite{CNV_MULTIPLAYER_SAMPLES} with limited bandwidth in mind. Usage is well within the suggested maximum average bandwidth required for modern games. Bandwidth usage also heavily influences latency. Few transmissions, uncongested, result in lower latency. The results are summarized in Table \ref{tab:bandwidth and latency}.

\begin{table}
\center
\begin{tabular}{ l c r }
  1 & 2 & 3 \\
  4 & 5 & 6 \\
  7 & 8 & 9 \\
\end{tabular}
\label{tab:bandwidth and latency}
\caption{Bandwidth and latency}
\end{table}
