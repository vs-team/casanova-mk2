Computer and video games have a large market which has grown to the point that the sales are higher than those of the music and film business \cite{ESAreport}. Sales of mobile games alone, have been predicted to exceed 100 billions of dollars in 2017 \cite{GLOBAL:GAMES:INVESTMENT:REVIEW:2014}. The adoption of games has changed to the point that they are not used only in the entertainment business, but they have also found applications in education, training, research, social interaction, and even raising awareness. These so-called \textit{serious games} are used in schools, hospitals, industries, and by the military and the government. Researchers use games to simulate and evaluate their results. Examples of successful serious games are: Microsoft Flight Simulator, VBS1, Second Life, and Phylo \cite{seriousgameslist}. These games do not enjoy the same rich market of entertainment games, but their social impact is nevertheless high.

The core of a computer game is composed by the \textit{state}, which is a data structure storing all the information about the simulation, and the \textit{game loop}, that is a function which loops indefinitely and calls two procedures: \textit{(i)} the update function which updates the game state according to the \emph{logic} of the game, and \textit{(ii)} the draw function which draws the scene. Each iteration of the game loop is called \textit{tick} or \textit{frame}.

The update function usually updates the state in two different ways: \textit{(i)} continuous dynamics, describing the physics behaviour of an entity, and \textit{(ii)} discrete dynamics, when the update is not made at any instant but just when some conditions are met, often describing interactions between characters (i.e. their behaviour), the \textit{Artificial Intelligence}, events happening in the game world.

Discrete dynamics requires complex concurrent, nested, parallel, interruptible programs. These programs are typically hand-crafted, with little to none language support. This is a costly (and error-prone) activity.
Thus, we argue that there is a need for language constructs that are capable of expressing the discrete dynamics of game logic while , at the same time, maintaining fast runtime. The aim of our work is the definition of such constructs and their optimization, taking advantage of the ``suspensive'' nature of discrete dynamics.

\paragraph{Problem statement}
In this paper we will focus on two main tasks centred around discrete dynamics.
\begin{enumerate}
\item Define a series of concurrent operators to describe discrete dynamics in games.
\item Define an optimized concrete semantics for these operators to ensure their runtime costs are acceptable for \textit{Real Time Games}.
\end{enumerate}

We will show that the use of these operators reduces the effort of building, maintaining, and optimizing discrete dynamics code employed in computer games. Our work focuses on tools and techniques to reduce complexity and improve performance. Developers are only asked to focus on the definition of the game with domain-appropriate operators, making it readable and leaving optimizations to the compiler.

\subsection{Structure of the paper}
We start with a discussion about games and their complexity, Through the use of a case study (Section \ref{sec:problem_statement}), we identify the issues of discrete dynamics implementation in traditional languages. We show the operators and the main concept behind Casanova 2 (Section \ref{sec:casanova}), a declarative-imperative programming language oriented to video game development. We introduce the idea of \emph{lazy waiting}, a hybrid static-dynamic analysis technique which allows us to optimize our operators runtime costs. (Section \ref{sec:idea}). We explain in details the syntax and semantics of our operators and provide a detail explanation of the implementation of lazy waiting (Section \ref{sec:details}). We show the effectiveness of our approach with the evaluation, by giving an implementation of our case study using our optimized operators in Casanova 2, and comparing it with alternative implementations (Section \ref{sec:evaluation}).