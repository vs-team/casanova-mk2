%%%%%%%%%%%%%%%%%%%%%%%%%%%%%%%%%%%%%%%%%%%%%%%%%%%%%%%%%%
% conclusions.tex
%%%%%%%%%%%%%%%%%%%%%%%%%%%%%%%%%%%%%%%%%%%%%%%%%%%%%%%%%%

In this paper we have presented the design of the Casanova language, a hybrid declarative/procedural language for making games. The language has a triple focus on simplicity and correctness (to increase developer productivity, given the complexity of game development) and performance (to ensure high framerates).

We have defined a model that generalizes an abstract game, and we have introduced four important properties that describe a good game. We have shown how the Casanova language respects these properties, that is:

\begin{itemize}
\item rules are applied exactly once for each entity
\item rules are order-independent
\item ticks always terminate
\item automated optimizations ensure fast execution
\end{itemize}

Our first goal is to implement a fully working prototype of the Casanova compiler that outputs F\# code. The compiler is still in its very early stages, and a lot of work is still needed to achieve this goal.

Further (and less obvious) improvements may be adding support for rendering, networking and (fully or partially) automated AI. Another venue that we are investigating is the support for reusable libraries of ready-made components, possibly with some form of statically resolved polymorphism (maybe similar to Haskell type classes) for performance reasons. Integration with an existing IDE (such as MonoDevelop or Visual Studio) is an important addition to a modern language. Finally, addressing the problem of generating less garbage (especially for the XBox 360 and for other platforms such as Windows Phone 7 and iOS through Mono) is another of our objectives.

As a final remark, some aspects of Casanova (namely scripting) already have been fully implemented, as described in \cite{FRIENDLY_FSHARP}.