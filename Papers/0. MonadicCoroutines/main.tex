%\def\newblock{\hskip .11em plus .33em minus .07em}

\documentclass{llncs}
\usepackage{graphicx}
\sloppy

\usepackage{amssymb}
\usepackage{amsmath}
\usepackage{graphicx}
\usepackage{epsfig}
\usepackage{subfigure}
\usepackage{listings}
\usepackage{verbatim}
\usepackage[T1]{fontenc} 
\usepackage{url}
\lstset{language=ml}
\lstset{commentstyle=\textit}
\lstset{mathescape=true}
\lstset{backgroundcolor=,rulecolor=}
\lstset{frame=single}
\lstset{breaklines=true}
\lstset{basicstyle=\ttfamily \small}

\newcommand{\Nat}{{\mathbb N}}
\newcommand{\Real}{{\mathbb R}}

\begin{document}

\title{Monadic Scripting in F\# for Computer Games}

\author{
G. Maggiore, \ M. Bugliesi, \ R. Orsini
  \institute{Universit\`a Ca' Foscari Venezia \\
  \email{\{maggiore,bugliesi,orsini\}@dais.unive.it}
}
}

\maketitle

\date{}

\begin{abstract}
Modern game architectures provide for a clean separation between
game engine and game content, making engines \textit{scriptable}, so
as to delegate the development of the main functionalities related to
gameplay to scripts coded in high-level languages. 

Game scripting poses two orders of problems. First, the scripting
language must be equipped with coroutining mechanisms to support a
smooth interaction between the discrete-time structure of the game
animation implemented by the engine with the execution of the
scripts, which typically implement character behavior spanning 
multiple simulation ticks. Secondly, these mechanisms should be transparent, to
ease design and development, and at the same time offer the highest
run-time performance to keep up with the interactive frameratess
expected by the user. 

We present a monadic framework that elegantly solves these problems
and compare it with the most commonly used systems, and discuss how it has
been integrated in actual projects. 
\end{abstract}

\keywords{games, monadic programming, state management, scripting}

\section{Introduction}
\label{sec:intro}
%----------------------------------------------------------------------------
%  intro.tex 
%----------------------------------------------------------------------------
Modern computer languages are very reliable when it comes to writing a large class of common, real-world applications. For example, relatively simple form applications or web sites can be built extremely easily in languages such as Java, C\# and many others. This is thanks to commonplace facilities like garbage collectors, classes and inheritance and large libraries which simplify many tasks which otherwise would be hard or error-prone. On the other hand, there is a not so small set of applications for which these languages do not perform even nearly as well; for example games, even though very powerful libraries such as XNA make them easier to write by encapsulating many useful patterns, are not so suitable for modern languages. For this reason most games are still written in C++ (sometimes even in C) and the transition to higher level languages is not happening as fast as it could. As another example we could consider mobile applications. The widespread adoption of very powerful, fully programmable smartphone like the iPhone, Google Android or Windows Phone 7 makes performance even more important to achieve: lighter applications mean much better applications where CPU cycles and battery are both scarce resources. On the other hand, to allow as many developers as possible to easily create applications for these platforms, it makes sense (as indeed it is happening) to allow programming these devices with as languages that are as high-level as possible. Finally, there are many real time or soft real time applications that migh benefit from using high level languages but which cannot afford the pauses or slowdowns that sometimes the garbage collector might require, especially on less powerful hardware.


In this paper we document the results of implementing a videogame in such a high-level language (F\#) for a mobile device. Early in our development cycle we discovered that the biggest problem of our application was that by continually allocating and deallocating instances of the same types (such as projectiles and particles) we kept triggering the garbage collector, forcing the application to a crawl everytime it started. Rather than implement object pooling, we decided to try and generalize this technique by implementing reference counting \cite{7_6} inside the state monad \cite{1_1}. The state monad contains and manages all the actual instances of the objects which require reference counting, and references to these objects can only be accessed through the state monad itself. This allowed us to track the lifetime of our entities, in a lightweight enough fashion to ensure the increase in performance that we needed and transparently enough that it became convenient to track other resource types as well, such as files or GPU memory which require manual disposal.


In this paper we discuss a possible generalization of the work described above, where in addition to a monadic reference counting system we also discuss how type-level meta-programming \cite{3_1,3_2,3_3,6_1} and the parameterized monad \cite{1_7} can be used to track the types of resources with a strongly typed heterogeneous list \cite{4_1}, thereby removing even the need for the developer to "discover" what type the state must have. Also, this system could be the stepping stone for a more refined state-tracking monad which uses type-level values and phantom types to track expected properties of the state monad at various points in the program.


An important note is about the computer language in which we have written the samples. We have used a pseudo-Haskell, and we believe our listings may be turned into a working program without too much effort. This notwithstanding, it must be noted that said effort has not been made by us: our software system is mostly interested in performance, and from the point of view of benchmarking performance Haskell may not be the best suited language given its lazy evaluation strategy. For us this pseudo-Haskell has acted as a clear and easily implementable specification, and we do not claim anything about its workability in any present or future versions of the language.
 
 
% 0 pages

\section{Game Engine Architectures}
\label{sec:scripting_in_games}
%%%%%%%%%%%%%%%%%%%%%%%%%%%%%%%%%%%%%%%%%%%%%%%%%%%%%%%%%%
% scripting_in_games.tex
%%%%%%%%%%%%%%%%%%%%%%%%%%%%%%%%%%%%%%%%%%%%%%%%%%%%%%%%%%

In this section we briefly illustrate a (heavily) simplified game architecture in order to describe the main problems that a scripting system faces when introduced inside a game engine.

A game engine is based on three fundamental components: 
({\em i}) the game state, which is a snapshot of the game world
and includes a description of all the various entities of the game; 
({\em ii}) the update  function, which computes the next value of the
game state, and ({\em iii}) the draw/rendering function, which draws
the game state to the screen.

The game loop is the code that defines how the game is run; it is a
recursive function that continuously invokes the update and draw
functions with the current state as input. The game loop also computes
the time delta between iterations, so that the update function will be
able to adjust its computations to cover the actual amount of time
elapsed between its invocations: 

\begin{lstlisting}
let run_game (game:Game) t =
  let t' = get_time()
  do game.Update (game.State, t'-t, t)
  do game.Draw (game.State)
  do run_game game t'
\end{lstlisting}

Central to our present concerns is the update function, which
implements all the functionalities that modify the game state. As 
discussed earlier, most of these functionalities, typically the
physics of the various entities, such as forces, 
collision detection, $\dots$ etc, the interaction with the
input/output and other devices are coded in low-level languages such
as C or C++ to guarantee the fast framerates  needed for a smooth play
experience. On the other hand, higher-level aspects of the game,
related to gameplay, are typically left outside the code of the 
update function, and made scriptable.  

The most important function of the scripts is to model the behaviors
of the computer characters and of the other in-game objects. To
illustrate, the following pseudo-code describes the behavior of  
a prince in a Role Playing Game: 

\begin{lstlisting}
prince:
  princess = find_nearest_princess()
  walk_to(princess)
  save(princess)
  take_to_castle(princess)
\end{lstlisting}

The main problem in coding this behavior with a script is to achieve a
smooth interaction between the discrete-time structure of the game
animation implemented by the simulation engine, and the behavior
implemented by the script, which spans multiple time slots of the
simulation engine. Specifically, in order to guarantee a smooth user
experience, each such script must be interruptible, so that at 
each discrete step of the simulation engine the script performs a
finite number of transitions and then suspend itself: failing to do 
so would slow down the simulation steps, hence the resulting framerate
of the game would decrease, thereby reducing the player immersion. 

The problem is sometimes addressed by coding scripts as state machines
(SMs), whose execution gets interrupted at each state transition. 
However, while SMs represent a viable design choice for simple
scripts, they are far less effective for modelling objects with
complex behavior, as their structure grows easily out of control and
becomes rather hard to maintain. Modern scripting languages adopt 
\textit{coroutines} as a mechanism to build state machines
implicitly, by way of their (the coroutines') built-in mechanisms to
suspend and resume execution. With coroutines the 
code for a SM is written ``linearly'' one statement after another, but
each action may suspend itself (an operation often called ``yield'')
many times before completing. The local state of the state machine
is stored in the continuation of the coroutine. Some of the most used 
scripting languages, which are Lua, Python and C\#, all offer some
suspension mechanisms similar to coroutines that game developers use
for scripting; for a detailed discussion of couroutines in these
languages, see \cite{PYTHON_COROUTINES,LUA_COROUTINES,CSHARP_YIELD}. 

\begin{comment}
\subsection{Coroutines in action}
In the remainder of this section we analyze coroutines in action in Lua. We will also briefly discuss how coroutines are emulated in Python and C\# with generators. For a more detailed discussion of the mechanisms of coroutines in Lua, Python and C\# see \cite{PYTHON_COROUTINES,LUA_COROUTINES,CSHARP_YIELD}.

Lua coroutines are based on the three functions \texttt{coroutine.yield, coroutine.resume} and \texttt{coroutine.create} which respectively pause execution of a coroutine, resume execution of a paused coroutine and create a coroutine from a function.

\begin{lstlisting}
function walk_to(self,target)
  return coroutine.create(
    function()
      while(dist(self,target) > self.reach) do
        self.Velocity = towards(self, target)
        coroutine.yield()
      end
    end)
end

...

function prince(self)
  return coroutine.create(
    function()
      princess = bind_co(find_nearest_princess(self))
      bind_co(walk_to(self, princess))
      bind_co(save(self, princess))
      bind_co(take_to_castle(self, princess))
    end)
end
\end{lstlisting}

Notice that to invoke a coroutine we need to explicitly bind it with the \texttt{bind\_co} function (we do not show here the source for reasons of space), which keeps resuming a coroutine until it yields for the last time, and then it returns the resulting value.

A mechanism to implement coroutines in Python makes use of generators. A generator is a special routine that returns a sequence of values. However, instead of building an array containing all the values and returning them all at once, a generator yields the values one at a time; yielding effectively suspends the execution of the generator until the next element of the sequence is requested by the caller. Python generators may appear as a way to return lazy sequences but they are powerful enough to implement coroutines. We can adopt the convention that a coroutine is actually a generator which yields a sequence of null (\texttt{None}) values until it is ready to return; the returned value will be yielded last.

\begin{lstlisting}
def walk_to(self,target):
  while(dist(self,target) > self.Reach):
    self.Velocity = towards(self,target)
    yield

...

def prince(self):
  for princess in find_nearest_princess(self):
    yield
  for x in walk_to(self,princess):
    yield
  for x in save(self,princess):
    yield
  for x in take_to_castle(self,princess):
    yield
\end{lstlisting}

As in Python, C\# supports generators. Since C\# is statically typed, we need to assign a type to our coroutines. We have two alternatives; a coroutine that returns nothing (\texttt{void}) has type \texttt{IEnumerable}, that is it returns a sequence of \texttt{Object}s that are all null (a similar strategy is used by Unity, even though with unsafe casts \cite{UNITY_YIELD}) and we can type a coroutine that returns a value of type \texttt{T} as \texttt{IEnumerable<T?>}, where \texttt{T?} is either \texttt{null} or an instance of \texttt{T}.

We omit the C\# sample for brevity, and also because of its similarity with Python. Moreover, when compared with coroutines, generators are quite cumbersome in a scripting language and indeed LUA is by far more used in games.

In the remainder of the paper we will present a different approach to coroutines, namely building a meta-programming abstraction (called \textit{monad}) to implement coroutines in F\#. We will discuss how our approach produces code which is faster and shorter than similar implementations in Lua, Python and C\#. We will also discuss how our approach is very customizable, thanks to the fact that coroutines are not \textit{wired} into the language runtime but rather we have defined them with our monad. Monads simplify the use of coroutines, making them completely invisibile to the user. Also, thanks to type inference the resulting scripts require no typing annotations. 
Finally (see Section \ref{sec:benchmarks} for the details), our system offers a good runtime performance and is type safe; this makes it suitable for large and complex scripts.
\end{comment}
 
% 6 pages

\section{The Script Monad}
\label{sec:script_monad}
%%%%%%%%%%%%%%%%%%%%%%%%%%%%%%%%%%%%%%%%%%%%%%%%%%%%%%%%%%
% script_monad.tex
%%%%%%%%%%%%%%%%%%%%%%%%%%%%%%%%%%%%%%%%%%%%%%%%%%%%%%%%%%

Monads can be used for many purposes \cite{MOGGI_MON,DECL_IMP,COMPR_MON,EFF_MON,CSHARP_ASYNC,CSHARP_LINQ}. Indeed, monads allow us to
overload the bind operator, in order to define exactly what happens
when we bind an expression to a name.  We will use this capability of monads to implement a DSL for coroutines that allows to chain coroutines together with the binding operator. The monad we define will \textit{suspend} itself at every bind and return its continuation as a lambda. The monad type is \texttt{Script}: 

\begin{lstlisting}
type Script<'a,'s> = 's -> Step<'a,'s>
 and Step<'a,'s> = Done of 'a | Next of Script<'a,'s>
\end{lstlisting}

Notice that the signature is very similar to that of the regular
state monad, but rather than returning a result of type $\alpha$
it returns either \texttt{Done\ of} $\alpha$ or the continuation 
\texttt{Next\ of\ Script<}$\alpha,\sigma$\texttt{>}. The continuation stores, in its closure, the local state of a suspended script. Our monad allows us to access the state which will be passed by the game engine; this way our scripts will be able to read, write or modify the main state of the game to interact with the processing performed by the game engine.

Returning a result in this monad is simple: we just wrap it in the
$\mathtt{Done}$ constructor since obtaining this value  requires no
actual computation steps. Binding together two statements is more complex. 
We try executing the first statement; if the result is
$\mathtt{Done\ x}$, then we return
$\mathtt{k\ x\ s}$, that is we perform the binding and we 
continue with the rest of the program with the result of the first
statement plugged in it. If the result is \texttt{Next\ p'}, then we cannot
yet invoke $\mathtt{k}$. This means
that we have to bind \texttt{p'} to \texttt{k}, so that at the next
execution step we will continue the execution of \texttt{p} from where
it stopped.

\begin{lstlisting}
type ScriptBuilder() = 
  member this.Bind(p:Script<'a,'s>, k:'a->Script<'b,'s>) 
    : Script<'b,'s> =
    fun s ->
      match p s with
      | Done x -> k x s
      | Next p' -> Next(this.Bind(p',k))

  member this.Return(x:'a) : Script<'a,'s> = fun s -> Done x

let script = ScriptBuilder()
\end{lstlisting}

We now define the \texttt{get\_state} coroutine that extracts the current state:

\begin{lstlisting}
let get_state : Script<'s,'s> = fun s -> Done s
\end{lstlisting}

We also define another coroutine that forces a suspension:

\begin{lstlisting}
let suspend :Script<Unit,'s> = fun s -> Next(fun s -> Done())
\end{lstlisting}

We assume the standard F\# convention that \texttt{let! x = script1 in script2} is translated into \texttt{Bind(script1, fun x -> script2)} and \texttt{return x} is translated into \texttt{Return(x)}.

Let us now see a small, self-contained example of our scripting system in action. Coroutines can be used in many ways to achieve various results; what we are more interested in, is using coroutines as a means to perform long and complex computations asynchronously inside the main loop of an application. We wish to build an application that computes a very large Fibonacci number, but does so while continuously writing on the console that it is still alive and responsive. This application has no shared state, and so all our coroutines will have type \texttt{Script<'a,Unit>}.

The coroutine version of the Fibonacci function is very similar to a regular implementation of the Fibonacci function, with the only difference that we use monadic binding to recursively invoke the function itself. Each time we recurse, the coroutine suspends:

\begin{lstlisting}
let rec fibonacci n : Script<int,Unit> = 
  script{
    match n with
    | 0 -> return 0
    | 1 -> return 1
    | n -> 
      do! suspend
      let! n1 = fibonacci (n-1)
      let! n2 = fibonacci (n-2)
      return n1+n2 
  }
\end{lstlisting}

Running the Fibonacci function now requires many steps of our scripting monad; for this reason we can safely invoke this function with the knowledge that it will run for a short time before returning either the final result with \texttt{Done} or its continuation with \texttt{Next}. We can define the main loop of our application as follows:

\begin{lstlisting}
let main_loop() =
  let rec main_loop (f:Script<int,Unit>) = 
    do printf "I am alive.\n"
    match f () with
    | Done result -> 
      do printf "The result is %d\n" result
    | Next f' ->
      main_loop f'
  do main_loop (fibonacci 1000000) 
\end{lstlisting}

The main loop above can be seen as a simplification of the game loop we have seen in Section \ref{sec:intro}. Of course in a game the state datatype would be more complex than just \texttt{Unit}, and the application would perform its full update and draw instead of just printing a string on the screen. This said, adding the above pattern matching to each iteration of the main loop of the game is all that is required to integrate our scripting system with an existing game engine.

 
% 3 pages

\section{A DSL for Scripting}
\label{sec:script_combinators}
%%%%%%%%%%%%%%%%%%%%%%%%%%%%%%%%%%%%%%%%%%%%%%%%%%%%%%%%%%
% script_monad_combinators.tex
%%%%%%%%%%%%%%%%%%%%%%%%%%%%%%%%%%%%%%%%%%%%%%%%%%%%%%%%%%

The script monad is the runtime core of our DSL. A DSL can be augmented by defining a series of operators that automate or simplify common operations for the DSL developers. The best set of operators for a scripting DSL is strongly dependent upon the kind of game to be scripted. In this section we describe a general-purpose set of operators that make up a basic calculus of coroutines, but we would expect that other game developers would define additional operators that are a tighter fit to their games. The operators of our calculus of coroutines take as input one or more coroutines and return as output a new coroutine:

\begin{itemize}
\item \texttt{parallel} ($s_1 \wedge s_2$) executes two scripts in parallel and returns both results
\item \texttt{concurrent} ($s_1 \vee s_2$) executes two scripts concurrently and returns the result of the first to terminate
\item \texttt{guard} ($s_1 \Rightarrow s_2$) executes and returns the result of a script only when another script evaluates to \texttt{true}
\item \texttt{repeat} ($\uparrow s$) keeps executing a script over and over
\item \texttt{atomic} ($\downarrow s$) forces a script to run in a single tick of the \textit{discrete simulation engine}
\end{itemize}

We show here the implementation of one of these combinators with our monadic system. To see the other implementations, see \cite{FRIENDLY_FSHARP}.

\begin{lstlisting}
let rec parallel_ (s1:Script<'a>) (s2:Script<'b>) 
          : Script<'a * 'b> =
  fun s ->
    match s1 s,s2 s with
    | Done x, Done y    -> Done (x,y)
    | Next k1, Next k2  -> parallel_ k1 k2
    | Next k1, Done y   -> parallel_ k1 (fun s -> Done y)
    | Done x, Next k2   -> parallel_ (fun s -> Done x) k2
\end{lstlisting}


We can now give another self-contained example that shows a producer and a consumer running in parallel. We start by defining the state as a single memory location, which can either be empty (\texttt{None}) or contain a value (\texttt{Some x}):

\begin{lstlisting}
type Buffer = { mutable Contents : Option<int> }
let buffer = { Contents = None }
\end{lstlisting}

We define some additional helper functions to access the state. A good engineering rule is that the more complex is the state, the less coroutines directly use the \texttt{get\_state} function; rather, when the state is complex, we define a series of additional accessor functions that help us manipulating the various aspects of the state:

\begin{lstlisting}
let set_buffer v : Script<Unit,Buffer> = 
  script{
    let! s = get_state
    s.Contents <- Some v }
let is_buffer_empty : Script<bool,Buffer> = 
  script{
    let! s = get_state
    return s.Contents = None }
\end{lstlisting}

We define the producer (the consumer is symmetric) as follows; notice that each access to the state automatically suspends the coroutine, so we do not need to explicitly invoke \texttt{suspend}:

\begin{lstlisting}
let rec wait_buffer_empty:Script<Unit,Buffer> = 
  script{
    let! c = is_buffer_empty
    if c = false then
      do! suspend
      do! wait_buffer_empty }

let producer =
  let rec producer i : Script<Unit,Buffer> =
    script{
      do! wait_buffer_empty
      do! set_buffer i
      do! producer (i+1) }
\end{lstlisting}

The main loop is very similar to the main loop seen for the Fibonacci sample above; the only difference is that we invoke the producer and the consumer coroutines in parallel, by passing to the inner \texttt{main\_loop} function the value \texttt{parallel\_ producer consumer}.


\subsection{Scripting in Games}

We now discuss the introduction of our scripting system in games. In the following we outline how we have built most of the game logic of the upcoming RTS game Galaxy Wars (which is also released with a fully open source \cite{GALAXY_WARS}). In the game we are considering the players compete to conquer a series of \textit{star systems} by sending fleets to reinforce their systems or to conquer the opponent's.

\subsubsection{Game Patterns}

Thanks to our general combinators we can define a small set of recurring game patterns; by instantiating these game patterns one can build the actual game scripts with great ease. These patterns may be adapted for the specific domain of a game, or altogether new patterns may be created that better fit one's game. 
The first game pattern we see is the most general, and for this reason it is called \texttt{game\_pattern}. This pattern initializes the game in a single tick, then performs a game logic (while the game is not over) and finally it performs the ending operation before returning some result. The initialization is performed by the \texttt{init} script, which returns a result of a generic type \texttt{'a}; this result is the state of the script, and contains data that may be helpful for tracking additional information that is useful to our scripts but which is not stored in the game state. The logic of the various game entities, such as their AI, is then performed, repeatedly, by the \texttt{logic} script. While the logic script is run, the \texttt{game\_over} script continuously checks to see if the game has been won or lost and thus must be terminated; when the termination condition is met, the \texttt{ending} script is invoked that may show some recap of the game that has just ended. The game pattern is implemented as follows:

\begin{lstlisting}
let game_pattern 
      (init:Script<'a>) 
      (game_over:'a -> Script<'bool>) 
      (logic:'a -> Script<Unit>) 
      (ending:'a -> Script<'c>) 
      : Script<'c> =
  script{
    let! x = init |> atomic_
    let! (Left y) = 
       concurrent_ 
         (guard_ (game_over x) (ending x)) 
         (logic x |> repeat_)
    return y }
\end{lstlisting}

Notice that with the introduction of appropriate operators we may remove many of the parentheses of the above sample if they are seen as a hindrance to readability. The main portion of the code above may then be rewritten as:

\begin{lstlisting}
(game_over x => ending x) .|| (logic x |> repeat_)
\end{lstlisting}

The game pattern above is very general, but not all scripts always need all of its parameters. We can build less general game patterns by reducing the number of parameters; for example, we may build a game pattern that has no initialization, logic or ending sequence; such a game pattern would implement the case of a game script whose sole responsibility is to check the termination conditions for a game (those that trigger the ``game over'' screen):

\begin{lstlisting}
let wait_game_over (game_over:Script<bool>) : Script<Unit> = 
  let null_script = script{ return () }
  game_pattern 
    null_script
    (fun () -> game_over)
    (fun () -> null_script)
    (fun () -> null_script)
\end{lstlisting}

Writing a script with our system will consist of instantiating one game pattern with specialized scripts as its parameters; these scripts will alternate accesses to the specific state of the game with invocations of combinators from the calculus seen above. In the next session we will see an example of this.
 
% 5 pages

\label{sec:script_monad_case_study}
%%%%%%%%%%%%%%%%%%%%%%%%%%%%%%%%%%%%%%%%%%%%%%%%%%%%%%%%%%
% script_monad_case_study.tex
%%%%%%%%%%%%%%%%%%%%%%%%%%%%%%%%%%%%%%%%%%%%%%%%%%%%%%%%%%

\subsubsection{A sample script}

The state of the game contains a series of star systems, fleets, players and various other information:

\begin{lstlisting}
type GameState = {
    StarSystems : StarSystem List
    Fleets      : Fleet List
    ... }
\end{lstlisting}

We define a series of accessor functions that allow a coroutine to access the current state of the game; such an accessor function, for example, is the \texttt{get\_fleets} that returns the list of active fleets:

\begin{lstlisting}
let get_fleets = 
  script{
    let! state = get_state
    return state.Fleets }
\end{lstlisting}

The basic mode of the game uses our scripting system to determine the winner of the game; as long as there is more than one player standing, the script waits. This script computes the union of the set of active fleet owners with the set of system owners:  

\begin{lstlisting}
let alive_players_set = 
  script{
    let! fs = get_fleets
    let fleet_owners = 
             fs |> Seq.map (fun f -> f.Owner) 
             |> Set.ofSeq
    let! ss = get_systems
    let system_owners = 
             ss |> Seq.map (fun s -> s.Owner) 
             |> Set.ofSeq
    return fleet_owners + system_owners }

let game_over =
  script{
    let! alive_players = alive_players_set
    let num_alive_players = alive_players |> Seq.length
    return num_alive_players = 1 }
\end{lstlisting}

The main task of our script is to wait until the set of active players has exactly one element; when this happens, that player is returned as the winner: 

\begin{lstlisting}
let basic_game_mode = wait_game_over game_over
\end{lstlisting}

The two short snippets above are all there is to the main game mode.

Variations of the game are soccer (one system acts as the ball which can be moved around), capture the flag, siege and others. We omit a detailed discussion of the other variations for reasons of space; the important thing to realize is that all of these variations have been implemented with the same simplicity of the scripts above, by instancing one game pattern with appropriate scripts which are built with a mix of combinators interspersed with accesses to the game state.

\subsubsection{Input Management}

Another large subsystem where we have used our scripting system is input management. Input is divided into a series of pairs of scripts; each pair of scripts is separated by a guard: the first script performs an event detection, while the second performs an event response. Each pair of scripts is repeated forever, in parallel with all other scripts.

As an example, consider the following script that decides whether to launch ships or not against a target:
\begin{lstlisting}
let input world = 
  repeat_(
  script{ 
    if mouse_clicked_left() then
      let mouse = mouse_position()
      // find closest planet under the cursor
      let clicked:Option<Planet> = ... 
      return Some(clicked) } => 
    fun p -> script{ world.SourcePlanet := Some(p) }) .||
  repeat_(
  script{ 
    if mouse_clicked_right() && 
       world.SourcePlanet <> None then
      let mouse = mouse_position()
      // find closest planet under the cursor
      let clicked:Option<Planet> = ... 
      match clicked with 
      | Some planet -> 
        return Some(planet,world.SourcePlanet.Value)
      | None -> return None } => 
    fun (source,target) -> 
      script{ mk_fleet world source target })
\end{lstlisting}

A distinct advantage of this technique is that it allows us to cleanly separate the code that reads the actual user input from the code that performs something meaningful on the game world with this input. By parametrizing the code above with respect to the input detection scripts we could make it possible to support different controller types (game pad, touch panel, mouse, keyboard, etc.).

\subsubsection{Further uses}

Since our scripting system proved effective in the various areas where we tried it, and given that its performance is fully satisfactory, we have decided to use it more pervasively all over the game. The result is that each entity (planets, ships, players, etc.) has a large chunk of its game logic moved from the update loop into appropriate scripts.

Our menu system is heavily based on our scripts (this proved especially useful when implementing the multi-player lobby, where the regular menu (buttons, input detection, etc.) had to be run interleaved with the scripts that synchronize the list of players across the network.

Finally, we have used scripts for the entire networking system, given that many operations such as connections and time-outs require timers and many parallel operations. 
% 2 pages

\section{Benchmarks}
\label{sec:benchmarks}
%%%%%%%%%%%%%%%%%%%%%%%%%%%%%%%%%%%%%%%%%%%%%%%%%%%%%%%%%%
% benchmarks.tex
%%%%%%%%%%%%%%%%%%%%%%%%%%%%%%%%%%%%%%%%%%%%%%%%%%%%%%%%%%

Our system is mainly concerned with optimizing away the overhead that dynamically building and maintaining an X3D scene produces. To show that we have achieved our objective of increasing performance in X3D scenes with only regular nodes and routes (we have not yet profiled external scripts extensively), we have tested the same scene on multiple browsers and profiled the resulting framerates. The browsers we have used are BS Contact and Octaga.

We have tested for scenes with a relatively low number of shapes (300 and 680). We are not really interested in testing the rendering performance, since such a test would mainly compare the efficiency of the underlying rendering APIs and would not be relevant in this context. Both scenes are compared against two other scenes with the same shapes but with 3 \texttt{color interpolators}, 2 \texttt{timers} and 6 \texttt{routes} for each shape. The resulting routing and logic are quite heavy and constitutes a good test the underlying execution model for routes and logical nodes. The tested X3D files are an advanced version of the example seen in Section \ref{sec:case_study}: there is a (rather large) set of shapes, colors and timers and the colors of the shapes are changed according to the timers through heavy use of routes. This benchmark shows how heavy the traditional dynamic model is when handling many routes and large scenes.

Tables 1 through 3 show a comparison in performance for each browser with various hardware configurations; we have compared the performance of our implementation against Octaga and BS Contact, apart from the second testing machine where Octaga had trouble installing and then running at a reasonable speed (to avoid polluting the results we have omitted Octaga from that test):

\begin{table}[htb]\small
\centering
\begin{tabular}{|l|c|c|c|}
\hline
Browser & FPS & FPS (with routes) & Diff \% 	 \\
\hline
\textbf{Test machine 1}  & & & \\
\hline
Ours (300 shapes) & 580 & 510 & -12 \\
Ours (680 shapes) & 265 & 224 & -15 \\
Octaga (300 shapes) &  670 & 340 & -49 \\
Octaga (680 shapes) &  372 & 150 & -60 \\
BS C. (300 shapes) & 370 & 300 & -19 \\
BS C. (680 shapes) & 185 & 145 & -22 \\
\hline
\textbf{Test machine 2}  & & & \\
\hline
Ours (300 shapes) & 670 & 590 & -12 \\
Ours (680 shapes) & 310 & 265 & -15 \\
BS C. (300 shapes) & 530 & 368 & -31 \\
BS C. (680 shapes) & 285 & 146 & -49 \\
\hline
\textbf{Test machine 3}  & & & \\
\hline
Ours (300 shapes) & 640 & 600 & -6 \\
Ours (680 shapes) & 310 & 280 & -10 \\
Octaga (300 shapes) &  720 & 403 & -44 \\
Octaga (680 shapes) &  345 & 181 & -48 \\
BS C. (300 shapes) & 500 & 360 & -28 \\
BS C. (680 shapes) & 215 & 135 & -37 \\
\hline
\end{tabular}
\caption{Test results}
\end{table}

\begin{figure}
\begin{center}
\includegraphics[scale=0.2]{browsers.jpg}
\end{center}
\caption{WP7 Emulator, BS Contact and XNA Windows Application}
\end{figure}

It is clear that thanks to our approach the scene logic weighs far less than it does in the other browsers.

Moreover, as we can see in Figure 3, the code that is generated by our system can be run, \textit{without modification} also in Windows Phone 7 devices; in the figure we can see the same scene run in the Windows Phone 7 emulator (top left), our system (bottom left) and BS Contact (top right). 

Table 4 shows the performance of running, on a Windows Phone 7 device, two compiled scenes with 150 and 300 shapes respectively plus the same routes for each shape used in the tests discussed above. The performance is very good when considered that it is a mobile device; the same technique could be applied to other mobile devices such as iOS or Android, but in our case having used XNA a porting to Windows Phone 7 required literally no effort beyond modifying a compiler switch.

\begin{table}[htb]\small
\centering
\begin{tabular}{|l|c|}
\hline
Scene & FPS 	 \\
\hline
150 shapes with routes & 30 \\
300 shapes with routes & 24 \\
\hline
\end{tabular}
\caption{WP7 (LG Optimus 7)}
\end{table}

At this point we have completed supporting the static aspects of an X3D scene, those that are involved in nodes that are not added or removed dynamically. This approach clearly yields an increase in performance for scenes with a complex logic in terms of timers, routes, interpolators, etc.
 
% 2 pages

\section{A Visual Editor (future work)}
\label{sec:visual_editor}
%%%%%%%%%%%%%%%%%%%%%%%%%%%%%%%%%%%%%%%%%%%%%%%%%%%%%%%%%%
% future work
%%%%%%%%%%%%%%%%%%%%%%%%%%%%%%%%%%%%%%%%%%%%%%%%%%%%%%%%%%

One of our main concerns is that our scripting system as presented so far is aimed at technical users. While this definitely offers advantages, since the game engine itself can be partially coded with scripts to the great benefit of the productivity of developers, the need for less technically inclined users such as designers to access the scripting system is very relevant. After all, designers play a very important role in shaping the behaviors and scripts of a game.

For this reason we are studying a visual editor that allows to create scripts without having to write source code. This editor (which is still a work in progress) would allow the user to drag blocks that represent operations and combinators of our scripts, and assemble them in a fashion that is similar to that of flow-charts.

One of the scripts that we have seen above:

\begin{lstlisting}
(game_over x => ending x) .|| (logic x |> repeat_)
\end{lstlisting}

Would appear in our editor as:

\begin{center}
\includegraphics[scale=0.5]{visual_script.png}
\label{Visual script}
\end{center}
 

\section{Conclusions}
\label{sec:conclusions}
%
% conclusions.tex
%

Innovations in game architecture are often responsible for pushing further the boundaries of game design. By building more powerful \textit{and} more scriptable game engines it is possible at the same time to create games where game logic is faster and more complex, resulting in a more compelling experience for the player.

Building a system that allows for a smarter, individualized AI could usher an era of games that offer a smarter experience where units and non-playing characters make smart choices rather than waiting for ``obvious'' instructions from the player.

It must also be noted that games have been historically responsible for many evolutions of modern personal computing, by pushing the hardware to its limits and by offering powerful visual experiences that years or decades later become paradigms in non-gaming software. For this reason this kind of language evolution must be put in its right context: it is not just a way to make better games, but it is also a prototype for a way to write better \textit{programs} in general.

\paragraph{Future research directions in the field}
From a survey of the current panorama of game development tools and languages it appears that a technological shift is happening: from the almost exclusive use of C and C++ for creating games, plus the odd use of Lua or Python as scripting languages, more and more games are being made in modern managed languages such as C\# or Objective-C. The introduction of XNA (a framework that allows independent game developers to publish commercial games on the XBox 360 without costly licenses) has marked the beginning of the adoption of C\#-based frameworks for making games; other similar frameworks have been used in games between a C++ engine and the scripting system. This shift signals that the game development industry may be starting to have trouble with its old infrastructure given the rising development costs, the shrinking budgets and the decreasing development time and it is looking into new languages with more abstraction to increase its efficiency. For this reason many researchers (the authors of this report being part of this group) are studying new ways to leverage the power coming from the field of functional and declarative languages in order to more easily automate some repetitive programming tasks that often require complex and cumbersome libraries. 
% 1 pages

\vfill\eject

\bibliographystyle{plain}
\bibliography{references} 
% 0 pages

\end{document}
