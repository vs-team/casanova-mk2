Giuseppe Maggiore was born in Mirano (Venice), on 2/3/1985. He attended secondary school at \textit{Liceo Scientifico G. Bruno}, where he received his \textit{Maturità Scientifica} in 2003. During his high school years he discovered Computer Science and game development thanks to the International Olympiads in Informatics. 

To continue learning about Computer Science, Giuseppe attended a B.Sc. in Computer Science at \textit{Università Ca' Foscari di Venezia}. During his bachelor years, he discovered research thanks to the AI and Computer Vision group lead by Prof. Marcello Pelillo at Ca' Foscari. At the same time, he found out the joys and pains of teaching by assisting various courses, most notably Software Engineering with Prof. Agostino Cortesi. In the meantime, Giuseppe joined the ranks of the Microsoft Student Partners, traveling across Italy to speak at various universities about Game Development with Microsoft platforms. This way he could pursue game development, learning new technologies, and applying research (or at least trying to) all together. His thesis was a formal proof on the convergence of discrete replicator dynamics when applied to game theoretical scenarios.

After his B.Sc., Giuseppe went on to pursue his M.Sc. in the same university. During his M.Sc. studies, thanks to a great course in Functional Programming held by Prof. Michele Bugliesi, he found out that programming languages could come in many forms and shapes and that they could have incredible expressive power beyond that of ``regular'' mainstream computer languages. This was the start of a great passion that would drive the next years of his game development efforts. In a fortunate twist of fate, at the same time Microsoft was developing and releasing its F\# functional language, thereby giving Giuseppe an even stronger motivation, now as a consultant for Microsoft, to explore the technology. Giuseppe helped shaping the introductory course to programming by using F\# and tutoring, under the supervision of Prof. Sabina Rossi. His M.Sc. was concluded with a thesis on functional programming and parallel computation on the GPU with Prof, Bugliesi.

The same fascination for games that propelled Giuseppe forward in his studies, now mixed with interest towards programming languages, was alive and kicking even after his M.Sc. This led to the start of a Ph.D. in game development and functional languages under Prof. Bugliesi, and with great help and support from Prof. Renzo Orsini, a veteran of language research. During the ACG 13 conference in Tilburg, Giuseppe met Prof. Pieter Spronck. After a heated discussion on the relevance of Casanova, the two started working together toward the completion of Giuseppe's Ph.D.

While finishing his Ph.D., Giuseppe moved to The Netherlands in order to work closer to Prof. Spronck, in the city of Breda. There, at NHTV University, he found an opening at the prestigious IGAD institute of game development and architecture, where he is currently working as a researcher and a lecturer.

Giuseppe is really curious to see what will happen next in his research and work. But one thing is certain: game development will be a part of it.
